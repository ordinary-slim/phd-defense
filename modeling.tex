\begin{frame}
  \frametitle{Part-scale simulation}
  \framesubtitle{Lumped heat source}
  What do you do if you don't have a cluster or a nice GPU ?
  You \textbf{simplify the model}.

  \begin{figure}
    \centering
    \def\svgwidth{0.5\textwidth}%
    {\small
    \import{./figures/goldakProfile}{profile.pdf_tex}
    }
    \caption{Double ellipsoidal profile \citep{goldak1984}.
      $\xi$ is the welding direction.}
    \label{fig:goldak}
  \end{figure}
\end{frame}

\begin{frame}
  \frametitle{Part-scale simulation}
  \framesubtitle{Lumped heat source}
  \begin{figure}
    \centering
    \begin{tikzpicture}
      % include the original figure
      \node[inner sep=0] (img) {\def\svgwidth{0.5\textwidth}%
        {\small \import{./figures/goldakProfile}{profile.pdf_tex}}};

      % red "forbidden" cross over the whole image
      \draw[line width=2pt, red] ([shift={(-2mm,-2mm)}]img.south west)
                                 -- ([shift={(+2mm,+2mm)}]img.north east);
      \draw[line width=2pt, red] ([shift={(+2mm,-2mm)}]img.south east)
                                 -- ([shift={(-2mm,+2mm)}]img.north west);
    \end{tikzpicture}
  \end{figure}
  We give up on the accurate representation of the heat source
  since it is too costly to resolve.

  We use a \textbf{lumped heat source} instead.
\end{frame}

\begin{frame}
  \frametitle{Part-scale simulation}
  \framesubtitle{Lumped heat source}
  A time-step is chosen regardless of the heat source path.
  Elements that are intersected by the heat source path
  during the time-step are \textbf{heated uniformly}.
  \begin{figure}
    \centering
    \begin{subfigure}[t]{0.495\textwidth}
      \centering
      \def\svgwidth{\textwidth}
      \import{./figures/lumped_hs}{heated_els.pdf_tex}
      \caption{Elements heated uniformly during $[t^{n},\,t^{n+1}]$.\\
           \begin{tikzpicture}
             \draw[red, very thick, dashed] (0,0.3) -- (0.4,0.3);
           \end{tikzpicture}
           $\; \rightarrow \;$ Heat source path
         }
      \label{fig:lhs_heated_els}
    \end{subfigure}%
    \hfill%
    \begin{subfigure}[t]{0.48508\columnwidth}
      \centering
      \includegraphics[width=\textwidth]{lumped_hs/tem.png}
      \caption{Resulting temperature field.}
      \label{fig:lhs_temp_field}
    \end{subfigure}
    \caption{Lumped heat source example.}
  \end{figure}
\end{frame}

\begin{frame}
  \frametitle{Part-scale simulation}
  \framesubtitle{Lumped heat source}
  \begin{figure}
    \begin{subfigure}[t]{0.32\textwidth}
      \centering
      \includegraphics[width=\textwidth]{chiumenti2017-lhs/chiumenti2017multipleHatches.png}
      \caption{Multiple hatches per time-step.}
    \end{subfigure}%
    \hfill
    \begin{subfigure}[t]{0.32\textwidth}
      \centering
      \includegraphics[width=\textwidth]{chiumenti2017-lhs/chiumenti2017singleLayer.png}
      \caption{Single layer per time-step.}
    \end{subfigure}%
    \hfill
    \begin{subfigure}[t]{0.32\textwidth}
      \centering
      \includegraphics[width=\textwidth]{chiumenti2017-lhs/chiumenti2017multipleLayers.png}
      \caption{4 layers per time-step.}
    \end{subfigure}
    \caption{Lumped heat input strategies from \cite{chiumenti2017b}.}
  \end{figure}
  \textit{Pros and cons}
  \begin{itemize}
    \item[+] Feasible simulations
    \item[-] Distributed heat input $\;\longrightarrow\;$ temperatures below melt $\;\longrightarrow\;$ Numerical calibration (flash heating)
    \item[-] Loss of history: no resolution of heat source motion
  \end{itemize}
\end{frame}

