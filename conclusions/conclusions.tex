\begin{frame}
  \frametitle{Thesis conclusions}

  \begin{itemize}

    % Overlay 1
    \item<1-> \textbf{Substepping} provides a \textbf{robust} and \textbf{flexible}
    framework to localize temporal resolution, yielding substantial
    computational savings while preserving local accuracy.

    \item<1-> \textbf{Higher-order time integration} in the slow partition
    could further increase the allowable macro time step.

    % Overlay 2
    \item<2-> \textbf{Substepping alone is insufficient}:
    even with reduced cost per step, \textbf{decimeter-scale parts still require
    hundreds of billions of time steps} under the classical constraint
    \[
      \Delta t \;\le\; 1 T_{hs}
    \]

    % Overlay 3
    \item<3-> The \textbf{advected subdomain} method \textbf{bypasses this constraint},
    enabling time steps up to $\mathcal{O}(10)\,T_{\mathrm{hs}}$,
    but it \textbf{lacks robustness} for general scan paths.

    % Overlay 4
    \item<4-> \textbf{Decimeter-scale LPBF simulations
    remain out of reach}.

  \end{itemize}
\end{frame}
