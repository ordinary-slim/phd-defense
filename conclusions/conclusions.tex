\begin{frame}
  \frametitle{Thesis conclusions}

  \begin{itemize}
    \item \textbf{Main bottleneck} in \textbf{part-scale LPBF} simulation is
      extreme \textbf{time-scale disparity}; exploiting time-scale separation
      is essential to achieve computational tractability without sacrificing
      local accuracy.

    \item The \textbf{advected subdomain} method \textbf{bypasses time-step
      restrictions} and \textbf{improves accuracy close to the heat source},
      but it is \textbf{not robust} to complex scan paths
      involving direction changes and track overlaps.

    \item \textbf{Substepping} provides a \textbf{robust} and \textbf{flexible}
      framework to localize temporal resolution;
      we believe it \textbf{will become a standard tool in FEM-based LPBF simulation}.

    \item \textbf{First order time} integration in the \textbf{slow partition}
      \textbf{limits the maximum macro-step size}.
  \end{itemize}
\end{frame}

