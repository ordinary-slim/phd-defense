\begin{frame}
  \frametitle{LPBF}
  \framesubtitle{Extremely multiscale}
  LPBF is an \textit{extremely multiscale} application \citep{hodge2021}.\\
  Let's quantify this statement:
  \begin{itemize}
    \item The \textbf{smallest} spatial and temporal \textbf{scales} are governed by the
      \textbf{heat source}, characterized by its radius $\mathbf{R}$ and by
      the time it takes to travel one radius,
      \[
        \mathbf{T_{hs}} := \frac{R}{V}.
      \]
    \item The \textbf{largest} spatial and temporal \textbf{scales} are set
      by the \textbf{part} and the printer.
      We choose here the characteristic part length $\mathbf{L_{part}}$ and
      the net printing time $\mathbf{T_{print}}$, i.e. the cumulative
      laser-on time.
  \end{itemize}
\end{frame}

\begin{frame}
  \frametitle{LPBF}
  \framesubtitle{Extremely multiscale}
  \centering
  \begin{minipage}{0.59\textwidth}
    \begin{figure}[ht]
      \def\svgwidth{0.84\columnwidth}
      \import{./figures/cube-example}{cube-scan-schematic.pdf_tex}
      \caption{Cube scan path of \textbf{side length} $L$, \textbf{layer thickness} $t$
      and \textbf{hatch spacing} $h$.}
    \end{figure}
  \end{minipage}%
  \hfill%
  \begin{minipage}{0.38\textwidth}
    Consider printing a cube of side length $L$.
    The ratio of volume scales is
    $$
    \frac{L^3}{R^3}
    $$
    Let's compute the time scale ratio.
    Assume $t~=~h~=~R$ for simplicity, so that
    $$N_{layers} = N_{hatches\\/layer} = \frac{L}{R}$$
  \end{minipage}
\end{frame}

\begin{frame}
  \frametitle{LPBF}
  \framesubtitle{Extremely multiscale}
  For this simple geometry, the net print time is
  \begin{gather*}
    T_{print} = N_{hatches} \cdot T_{hatch} = N_{layers} \cdot N_{hatches/layer} \cdot T_{hatch}\\
    T_{hatch} = \frac{L}{V}\\
    \implies
    T_{print} = \frac{L}{R} \cdot \frac{L}{R} \cdot \frac{L}{V} = \frac{L^2}{R^2} \cdot \frac{L}{V} = \frac{L^3}{R^2 V}
  \end{gather*}
  Therefore, the time scale disparity is
  $$
  \frac{T_{print}}{T_{hs}} = \frac{L^3}{R^2 V} \cdot \frac{V}{R} = \left(\frac{L}{R}\right)^3
  $$
\end{frame}

\begin{frame}
  \frametitle{LPBF}
  \framesubtitle{Multiphysics}
  \begin{figure}
    \centering
    \includegraphics[height=0.8\textheight]{bayat2021.png}
    \caption{Relevant physics at melt-pool and part scales \citep{bayat2021}.}
    \label{fig:bayat2021}
  \end{figure}
\end{frame}

\begin{frame}
  \frametitle{Part-scale simulation}
  \framesubtitle{Why?}

  LPBF is both \textbf{extremely multiscale} and a \textbf{multiphysics} application.
  Despite this complexity, predictions are required at the \textbf{part scale}.

  \vspace{0.5em}
  Part-scale simulation aims to:
  \begin{itemize}
    \item Reduce costly experimental trial-and-error
    \item Predict residual stresses and part distortions
    \item Link thermal history to microstructure features
    \item Enable process-parameter optimization
    \item Shorten the design-to-manufacturing cycle
  \end{itemize}
\end{frame}

\begin{frame}
  \frametitle{Part-scale simulation}
  \framesubtitle{Problem statement}
  \begin{figure}[ht]
    \def\svgwidth{0.9\columnwidth}
    \hspace{-6mm}
    \import{./figures/lpbf_schematic}{schematic.pdf_tex}
    \caption{Schematic of the LPBF computational domain 
    $\Omega(t)$, encompassing the bulk (part and substrate) and powder bed regions,
    together with the applied heat source and convective/radiative heat losses.
  }
  \end{figure}
\end{frame}

\begin{frame}
  \frametitle{Part-scale simulation}
  \framesubtitle{PDE}
  {\small
  Define the extended temperature and liquid fraction fields
  \begin{equation*}
    T_e\dependson{\mathbf{x}, t} =
    \begin{cases}
      T\dependson{\mathbf{x}, t} & \mathbf{x} \in \overline{\Omega}(t)\\
      T_{dep} & \mathbf{x} \in \overline{\Omega}(t_{final}) \setminus \overline{\Omega}(t)
    \end{cases}
    \qquad
    f_{l,\; e}\dependson{\mathbf{x}, t} = f_l(T_e\dependson{\mathbf{x}, t})
  \end{equation*}
  where $T_{dep}$ is the deposition temperature.\\
  Find $T : \Omega(t) \times [0, T_{\text{final}}] \to \mathbb{R}$ such that
  \begin{align}
    \label{eq:original_pde}
    \rho c_p \partial_t T_e + \rho L \partial_t f_{l,\; e} - k \Delta T
    &= r\dependson{\mathbf{x}, t} &&\forall \mathbf{x} \in \Omega(t)\\
    \notag
    - k \partial_n T &= h_{conv} (T - T_{\text{env}}) + \varepsilon \sigma (T^4 - T_{\text{env}}^4) &&\forall \mathbf{x} \in \partial \Omega(t)\\
    \notag
    T\dependson{\mathbf{x}, 0} &= T_0 \qquad &&\forall \mathbf{x} \in \Omega(0)
  \end{align}
  }
    \todo{Comment on phase change treatment here}
\end{frame}

\begin{frame}
  \frametitle{Part-scale simulation}
  \framesubtitle{Discretization}
    Multiply \cref{eq:original_pde}
    by $\phi \in V_T(t) = H^{1}\left(\Omega(t)\right)$;
    integrate over $\Omega\dependson{t}$;
    apply integration by parts on the diffusion term; insert BCs;
    apply BDF1
    {\small
    \begin{gather*}
      \label{eq:weak_heat}
      \int_{\Omega} \phi \rho \left({c_p \frac{T^{n+1} - T^n}{\Delta t} + L \frac{f_l(T^{n+1}) - f_l(T^n)}{\Delta t}}\right)
      + \int_{\Omega} \nabla \phi \cdot \left(k \nabla T\right)\\
      \notag
      \forall \phi \in V_T(t) \hspace{1cm}
      \;=\; \int_{\Omega} \phi r
      + \int_{\partial \Omega} \phi \left({h_{\text{conv}} \left( T - T_{\text{env}} \right)
      + \varepsilon \sigma \left( {T}^4 - T_{\text{env}}^4 \right)}\right)
    \end{gather*}
    }
\end{frame}

\begin{frame}
  \frametitle{Part-scale simulation}
  \framesubtitle{Discretization}
  \begin{figure}
    \begin{subfigure}[t]{0.30\textwidth}
      \includegraphics[width=\textwidth]{schematic_melting/0.png}
      \caption{Bare substrate below an inactive powder layer.}
      \label{fig:refModelBareSubstrate}
    \end{subfigure}\hfill%
    \begin{subfigure}[t]{0.30\textwidth}
      \includegraphics[width=\textwidth]{schematic_melting/1.png}
      \caption{A powder layer is activated during a recoating step.}
      \label{fig:powderLayer}
    \end{subfigure}\hfill%
    \begin{subfigure}[t]{0.30\textwidth}
      \includegraphics[width=\textwidth]{schematic_melting/2.png}
      \caption{After a heating step, elements whose average temperature
      surpasses $T_m$ are set to bulk.}
      \label{fig:activPhaseChange3}
    \end{subfigure}\hfill%
    \caption{Illustration of deposition and melting processes.\qquad
      \legendpowderbulk{}
    }
    \label{fig:activPhaseChange}.
  \end{figure}
  Same treatment of phase change as in \citep{kollmannsberger2018}
  \todo{Maybe remove this slide, contradicting legend on next slide?}
\end{frame}

\begin{frame}
  \frametitle{Part-scale simulation}
  \framesubtitle{Demo video}

  \begin{figure}
    \video<1>[above,height=0.7,aspectratio=62/27,fit=fill,controls]
  at (0.5,0.18) {videos/2dlpbf_2d_lpbf_ref.mp4}
    \vspace{0.7\pageheight+3pt}
    \videoCaption{
      Demo simulation of 2D LPBF with element activation.\\
      Wireframe elements (\;\wireframeTriangle{}) correspond to powder region.
    }
  \end{figure}

\end{frame}

\begin{frame}
  \frametitle{Part-scale simulation}
  \framesubtitle{Impossible}
  Previous slide: ``uniform'' mesh in part and powder region.
  Recall the volume scale ratio;
  In 3D, we would need
  \begin{equation}
    \label{eq:num_elements_uniform_mesh}
    \text{\# elements} = \mathcal{O}\left(\frac{L^3}{R^3}\right)
  \end{equation}
  to resolve the heat source throughout the print.

  In practice, no one uses uniform meshes for LPBF simulation;
  \textbf{AMR} is regarded as the \textbf{de facto standard}.
\end{frame}

\begin{frame}
  \frametitle{Part-scale simulation}
  \framesubtitle{Impossible}
  But what about time-steps?
  The interval of interest is $]0, T_{final}[$ with
  $$
  T_{final} = T_{print} + T_{cool}
  $$
  i.e. the net printing time plus cooling.
  The time scale disparity requires again \cref{eq:num_elements_uniform_mesh} time-steps
  \begin{gather}
    \label{eq:num_timesteps_uniform_mesh}
    \text{\# time-steps} > \frac{T_{print}}{T_{hs}} = \mathcal{O}\left(\frac{L^3}{R^3}\right)\\
  \end{gather}
  when discretizing with \textbf{uniform time-steps}.
\end{frame}

\begin{frame}
  \frametitle{Part-scale simulation}
  \framesubtitle{Impossible}
  Intuitively, if we don't respect the constraint
  \begin{equation}
    \label{eq:origconstraint}
    \Delta t \leq T_{hs}
  \end{equation}
  , we won't resolve the motion of the heat source.\\
  In practice, that's indeed what happens:
  if the time-step is larger than $T_{hs}$ i.e. the heat source
  travels more than $1 R$ per time-step,
  it skips over parts of the domain,
  and generates artificial temperature spikes.
  \begin{figure}
    \begin{subfigure}[t]{0.49\textwidth}
      \includegraphics[width=\textwidth]{timestep-2ths/1R.png}
      \caption{$\Delta t = 1 T_{hs}$}
      \label{fig:spots1R}
    \end{subfigure}
    \begin{subfigure}[t]{0.49\textwidth}
      \includegraphics[width=\textwidth]{timestep-2ths/2R.png}
      \caption{$\Delta t = 2 T_{hs}$}
      \label{fig:spots2R}
    \end{subfigure}
    % \caption{2D heating track example with admissible and inadmissible
    % time-step sizes according to inequality \eqref{eq:origconstraint}.}
    % \label{fig:spots}
  \end{figure}
\end{frame}

\begin{frame}
  \frametitle{Part-scale simulation}
  \framesubtitle{Impossible}
  So \eqref{eq:num_timesteps_uniform_mesh} is indeed a \textbf{lower bound} on the number of time-steps
  when using uniform time-stepping.

  \underline{Bad news:}
  \begin{itemize}
    \item \textbf{Uniform time-stepping} is the \textbf{de facto standard} in LPBF simulation.
    \item Some references require time-steps much smaller than $T_{hs}$
      to ensure stability and accuracy \citep{hodge2014, hodge2021, elahi2025}.
  \end{itemize}
  There are basically 4 (!) groups in the world that can run simulations in the order of $\mathcal{O}(10 \unit{\milli\meter})$:
  Pittsburgh, Northwestern, LLNL, TUM.\\
  Decimeter-scale parts are currently unfeasible.
  \todo{Expand}
\end{frame}

