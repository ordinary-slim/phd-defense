\begin{frame}
  \frametitle{Advected subdomain}
  Resolving the motion requires prohibitively small time-steps.

  Idea: \textbf{Get rid of the motion}!

  Define a subdomain $\Omega_m$ that moves with the heat source
  and solves the heat equation in its reference frame.

  Due to motion of reference frame, the heat equation in $\Omega_m$
  acquires an advection term $\longrightarrow$ \textbf{advected subdomain}
  (AS) method.
\end{frame}

\begin{frame}
  \frametitle{Advected subdomain}
  \begin{figure}
    \centering
    \def\svgwidth{1.05\columnwidth}
    \small
    \import{./figures/chimera_schematic/}{chimera_schematic.pdf_tex}
    \caption{Schematic of Robin-Robin variant of advected subdomain method.}
    \label{fig:chimera_schematic}
  \end{figure}
\end{frame}

\begin{frame}
  \frametitle{Advected subdomain}
  
    {\small
    \begin{gather*}
      (\mathbf{x}, t) \longrightarrow (\boldsymbol{\xi}, \eta) \thinspace , \thinspace
      \begin{cases}
        \boldsymbol{\xi} = \mathbf{x} - \int_0^t \mathbf{v}\dependson{t} dt\\
        \eta = t
      \end{cases}
    \end{gather*}
    \begin{gather*}
      \Longrightarrow
      \begin{cases}
        \partial_{x_i} &= \partial_{\xi_i} \Rightarrow \nabla_{\mathbf{x}} = \nablaxi\\
        \partial_t &= \partial_\eta - v_i \partial_{\xi_i} = \partial_\eta - \mathbf{v}\dependson{t} \cdot \nablaxi
      \end{cases}
    \end{gather*}

    Find $T_m : \Omega_m(t) \times [0, T_{\text{final}}] \to \mathbb{R}$
    and $T_f : \Omega_f(t) \times [0, T_{\text{final}}] \to \mathbb{R}$
    such that
    \begin{align*}
      \rho c_p \Big( \partial_t T^m{\scriptstyle (\xi, t)} - \mathbf{v}\dependson{t} \cdot \nabla T^m{\scriptstyle (\xi, t)} \Big) -
      \nabla \cdot ( k \nabla T^m{\scriptstyle (\xi, t)}) &= r{\scriptstyle (\xi,t)}  &\xi &\in \Omega_m(t)
    \end{align*}
    \begin{align*}
    \rho c_p \partial_t T^f{\scriptstyle (x, t)} - \nabla \cdot (k \nabla T^f{\scriptstyle (x, t)}) &= 0 \qquad &x &\in \Omega_f(t)
    \end{align*}
    }
\end{frame}

\begin{frame}
  \frametitle{Advected subdomain}
  \cite{slimani2024}: Dirichlet-Neumann multi-mesh variant of advected
  subdomain method for linear heat equation.
  \begin{figure}[h]
    \centering
    \includegraphics[width=0.6\textwidth]{quadTriang.png}
    \caption{A finer mesh can be attached to the heat source
    to adress spatial scale disparity: not done here.}
  \end{figure}
  \begin{itemize}
    \item VMS stabilization to handle advection in $\Omega_m$
    \item Monolithic solution of the coupled problem
    \item Serial C++ implementation available on GitHub.
  \end{itemize}
\end{frame}

\begin{frame}
  \frametitle{Advected subdomain}
  Goal workflow:
  \begin{gather*}
    \texttt{steadiness\_metric}(T_m) := \frac{\eunorm{T_m\big|_{t^{n+1}} - T_m\big|_{t^{n}}}}{\eunorm{T_m\big|_{t^{n+1}}}} < \epsilon\\
    \big\Downarrow\\
    \text{Increase } \Delta t \text{ for } t^{n+2} = t^{n+1} + \Delta t\\
    \big\Downarrow\\
    \textrm{Resize } \Omega_m
  \end{gather*}
\end{frame}

\begin{frame}
  \frametitle{Advected subdomain}
  \framesubtitle{Non-trivial workflow}
  \begin{figure}
    \begin{subfigure}[t]{0.36\textwidth}
      \includegraphics[width=\textwidth]{2d_meshing_example/afterShaping.png}
      \caption{$\mathcal{T}^m \leftarrow \texttt{shapeSubdomain}(\mathcal{T}^m_{bg})$}
    \end{subfigure}
    \hfill%
    \begin{subfigure}[t]{0.36\textwidth}
      \includegraphics[width=\textwidth]{2d_meshing_example/after_intersec1.png}
      \caption{$\mathcal{T}^m \leftarrow \texttt{intersect}\big(\mathcal{T}^m, \mathcal{T}^f \textrm{ at } t^n\big)$}
    \end{subfigure}
    \begin{subfigure}[t]{0.36\textwidth}
      \includegraphics[width=\textwidth]{2d_meshing_example/after_intersec2.png}
      \caption{$\mathcal{T}^m \leftarrow \texttt{intersect}\big(\mathcal{T}^m, \mathcal{T}^f\big) \textrm{ at } t^{n+1}$}
    \end{subfigure}
    \hfill%
    \begin{subfigure}[t]{0.36\textwidth}
      \includegraphics[width=\textwidth]{2d_meshing_example/dd.png}
      \caption{$\mathcal{T}^f \leftarrow \texttt{subtract}\big(\mathcal{T}^f, \mathcal{T}^m\big)$}
    \end{subfigure}
    \label{fig:2d_meshing_example}
  \end{figure}
\end{frame}

% \begin{frame}
%   \frametitle{Advected subdomain}
%   \begin{table}
%     \centering
%     \begin{tabular}{lrl}
%       \toprule
%       Parameter & Value & Unit \\
%       \midrule
%       Density  & 4300 & $kg / m^3$\\
%       Specific heat  & 700 & $J / kg K$\\
%       Conductivity  & 10 & $W / m K$\\
%       Speed $V$  & 1000 & $mm / s$\\
%       Radius $R$  & 0.1 & $mm$\\
%       Power  & 50 & \unit{\watt}\\
%       Environment temperature  & 25 & \unit{\celsius}\\
%       \bottomrule
%     \end{tabular}
%     \label{tbl:mat}
%   \end{table}
% \end{frame}

\begin{frame}
  \frametitle{Advected subdomain}
  Analytical validation with 3D welding example
  \citep{nguyen1999,vanelsen2007}
  and Ti64-like material properties.
  \vspace{-5mm}
  \begin{figure}[]
      \centering
      \begin{tabular}{cc}
  \begin{subfigure}[t]{0.49\linewidth}
    \centering
    \includegraphics[width=\textwidth]{3d_weldingLpbf/analytical.png}
    \caption{Analytical}
  \end{subfigure} &
  \addtocounter{subfigure}{2}
  \begin{subfigure}[t]{0.49\linewidth}
    \centering
    \includegraphics[width=\textwidth]{3d_weldingLpbf/allContours.png}
    \definecolor{referenceColor}{HTML}{1762FB}
    \definecolor{proposedColor}{HTML}{00AA00}
    \vspace{-0.7cm}
    \caption{
         ({\color{black} \rule[-1.5 pt]{8 pt}{8 pt}}) Analytical \;\;
         ({\color{proposedColor} \rule[-1.5 pt]{8 pt}{8 pt}}) AS \;\;
         ({\color{referenceColor} \rule[-1.5 pt]{8 pt}{8 pt}}) Reference \;\;
         \\
      Isothermals and position of maximum temperature.
    }
    \label{fig:3dweldallcontours}
  \end{subfigure} \\
  \addtocounter{subfigure}{-3}
  \begin{subfigure}[t]{0.49\linewidth}
    \centering
    \includegraphics[width=\textwidth]{3d_weldingLpbf/coupled_elsPerRad8.png}
    \caption{Advected subdomain (AS), $\Delta t = 2\,T_{hs}$}
    \label{fig:3dweldcontourprop}
  \end{subfigure} & 
  \addtocounter{subfigure}{2}
  \begin{subfigure}[t]{0.49\linewidth}
    \centering
    \includegraphics[width=\textwidth]{3d_weldingLpbf/err_coupled_elsPerRad8.png}
    \caption{Pointwise error of \ref{fig:3dweldcontourprop}. Max error is $\approx \mathbf{12}$}
  \end{subfigure} \\
  \addtocounter{subfigure}{-3}
  \begin{subfigure}[t]{0.49\linewidth}
    \centering
    \includegraphics[width=\textwidth]{3d_weldingLpbf/reference_elsPerRad8_tstepsPerRad4.png}
    \caption{Reference method, $\Delta t = 0.25\,T_{hs}$}
    \label{fig:3dweldcontourref}
  \end{subfigure} & 
  \addtocounter{subfigure}{2}
  \begin{subfigure}[t]{0.49\linewidth}
    \centering
    \includegraphics[width=\textwidth]{3d_weldingLpbf/err_reference_elsPerRad8_tstepsPerRad4.png}
    \caption{Pointwise error of \ref{fig:3dweldcontourref}. Max error is $\approx \mathbf{263}$}
    \label{fig:3dweldpointwiseref}
  \end{subfigure} \\
      \end{tabular}
    \caption{Solution (left) and error (right) contour at steady-state.}
  \end{figure}
\end{frame}

\begin{frame}
  \frametitle{Advected subdomain}
    \begin{figure}
      \centering
      \seudoembeddedvod{width=0.8\textwidth}{videos/chimera.mp4}{2d_welding_chimera/thumbnail-chimera.png}
      \vspace{-3mm}
      \caption{2D welding example with growing time-step workflow.}
    \end{figure}
    \vspace{-3mm}
    \begin{figure}
      \centering
      \begin{subfigure}[t]{0.3\linewidth}
        \centering
        \stackinset{l}{ 0.35\textwidth}{b}{ 0.025\textwidth}{
          {\color{white}$\Omega_f$}
        }{
          \includegraphics[width=\textwidth]{2d_welding_chimera/changeOfDir_OFF.png}
        }
        \caption{No AS, $\Delta t = 0.5\,T_{hs}$.}
        \label{fig:2dWeldNoShear}
      \end{subfigure}
      \hfill
      \begin{subfigure}[t]{0.3\linewidth}
        \centering
        \stackinset{r}{ 0.32\textwidth}{b}{ 0.1\textwidth}{
          {\color{white}$\Gamma$}
        }{
        \stackinset{r}{ 0.19\textwidth}{t}{ 0.075\textwidth}{
          {\color{white}$\Omega_m$}
        }{
        \stackinset{l}{ 0.35\textwidth}{b}{ 0.025\textwidth}{
          {\color{white}$\Omega_f$}
        }{
        \includegraphics[width=\textwidth]{2d_welding_chimera/changeOfDir_ON.png}
        }}}
        \caption{AS, $\Delta t = 0.5\,T_{hs}$.}
        \label{fig:2dWeldShear_05R}
      \end{subfigure}
      \hfill
      \begin{subfigure}[t]{0.3\linewidth}
        \centering
        \stackinset{r}{ 0.32\textwidth}{b}{ 0.1\textwidth}{
          {\color{white}$\Gamma$}
        }{
        \stackinset{r}{ 0.19\textwidth}{t}{ 0.075\textwidth}{
          {\color{white}$\Omega_m$}
        }{
        \stackinset{l}{ 0.35\textwidth}{b}{ 0.025\textwidth}{
          {\color{white}$\Omega_f$}
        }{
        \includegraphics[width=\textwidth]{2d_welding_chimera/changeOfDir_ON_finer.png}
        }}}
        \caption{AS, $\Delta t = 0.25\,T_{hs}$.}
        \label{fig:2dWeldShear_025R}
      \end{subfigure}
      \vspace{-3mm}
      \caption{``Shearing'' of previous thermal tail by the AS.}
      \label{fig:2dWeldTurn}
    \end{figure}
\end{frame}

\begin{frame}
  \frametitle{Advected subdomain}
  \framesubtitle{Thin wall}
  \begin{figure}
    \centering
    \begin{subfigure}[t]{0.48\textwidth}
      \includegraphics[draft=\isdraft, width=\textwidth]{3d_lpbf_chimera/overview1_compressed.png}
      \caption{Mesh and bulk-powder distribution at final time-step.\\
      % \legendMats
      }
      \label{fig:3dLpbfMeshMat}
    \end{subfigure}\hfill%
    \begin{subfigure}[t]{0.52\textwidth}
      \centering
      \includegraphics[width=\textwidth]{3d_lpbf_chimera/overview2.png}
      \caption{Temperature contour at intermediate time-step, metal only.
      last $\Delta t = 8 T_{hs}$, \enskip $\Gamma$ extends $50 R$ behind
      heat source.}
      \label{fig:3dLpbfPerspective}
    \end{subfigure}%
    \vspace{-3mm}
    \caption{Mesh and $\Omega_m$.}
  \end{figure}
  One track of $100R$ per layer, 10 layers, Ti64-like material properties.
\end{frame}

\begin{frame}
  \frametitle{Advected subdomain}
  \framesubtitle{Thin wall}
  \begin{minipage}[t]{0.58\textwidth}
  \pgfplotsset{compat=newest}
  \begin{figure}
    \centering
    \begin{tikzpicture}[
        trim axis left,
        trim axis right,
        baseline,
    ]

      \begin{axis}[
            clip=true,
            enlarge x limits=false,
            enlarge y limits=false,
            width=1.21\textwidth,
            grid=major,
            xlabel={x},
            x unit=\unit{\milli\meter},
            ylabel={T},
            y unit=\unit{\celsius},
            xmin=-6.0,
            xmax=+6.0,
            ymin=20.0,
            ymax=2575.0,
            ytick={25,500,1000,2000,2500},
            extra x ticks={-5, 5},
            every extra y tick/.append style={overlay,},
            legend cell align=left,
            legend pos=north west,
            legend style={font=\footnotesize},
            tick label style={font=\footnotesize},
            ylabel style={yshift=-2mm, font=\footnotesize, at={(axis description cs:-0.05,0.6)}},
            xlabel style={font=\footnotesize, at={(axis description cs:0.5,-0.1)}},
        ]
      \addlegendentry{AS, last $\Delta t = 8\,T_{hs}$}
      \addplot
      [mark=none, color=red, opacity=0.6, thin]
      table [col sep=comma, x="Points:0", y="T"] {figures/3d_lpbf_chimera/plots/endFifthLayer_coupled_rerun.csv};

      \addlegendentry{Reference, $\Delta t = 0.5\,T_{hs}$}
      \addplot
      [mark=none, color=blue, opacity=0.6, thin]
      table [col sep=comma, x="Points:0", y="T"] {figures/3d_lpbf_chimera/plots/endFifthLayer_ref_normal.csv};

      \addlegendentry{Reference, $\Delta t = 0.125\,T_{hs}$}
      \addplot
      [mark=none, color=black, opacity=0.6, thin]
      table [col sep=comma, x="Points:0", y="T"] {figures/3d_lpbf_chimera/plots/endFifthLayer_ref_fine.csv};

      \addplot
      [forget plot, color=red, thin, dashed, opacity=0.6]
      coordinates
      {(0.93, \pgfkeysvalueof{/pgfplots/ymin})
       (0.93, \pgfkeysvalueof{/pgfplots/ymax})};

      \addplot
      [forget plot, color=red, thin, dashed, opacity=0.6]
      coordinates
      {(5.0, \pgfkeysvalueof{/pgfplots/ymin})
       (5.0, \pgfkeysvalueof{/pgfplots/ymax})};
    \end{axis}

    \end{tikzpicture}
    \vspace{-3mm}
    \caption{Plot along top centerline of build towards end of fifth layer.}
    \label{fig:3dLpbf_plot5}
  \end{figure}
  \end{minipage}%
  \hspace{-0.01\textwidth}%
  \begin{minipage}[t]{0.43\textwidth}
    \small
    \begin{itemize}
      \item About 6 times fewer time-steps with AS for the same build.
      \item Each AS time-step is roughly 2 times more expensive to compute.
      \item[$\rightarrow$] Overall net speedup of about 3 times in wall-clock time.
      \item Cost of boolean operations is negligible; overhead mainly comes
            from the increased linear solve time.
    \end{itemize}
  \end{minipage}
\end{frame}

\begin{frame}
  \frametitle{Advected subdomain}
  \framesubtitle{Intermediate conclusions}
  Conclusions of \cite{slimani2024}:
  \begin{itemize}
     \item AS recovers accurate thermal profiles in the HAZ and enables
       larger time-steps.
     \item Trade-off: increased algebraic complexity and potential
       oscillations at subdomain discontinuities.
     \item Performance degrades for short scanning tracks; minimum track
       length $\approx 5R$ recommended.
     \item Future work: combine with substepping algorithms and explore
       alternative domain decomposition strategies.
  \end{itemize}
\end{frame}
