\begin{frame}
  \centering
  \begin{tikzpicture}[
    font=\small,
    box/.style={draw, rounded corners, align=center, inner sep=6pt},
    arr/.style={->, thick}
  ]

  \node[box] (ms) {LPBF multiscale constraints};

  \node[box, below=5mm of ms] (dd) {Domain decomposition\\(core framework)};

  \node[box, below left=4mm and 4mm of dd] (as) {Advected subdomain\\\citep{slimani2024}};

  \node[box, below right=4mm and 4mm of dd] (ss) {Substepping\\\citep{slimani2025}};

  \draw[arr] (ms) -- (dd);
  \draw[arr] (dd) -- (as);
  \draw[arr] (dd) -- (ss);

  \end{tikzpicture}
\end{frame}

\begin{frame}
  \frametitle{Domain decomposition}
  \framesubtitle{Basics}
  We've established that the part-scale build is driven
  by the motion of a very localized heat source.

  What if we treat differently this region aka the Heat Affected Zone (HAZ)?

  We will do so via \textbf{domain decomposition} (DD).

  Let $\Omega_1$ and $\Omega_2$ be a non-overlapping decomposition of
  the domain $\Omega = \Omega_1 \sqcup \Omega_2$ such that
  $\Omega_1$ contains the HAZ;
  we will pose different problems in $\Omega_1$ and $\Omega_2$
  and enforce \textbf{transmission conditions} at the interface
  $\Gamma = \partial \Omega_1 \cap \partial \Omega_2$
\end{frame}

\begin{frame}
  \frametitle{Domain decomposition}
  \framesubtitle{Basics}
\begin{figure}
  \centering
  \def\svgwidth{1.0\columnwidth}
  \import{figures/transmission_conds/}{drawing.pdf_tex}
  \caption{Non-overlapping DD schematic for a welding problem. $\Omega_1$ and $\Omega_2$
  cover the HAZ and underlying substrate, respectively. Transmission conditions
  must be satisfied at the interface $\Gamma = \partial \Omega_1 \cap \partial \Omega_2$.
  Admissible and inadmissible transmission conditions are shown in the miniature plots,
  which represent temperature fields across $\Gamma$.
  }
  \label{fig:transmission_conditions}
\end{figure}
\end{frame}

\begin{frame}
  \frametitle{Domain decomposition}
  \framesubtitle{Basics}
  Consider the linear heat equation with Dirichlet boundary conditions.
  Solve for $T_i$ in each subdomain $\Omega_i$:
  \begin{equation}
  \label{eq:lproblem}
  \left\{
  \begin{aligned}
    \partial_t T_i - \alpha \Delta T_i &= r && \forall \mathbf{x} \in \Omega_i\\
    T_i(\mathbf{x}, t) &= T_d && \forall \mathbf{x} \in \partial \Omega_{D,i}\\
    \mathcal{I}_i \left(T_i, T_j\right) &= 0 && \forall \mathbf{x} \in \Gamma, \enskip i \neq j
  \end{aligned}
  \right.
  \end{equation}
  This is equivalent to solving the original problem in $\Omega$ given
  that the \textbf{transmission conditions}
  are satisfied
  \begin{equation}
    \label{eq:interface_eq_conditions}
    \begin{cases}
      \mathcal{I}_1(T_1, T_2) = 0\\
      \mathcal{I}_2(T_1, T_2) = 0
    \end{cases}
    \iff
    \begin{cases}
      T_1 = T_2\\
      k_1 \partial_n T_1 = k_2 \partial_n T_2
    \end{cases}
    \qquad
    \qquad
    \forall \mathbf{x} \in \Gamma
  \end{equation}
\end{frame}

\begin{frame}
  \frametitle{Domain decomposition}
  \framesubtitle{Basics}
  Any linearly independent combination of \eqref{eq:interface_eq_conditions}
  lays a valid pair of subproblems on $\Omega_1$ and $\Omega_2$
  {\small
  \begin{equation}
    \label{eq:lp1}
    \left\{
    \begin{aligned}
      \rho c_p \partial_t T_1 - k \Delta T_1 &= r && \text{ in } \Omega_1\\
      T_1 &= T_d && \text{ in } \partial \Omega_{D,1}\\
      \gamma_1 T_1 + \eta_1 k_1 \frac{\partial T_1}{\partial n_1} &= \gamma_1 T_2 + \eta_1 k_2 \frac{\partial T_2}{\partial n_1} && \text{ in } \Gamma
    \end{aligned}
    \right.
  \end{equation}
  \begin{equation}
    \label{eq:lp2}
    \left\{
    \begin{aligned}
      \rho c_p \partial_t T_2 - k \Delta T_2 &= r && \text{ in } \Omega_2\\
      T_2 &= T_d && \text{ in } \partial \Omega_{D,2}\\
      \gamma_2 T_2 + \eta k_2 \frac{\partial T_2}{\partial n_2} &= \gamma_2 T_2 + \eta_2 k_2 \frac{\partial T_2}{\partial n_2} && \text{ in } \Gamma
    \end{aligned}
    \right.
  \end{equation}
  }
  \begin{itemize}
    \item $\gamma_1 = \eta_2 \neq 0$ and $\gamma_2 = \eta_1 = 0 \longrightarrow\;$ Dirichlet-Neumann

    \item $\gamma_i \neq 0$ and $\eta_i \neq 0 \longrightarrow\;$ Robin-Robin

    \item $\gamma_1 \neq 0$, $\eta_1 = 0$, $\gamma_2 \neq 0$ and $\eta_2 \neq 0
      \longrightarrow\;$ Dirichlet-Robin
  \end{itemize}
\end{frame}

\begin{frame}
  \frametitle{Domain decomposition}
  \framesubtitle{Basics}
  Assuming steady-state and applying FEM discretization
  on both subdomains, we obtain the following
  monolithic system:
  \renewcommand{\arraystretch}{1.2} % increases spacing (default is 1.0)
  \begin{equation}
    \label{eq:monolithic}
    \begin{pmatrix}
      \mathbf{A}_{11} & \mathbf{A}_{1\Gamma_1} & \cdot & \cdot\\
      \mathbf{A}_{\Gamma_1 1} & \mathbf{A}_{\Gamma_1 \Gamma_1} + \mathbf{R}_{11} & \cdot & \mathbf{R}_{12}\\
      \cdot & \cdot & \mathbf{A}_{22} & \mathbf{A}_{2\Gamma_2}\\
      \cdot & \mathbf{R}_{21} & \mathbf{A}_{\Gamma_2 2} & \mathbf{A}_{\Gamma_2 \Gamma_2} + \mathbf{R}_{22}
    \end{pmatrix}
    \begin{pmatrix}
      \mathbf{T}_1\\
      \mathbf{T}_{\Gamma_1}\\
      \mathbf{T}_2\\
      \mathbf{T}_{\Gamma_2}
    \end{pmatrix}
    =
    \begin{pmatrix}
      \mathbf{F}_1\\
      \mathbf{F}_{\Gamma_1}\\
      \mathbf{F}_2\\
      \mathbf{F}_{\Gamma_2}
    \end{pmatrix}
  \end{equation}
  $$
      \underbrace{\int_{\Omega_i} \nabla v_i \cdot k \nabla T_i}_{\mathbf{A}}\,
    - \underbrace{\int_{\Gamma} v_i \gamma_i T_i}_{\mathbf{R}_{ii}}\,
    = \underbrace{\int_{\Omega_i} r v_i\,}_{\mathbf{F}}\,
    - \underbrace{\int_{\Gamma} v_i \left(\gamma_i T_j + \eta_i k \partial_n T_j\right)}_{\mathbf{R}_{ij}}
  $$
  Solving \eqref{eq:monolithic} directly $\; \longrightarrow\;$ \textbf{monolithic} approach
\end{frame}

\begin{frame}
  \frametitle{Domain decomposition}
  \framesubtitle{Basics}
  
  Solving problem on $\Omega_1$
  \renewcommand{\arraystretch}{1.2} % increases spacing (default is 1.0)
  \begin{equation}\label{eq:staggered1}
    \begin{pmatrix}
      \mathbf{A}_{11} & \mathbf{A}_{1\Gamma_1}\\
      \mathbf{A}_{\Gamma_1 1} & \mathbf{A}_{\Gamma_1 \Gamma_1} + \mathbf{R}_{11}
    \end{pmatrix}
    \begin{pmatrix}
      \mathbf{T}^{k+1}_1\\
      \mathbf{T}^{k+1}_{\Gamma_1}
    \end{pmatrix}
    =
    \begin{pmatrix}
      \mathbf{F}_1\\
      \mathbf{F}_{\Gamma_1} - \mathbf{R}_{12} \mathbf{T}^k_{\Gamma_2}
    \end{pmatrix}
  \end{equation}
  and feeding the interface values to the problem on $\Omega_2$
  \begin{equation}\label{eq:staggered2}
    \begin{pmatrix}
      \mathbf{A}_{22} & \mathbf{A}_{2\Gamma_2}\\
      \mathbf{A}_{\Gamma_2 2} & \mathbf{A}_{\Gamma_2 \Gamma_2} + \mathbf{R}_{22}
    \end{pmatrix}
    \begin{pmatrix}
      \mathbf{T}^{k+1}_2\\
      \mathbf{T}^{k+1}_{\Gamma_2}
    \end{pmatrix}
    =
    \begin{pmatrix}
      \mathbf{F}_2\\
      \mathbf{F}_{\Gamma_2} - \mathbf{R}_{21} \mathbf{T}^{k+1}_{\Gamma_1}
    \end{pmatrix}
  \end{equation}
  \renewcommand{\arraystretch}{1.0}
  and iterating until convergence $\; \longrightarrow\;$ \textbf{staggered} approach
\end{frame}

