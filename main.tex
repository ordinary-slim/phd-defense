%! TeX program = lualatex
\documentclass{beamer}
\usetheme{metropolis}

\setbeamerfont{page number in head/foot}{size=\tiny}
\setbeamercolor{footline}{fg=gray!70}

% Footline highlight
\colorlet{footlineHlColor}{black!60}
\newcommand{\footlineHl}[1]{
  \textbf{\textcolor{footlineHlColor}{#1}}
}

\newif\ifisdraft
\isdrafttrue
\newcommand{\isdraft}{\ifisdraft true\else false\fi}
\newcommand{\todo}[1]{
  \ifisdraft {
    \Large
    $\color{black}\star$\color{red}~\textit{#1}\color{black}~$\star$
  } \fi
}

\definecolor{powderColor}{rgb}{0.878,0.859,0.811}
\definecolor{metalColor}{rgb}{0.420,0.408,0.384}
\newcommand{\legendpowderbulk}{%
  \begin{tabular}{rlrl}
     ({\color{powderColor} \rule[-1.5 pt]{8 pt}{8 pt}}) & Powder & ({\color{metalColor} \rule[-1.5 pt]{8 pt}{8 pt}})  & Bulk
  \end{tabular}
}
\newcommand{\wireframeTriangle}{%
    \begin{tikzpicture}[scale=0.2] % adjust scale as needed
      % Bottom triangle
      \draw[]
        (0,0) coordinate (A)
        -- (0,1) coordinate (B)
        -- (1,0) coordinate (C)
        -- cycle;
      % Upper triangle
      \draw[]
        (0,1) coordinate (D)
        -- (1,1) coordinate (E)
        -- (1,0) coordinate (F);
    \end{tikzpicture}
}

\defbeamertemplate*{title page}{customized}[1][]
{
  \usebeamerfont{title}\inserttitle\par
  \usebeamerfont{subtitle}\usebeamercolor[fg]{subtitle}\insertsubtitle\par
  \bigskip
  \usebeamerfont{author}\insertauthor\par
  \usebeamerfont{institute}\insertinstitute\par
  \usebeamerfont{date}\insertdate\par
  \usebeamercolor[fg]{titlegraphic}\inserttitlegraphic
}
\makeatletter
  \setbeamertemplate{frametitle}{%
    \begin{beamercolorbox}[%
      wd=\paperwidth,%
      sep=0pt,%
      leftskip=\metropolis@frametitle@padding,%
      rightskip=\metropolis@frametitle@padding,%
      ]{frametitle}%
      \metropolis@frametitlestrut@start%
      \insertframetitle%
      \ifx\insertframesubtitle\@empty%
      \else%
        \hfill%
        {\usebeamerfont{framesubtitle}\usebeamercolor[fg]{framesubtitle}\insertframesubtitle}%
      \fi%
      \nolinebreak%
      \metropolis@frametitlestrut@end%
    \end{beamercolorbox}%
  }
\makeatother

\usepackage{caption}
\captionsetup{font=scriptsize,labelfont=scriptsize}
\usepackage{subcaption}
\usepackage{siunitx}
\sisetup{
  range-phrase = --,
  range-units = single,
}
\usepackage{multirow}
\usepackage{booktabs}

\usepackage{cleveref}

\usepackage{natbib}

\usepackage{multimedia}

\usepackage{pifont}

% PSEUDO-CODE
\usepackage{algorithm}
\usepackage[
  rightComments=false,
]{algpseudocodex}

\usepackage{stackengine}
\usepackage{changepage}

% CASTEL PACKAGES
\usepackage{import}
\usepackage{xifthen}
\usepackage{transparent}
% END CASTEL PACKAGES
\newcommand{\castelincfig}[2][1.0]{%
  \def\svgwidth{#1\columnwidth} \import{./figures/}{#2.pdf_tex}
}

% Math commands
\newcommand{\dependson}[1]{
  {\scriptstyle(#1)}
}

\usepackage{calc}
\usepackage{tikz}
\usepackage{pgfplots}\pgfplotsset{compat=newest}
\usetikzlibrary{decorations.pathreplacing,calc,positioning,spy,pgfplots.units}
\usepgfplotslibrary{groupplots}
\tikzset{
  annotate box/.style={
    fill=white,
    fill opacity=.4,
    text opacity=1,
    rounded corners=2pt,
    inner xsep=4pt,
    inner ysep=3pt,
    draw=black,
    line width=.4pt,
  },
}
\DeclareRobustCommand{\labelbox}[2][]{%
  % Inline TikZ node; safe in text, math, tabular, makebox, etc.
  \tikz[baseline=(X.base)]\node[annotate box,#1] (X) {#2};%
}
\newcommand{\nablaxi}[0]{\nabla_{\boldsymbol{\xi}}}
\newcommand{\deltaxi}[0]{\Delta_{\boldsymbol{\xi}}}
\newcommand{\eunorm}[1]{
  \text{\Large$|\hspace{-0.8mm}|$}\; #1 \;\text{\Large$|\hspace{-0.8mm}|_{\scriptscriptstyle{2}}$}
}

\title{Computational strategies for time-accurate simulation of part-scale LPBF}
\subtitle{Time scale disparity in moving heat source problems}
\author{Mehdi Slimani}
\date{January 23, 2026}

\graphicspath{{figures/}}

\newif\ifisdraft
\isdraftfalse
\newcommand{\isdraft}{\ifisdraft true\else false\fi}
\newcommand{\todo}[1]{
  \ifisdraft {
    \Large
    $\color{black}\star$\color{red}~\textit{#1}\color{black}~$\star$
  } \fi
}

% Used in advected_subdomain.tex
\definecolor{advectionColor}{RGB}{213,94,0}   % vermillion / orange
\definecolor{diffusionColor}{RGB}{0,114,178}  % blue

\definecolor{powderColor}{rgb}{0.878,0.859,0.811}
\definecolor{metalColor}{rgb}{0.420,0.408,0.384}
\newcommand{\legendpowderbulk}{%
  \begin{tabular}{rlrl}
     ({\color{powderColor} \rule[-1.5 pt]{8 pt}{8 pt}}) & Powder & ({\color{metalColor} \rule[-1.5 pt]{8 pt}{8 pt}})  & Bulk
  \end{tabular}
}
\newcommand{\wireframeTriangle}{%
    \begin{tikzpicture}[scale=0.2] % adjust scale as needed
      % Bottom triangle
      \draw[]
        (0,0) coordinate (A)
        -- (0,1) coordinate (B)
        -- (1,0) coordinate (C)
        -- cycle;
      % Upper triangle
      \draw[]
        (0,1) coordinate (D)
        -- (1,1) coordinate (E)
        -- (1,0) coordinate (F);
    \end{tikzpicture}
}

\newcommand{\playthumb}[2][]{%
  \begin{tikzpicture}
    % Thumbnail image
    \node[inner sep=0] (img) {\includegraphics[#1]{#2}};
    % Dark circle
    \draw[fill=black!60, draw=white, line width=0.8pt]
      (img.center) circle[radius=0.6cm];
    % Triangle
    \draw[fill=white, draw=none, rounded corners=1.5pt]
      ([xshift=0.34cm]img.center) --
      ([xshift=-0.19cm,yshift=0.3cm]img.center) --
      ([xshift=-0.19cm,yshift=-0.3cm]img.center) -- cycle;
  \end{tikzpicture}%
}

\newcounter{video}
\renewcommand{\thevideo}{Video \arabic{video}}
\newcommand{\videoCaption}[1]{%
  \captionof{video}{#1}%
}

\newcommand{\externalvod}[3]{\movie[externalviewer]{\playthumb[#1]{#3}}{#2}}
\newcommand{\seudoembeddedvod}[3]{\movie[poster, showcontrols]{\playthumb[#1]{#3}}{#2}}

\DeclareRobustCommand{\labelbox}[2][]{%
  % Inline TikZ node; safe in text, math, tabular, makebox, etc.
  \tikz[baseline=(X.base)]\node[annotate box,#1] (X) {#2};%
}
\newcommand{\support}[1]{\text{supp}\left(#1\right)}
\newcommand{\nablaxi}[0]{\nabla_{\boldsymbol{\xi}}}
\newcommand{\deltaxi}[0]{\Delta_{\boldsymbol{\xi}}}
\newcommand{\eunorm}[1]{
  \text{\Large$|\hspace{-0.8mm}|$}\; #1 \;\text{\Large$|\hspace{-0.8mm}|_{\scriptscriptstyle{2}}$}
}

\pgfplotsset{
  colormap={rainbow}{
    rgb255(0.0cm)=(5,97,254);
    rgb255(0.023809523809523808cm)=(5,108,247);
    rgb255(0.047619047619047616cm)=(5,119,239);
    rgb255(0.07142857142857142cm)=(5,130,232);
    rgb255(0.09523809523809523cm)=(5,139,222);
    rgb255(0.11904761904761904cm)=(5,148,212);
    rgb255(0.14285714285714285cm)=(5,157,202);
    rgb255(0.16666666666666666cm)=(5,166,191);
    rgb255(0.19047619047619047cm)=(5,174,179);
    rgb255(0.21428571428571427cm)=(5,183,167);
    rgb255(0.23809523809523808cm)=(5,193,153);
    rgb255(0.2619047619047619cm)=(5,202,140);
    rgb255(0.2857142857142857cm)=(5,211,126);
    rgb255(0.30952380952380953cm)=(5,220,109);
    rgb255(0.3333333333333333cm)=(5,228,91);
    rgb255(0.35714285714285715cm)=(4,237,74);
    rgb255(0.38095238095238093cm)=(69,242,39);
    rgb255(0.40476190476190477cm)=(125,245,28);
    rgb255(0.42857142857142855cm)=(164,249,11);
    rgb255(0.4523809523809524cm)=(194,251,8);
    rgb255(0.47619047619047616cm)=(224,252,5);
    rgb255(0.5cm)=(254,254,3);
    rgb255(0.5238095238095238cm)=(254,243,20);
    rgb255(0.5476190476190477cm)=(254,232,37);
    rgb255(0.5714285714285714cm)=(254,220,55);
    rgb255(0.5952380952380952cm)=(254,208,55);
    rgb255(0.6190476190476191cm)=(254,196,55);
    rgb255(0.6428571428571429cm)=(254,183,55);
    rgb255(0.6666666666666666cm)=(254,171,55);
    rgb255(0.6904761904761905cm)=(254,159,55);
    rgb255(0.7142857142857143cm)=(254,147,55);
    rgb255(0.7380952380952381cm)=(254,132,55);
    rgb255(0.7619047619047619cm)=(254,118,55);
    rgb255(0.7857142857142857cm)=(254,104,55);
    rgb255(0.8095238095238095cm)=(253,84,53);
    rgb255(0.8333333333333334cm)=(251,66,48);
    rgb255(0.8571428571428571cm)=(252,37,53);
    rgb255(0.8809523809523809cm)=(242,29,64);
    rgb255(0.9047619047619048cm)=(230,20,74);
    rgb255(0.9285714285714286cm)=(218,10,84);
    rgb255(0.9523809523809523cm)=(203,11,91);
    rgb255(0.9761904761904762cm)=(189,11,98);
    rgb255(1.0cm)=(174,12,105);
  }
}
\pgfplotsset{
  colormap={fast}{
    rgb255(0.0cm)=(22,47,140);
    rgb255(0.16144cm)=(56,131,180);
    rgb255(0.351671cm)=(128,220,221);
    rgb255(0.501285cm)=(255,255,211);
    rgb255(0.620051cm)=(240,227,138);
    rgb255(0.835408342528245cm)=(191,113,65);
    rgb255(1.0cm)=(142,14,14);
  }
}
\pgfplotsset{
  colormap={rainbow_blended_white}{
    rgb(0cm)=(1, 1, 1);
    rgb(0.17cm)=(0, 0, 1);
    rgb(0.34cm)=(0, 1, 1);
    rgb(0.5cm)=(0, 1, 0);
    rgb(0.67cm)=(1, 1, 0);
    rgb(0.84cm)=(1, 0, 0);
    rgb(1cm)=(0.878431372549, 0, 1);
  }
}
\pgfplotsset{
  colormap={powderbulk}{
    rgb255(0cm)=(224,219,207);
    rgb255(0.499cm)=(224,219,207);
    rgb255(0.501cm)=(107.0,104.0,98.0);
    rgb255(1cm)=(107.0,104.0,98.0);
  }
}

\newcommand{\vcolorbar}[6]{
  \begin{tikzpicture}
    \pgfmathsetmacro{\myheight}{10*#2}
    \pgfmathsetmacro{\widthcolorbar}{#2}
    \pgfmathsetmacro{\cmin}{#3}
    \pgfmathsetmacro{\cmax}{#4}
    \def\extraticks{#6}
    \ifx\extraticks\empty
      \def\ticklist{0,1}
    \else
      \def\ticklist{0,\extraticks,1}
    \fi
    \begin{axis}[
      hide axis,
      scale only axis,
      height=\myheight,
      width=\widthcolorbar,
      colormap name={#1},
      colorbar,
      colorbar style={
        ytick={\ticklist},
        yticklabel pos=right,
        yticklabel style={font=\footnotesize},
        yticklabel={
          \pgfmathparse{\cmin+(\cmax-\cmin)*\tick}\pgfmathprintnumber[precision=1, fixed]{\pgfmathresult}
        },
        ylabel=#5,
        ylabel style={
          at={(0.0,0.5)},
          anchor=south,
          rotate=0,
          font=\footnotesize,
          overlay,
        },
        yticklabel style={
          font=\footnotesize,
          overlay,
        },
        width=\widthcolorbar,
      },
      point meta min=0,
      point meta max=1,
    ]
      % dummy plot just to draw colorbar:
      \addplot [draw=none] coordinates {(0,0) (0,1)};
    \end{axis}
  \end{tikzpicture}
}
\newcommand{\hcolorbar}[6]{
  \begin{tikzpicture}
    \pgfmathsetmacro{\myheight}{#2}
    \pgfmathsetmacro{\widthcolorbar}{10*#2}
    \pgfmathsetmacro{\cmin}{#3}
    \pgfmathsetmacro{\cmax}{#4}
    \def\extraticks{#6}
    \ifx\extraticks\empty
      \def\ticklist{0,1}
    \else
      \def\ticklist{0,\extraticks,1}
    \fi
    \begin{axis}[
      hide axis,
      scale only axis,
      height=\myheight,
      width=\widthcolorbar,
      colormap name={#1},
      colorbar horizontal,
      colorbar style={
        xtick={\ticklist},
        xticklabel pos=left,
        xticklabel style={font=\footnotesize},
        xticklabel={
          \pgfmathparse{\cmin+(\cmax-\cmin)*\tick}\pgfmathprintnumber[precision=1, fixed]{\pgfmathresult}
        },
        xticklabel style={font=\footnotesize},
        xlabel=#5,
        xlabel style={at={(0.5,1.0)}, rotate=0, anchor=south, font=\footnotesize},,
        width=\widthcolorbar,
      },
      point meta min=0,
      point meta max=1,
    ]
      % dummy plot just to draw colorbar:
      \addplot [draw=none] coordinates {(0,0) (0,1)};
    \end{axis}
  \end{tikzpicture}
}


\pgfplotsset{
  convplotstyle/.style={
    clip=true,
    xlabel={$\Delta t_{f} [T_{hs}]$},
    ylabel={$||u_h - u_{ex}||_2$},
    grid=major,
    xtick={1,0.5,0.25,0.125,0.0625},
    xticklabels={$1$,$\tfrac{1}{2}$,$\tfrac{1}{4}$,$\tfrac{1}{8}$,$\tfrac{1}{16}$},
    x dir=reverse,
    scale only axis,
  },
  oscillationstyle/.style={
    clip=true,
    enlargelimits=false,
    ymin=25.0,
    ymax=2400.0,
    grid=major,
    ytick={25,400,900,1300,1625,1900,2300},
    cycle list={
      {blue,solid},
      {red,solid},
      {brown,solid},
      {blue,  densely dashdotted},
      {red,   densely dashdotted},
      {brown, densely dashdotted}
    },
    xlabel={$x$},
    x unit=\si{\mm},
    ylabel={$T$},
    y unit=\si{\celsius},
    legend cell align=left,
    legend image post style={xscale=0.8},
    %legend style={draw=none, fill=none, font=\small},
    legend style={font=\footnotesize},
    tick label style={font=\footnotesize},
    ylabel style={font=\footnotesize},
    xlabel style={font=\footnotesize},
  },
  uhnstyle/.style={mark=none, thick, blue},
  uhnpstyle/.style={mark=none, thick, purple},
  substepstyle/.style={mark=none, thick, black},
  correctstyle/.style={mark=none, thick, green, dashdotted},
  truncatedddstyle/.style={
    clip=true,
    enlargelimits=false,
    ymin=25.0,
    ymax=2400.0,
    xmin=-1.0,
    xmax=0.6,
    grid=major,
    ytick={25,400,900,1300,1900,2300},
    xlabel={$x$},
    x unit=\si{\mm},
    ylabel={$T$},
    ylabel style={at={(axis description cs:-0.2,1.03)}, anchor=south, rotate=-90},
    y unit=\si{\celsius},
    legend cell align=left,
    legend image post style={xscale=0.8},
    %legend style={draw=none, fill=none, font=\small},
    trim axis left,
    scale only axis,
    legend style={font=\footnotesize, text opacity=1.0, fill opacity=0.6},
  },
  meltpoolstyle/.style={
    width=1.0\linewidth,
    grid=major,
    legend cell align=left,
    legend style={font=\footnotesize},
    tick label style={font=\footnotesize},
    cycle list name=color list,
    unit vector ratio=1 1,
    extra x ticks={50},
    extra x tick labels={},
  },
  powderStyle/.style={black, thick, mark=square},
  bulkStyle/.style={blue, thick, mark=diamond}
}

\newcommand{\gammafs}{
  \pgfkeysgetvalue{/pgfplots/ymin}\myymin
  \pgfkeysgetvalue{/pgfplots/ymax}\myymax
  \draw[densely dashed, very thin] (-0.15313, \myymin) -- (-0.15313, \myymax)
    node [pos=0.2, anchor=east] {$\Gamma_{fs}$};
}


\begin{document}

  \maketitle

  \setbeamertemplate{frame footer}{Computational strategies for time-accurate simulation of part-scale \footlineHl{LPBF}}

  \begin{frame}
    \frametitle{{LPBF}}
    \framesubtitle{MAM technology}
    Also known as PBF-LB/M (ISO nomenclature), one of the
    main Metal Additive Manufacturing (MAM) technologies

    \begin{figure}
      \begin{subfigure}[t]{0.32\textwidth}
        \centering
        \includegraphics[width=\linewidth]{waam.jpg}
        \textbf{Wire Arc Additive Manufacturing
        \raisebox{5pt}{(}\includegraphics[height=16pt]{waam-comic-effect.png}\raisebox{5pt}{)}}
      \end{subfigure}%
      \hfill
      \begin{subfigure}[t]{0.32\textwidth}
        \centering
        \includegraphics[width=\linewidth]{ded.jpg}
        \textbf{Directed Energy Deposition (DED)}
      \end{subfigure}%
      \hfill
      \begin{subfigure}[t]{0.32\textwidth}
        \centering
        \includegraphics[width=\linewidth]{lpbf.png}
        \textbf{Laser Powder Bed Fusion (LPBF)}
      \end{subfigure}
    \end{figure}
  \end{frame}

  % \begin{frame}
  %   \frametitle{LPBF}
  %   \framesubtitle{How does it work?}
  %   \todo{Add schematic of LPBF process here.}
  % \end{frame}
  %
  \begin{frame}
    \frametitle{{LPBF}}
    \framesubtitle{Fast, small, precise}

    \begin{table}
      \centering
      \begin{tabular}{l
        S
        S
        S }
        \toprule
        & {Radius $R$} 
        & {Speed $V$} 
        & {Power $P$} \\
        \midrule
        WAAM 
          & \qtyrange[]{2}{4}{\milli\meter}
          & \qtyrange[]{3}{10}{\milli\meter\per\second}
          & \qtyrange[]{5}{15}{\kilo\watt} \\
        DED 
          & \qtyrange[]{0.5}{1.5}{\milli\meter}
          & \qtyrange[]{5}{20}{\milli\meter\per\second}
          & \qtyrange[]{1}{4}{\kilo\watt} \\
        LPBF 
          & \qtyrange[]{25}{100}{\micro\meter}
          & \qtyrange[]{400}{1400}{\milli\meter\per\second}
          & \qtyrange[]{0.2}{1}{\kilo\watt} \\
        \bottomrule
      \end{tabular}
      \caption{Characteristic heat source parameters for the main MAM technologies.}
    \end{table}

    \vspace{-3mm}
    % LPBF offers a finer resolution and smoother surface finish,
    % but it is limited to smaller parts.
    \begin{figure}
      \centering
      \begin{subfigure}[t]{0.48\textwidth}
        \centering
        \includegraphics[width=0.9\linewidth]{waam-example.jpg}\\
        {\scriptsize WAAM: large parts (> \qty{1}{\meter}), coarse features.}
      \end{subfigure}%
      \hfill
      \begin{subfigure}[t]{0.48\textwidth}
        \centering
        \includegraphics[width=0.6\linewidth]{lpbf-fine.png}\\
        {\scriptsize LPBF: small part (< \qty{1}{\meter}), fine features.}
      \end{subfigure}
    \end{figure}

  \end{frame}

  \setbeamertemplate{frame footer}{Computational strategies for time-accurate \footlineHl{simulation} of \footlineHl{part-scale} LPBF}

  \begin{frame}
    \frametitle{LPBF}
    \framesubtitle{Extremely multiscale}
    LPBF is an \textit{extremely multiscale} application \citep{hodge2021}.\\
    Let's quantify this statement:
    \begin{itemize}
      \item The \textbf{smallest} spatial and temporal \textbf{scales} are governed by the
        \textbf{heat source}, characterized by its radius $\mathbf{R}$ and by
        the time it takes to travel one radius,
        \[
          \mathbf{T_{hs}} := \frac{R}{V}.
        \]
      \item The \textbf{largest} spatial and temporal \textbf{scales} are set
        by the \textbf{part} and the printer.
        We choose here the characteristic part length $\mathbf{L_{part}}$ and
        the net printing time $\mathbf{T_{print}}$, i.e. the cumulative
        laser-on time.
    \end{itemize}
  \end{frame}

  \begin{frame}
    \frametitle{LPBF}
    \framesubtitle{Extremely multiscale}
    \centering
    \begin{minipage}{0.59\textwidth}
      \begin{figure}[ht]
        \def\svgwidth{\columnwidth}
        \import{./figures/cube-example}{cube-scan-schematic.pdf_tex}
        \caption{Cube scan path of \textbf{side length} $L$, \textbf{layer thickness} $t$
        and \textbf{hatch spacing} $h$.}
      \end{figure}
    \end{minipage}%
    \hfill%
    \begin{minipage}{0.38\textwidth}
      Consider printing a cube of side length $L$.
      The ratio of volume scales is
      $$
      \frac{L^3}{R^3}
      $$
      Let's compute the time scale ratio.
      Assume $t~=~h~=~R$ for simplicity, so that
      $$N_{layers} = N_{hatches\\/layer} = \frac{L}{R}$$
    \end{minipage}
  \end{frame}

  \begin{frame}
    \frametitle{LPBF}
    \framesubtitle{Extremely multiscale}
    For this simple geometry, the net print time is
    \begin{gather*}
      T_{print} = N_{hatches} \cdot T_{hatch} = N_{layers} \cdot N_{hatches/layer} \cdot T_{hatch}\\
      T_{hatch} = \frac{L}{V}\\
      \implies
      T_{print} = \frac{L}{R} \cdot \frac{L}{R} \cdot \frac{L}{V} = \frac{L^2}{R^2} \cdot \frac{L}{V} = \frac{L^3}{R^2 V}
    \end{gather*}
    Therefore, the time scale disparity is
    $$
    \frac{T_{print}}{T_{hs}} = \frac{L^3}{R^2 V} \cdot \frac{V}{R} = \left(\frac{L}{R}\right)^3
    $$
  \end{frame}

  \begin{frame}
    \frametitle{Part-scale simulation}
    \framesubtitle{Why?}

    We've established that LPBF is extremely multiscale.
    In a few slides, we'll see why this is a problem for
    part-scale simulation.

    Why part-scale simulation?
    \begin{itemize}
      \item Avoid costly experimental trial-and-error
      \item Predict residual stresses and distortions
      \item Predict microstructure features
      \item Optimize process parameters
      \item Save time and costs in the design cycle
    \end{itemize}
  \end{frame}

  \begin{frame}
    \frametitle{Multiphysics \todo{Better title}}
    \begin{figure}[ht]
      \centering
      \includegraphics[height=0.8\textheight]{bayat2021.png}
      \caption{Relevant physics at melt-pool and part scales \citep{bayat2021}.}
      \label{fig:bayat2021}
    \end{figure}
  \end{frame}

  \begin{frame}
    \frametitle{Problem statement}
    \framesubtitle{Domain}
    \begin{figure}[ht]
      \def\svgwidth{\columnwidth}
      \hspace{-6mm}
      \import{./figures/lpbf_schematic}{schematic.pdf_tex}
      \caption{Schematic of the LPBF computational domain 
      $\Omega(t)$, encompassing the bulk (part and substrate) and powder bed regions,
      together with the applied heat source and convective/radiative heat losses.
    }
    \end{figure}
  \end{frame}

  \begin{frame}
    \frametitle{Problem statement}
    \framesubtitle{PDE system}
    {\small
    Define the extended temperature and liquid fraction fields
    \begin{equation*}
      T_e\dependson{\mathbf{x}, t} =
      \begin{cases}
        T\dependson{\mathbf{x}, t} & \mathbf{x} \in \overline{\Omega}(t)\\
        T_{dep} & \mathbf{x} \in \overline{\Omega}(t_{final}) \setminus \overline{\Omega}(t)
      \end{cases}
      \qquad
      f_{l,\; e}\dependson{\mathbf{x}, t} = f_l(T_e\dependson{\mathbf{x}, t})
    \end{equation*}
    where $T_{dep}$ is the deposition temperature.\\
    Find $T : \Omega(t) \times [0, T_{\text{final}}] \to \mathbb{R}$ such that
    \begin{align}
      \label{eq:original_pde}
      \rho c_p \partial_t T_e + \rho L \partial_t f_{l,\; e} - k \Delta T
      &= r\dependson{\mathbf{x}, t} &&\forall \mathbf{x} \in \Omega(t)\\
      \notag
      - k \partial_n T &= h_{conv} (T - T_{\text{env}}) + \varepsilon \sigma (T^4 - T_{\text{env}}^4) &&\forall \mathbf{x} \in \partial \Omega(t)\\
      \notag
      T\dependson{\mathbf{x}, 0} &= T_0 \qquad &&\forall \mathbf{x} \in \Omega(0)
    \end{align}
    }
      \todo{Comment on phase change treatment here}
  \end{frame}

  \begin{frame}
    \frametitle{Discretization}
      Multiply \cref{eq:original_pde}
      by $\phi \in V_T(t) = H^{1}\left(\Omega(t)\right)$;
      integrate over $\Omega\dependson{t}$;
      apply integration by parts on the diffusion term; insert BCs;
      apply BDF1
      {\small
      \begin{gather*}
        \label{eq:weak_heat}
        \int_{\Omega} \phi \rho \left({c_p \frac{T^{n+1} - T^n}{\Delta t} + L \frac{f_l(T^{n+1}) - f_l(T^n)}{\Delta t}}\right)
        + \int_{\Omega} \nabla \phi \cdot \left(k \nabla T\right)\\
        \notag
        \forall \phi \in V_T(t) \hspace{1cm}
        \;=\; \int_{\Omega} \phi r
        + \int_{\partial \Omega} \phi \left({h_{\text{conv}} \left( T - T_{\text{env}} \right)
        + \varepsilon \sigma \left( {T}^4 - T_{\text{env}}^4 \right)}\right)
      \end{gather*}
      }
  \end{frame}

  \begin{frame}
    \frametitle{Discretization}
    \framesubtitle{Element activation}
    \begin{figure}
      \begin{subfigure}[t]{0.30\textwidth}
        \includegraphics[width=\textwidth]{schematic_melting/0.png}
        \caption{Bare substrate below an inactive powder layer.}
        \label{fig:refModelBareSubstrate}
      \end{subfigure}\hfill%
      \begin{subfigure}[t]{0.30\textwidth}
        \includegraphics[width=\textwidth]{schematic_melting/1.png}
        \caption{A powder layer is activated during a recoating step.}
        \label{fig:powderLayer}
      \end{subfigure}\hfill%
      \begin{subfigure}[t]{0.30\textwidth}
        \includegraphics[width=\textwidth]{schematic_melting/2.png}
        \caption{After a heating step, elements whose average temperature
        surpasses $T_m$ are set to bulk.}
        \label{fig:activPhaseChange3}
      \end{subfigure}\hfill%
      \caption{Illustration of deposition and melting processes.\qquad
        \legendpowderbulk{}
      }
      \label{fig:activPhaseChange}.
    \end{figure}
    Same treatment of phase change as in \citep{kollmannsberger2018}
    \todo{Maybe remove this slide, contradicting legend on next slide?}
  \end{frame}

  \begin{frame}
    \frametitle{Discretization}
    \begin{figure}
      \externalvod{width=0.8\textwidth}{videos/2dlpbf_2d_lpbf_ref.mp4}{thumbnail-2dlpbf_2d_lpbf_ref.png}
      \caption{Demo simulation of 2D LPBF with element activation.\\
        Wireframe elements (\;\wireframeTriangle{}) correspond to powder region.
      }
    \end{figure}
    \todo{Merge with previous slide?}
  \end{frame}

  \begin{frame}
    \frametitle{Part-scale simulation}
    \framesubtitle{Impossible}
    Previous slide: ``uniform'' mesh in part and powder region.
    Recall the volume scale ratio;
    In 3D, we would need
    \begin{equation}
      \label{eq:num_elements_uniform_mesh}
      \text{\# elements} = \mathcal{O}\left(\frac{L^3}{R^3}\right)
    \end{equation}
    to resolve the heat source throughout the print.

    In practice, no one uses uniform meshes for LPBF simulation;
    \textbf{AMR} is regarded as the \textbf{de facto standard}.
  \end{frame}

  \begin{frame}
    \frametitle{Part-scale simulation}
    \framesubtitle{Impossible}
    But what about time-steps?
    The interval of interest is $]0, T_{final}[$ with
    $$
    T_{final} = T_{print} + T_{cool}
    $$
    i.e. the net printing time plus cooling.
    The time scale disparity requires again \cref{eq:num_elements_uniform_mesh} time-steps
    \begin{gather}
      \label{eq:num_timesteps_uniform_mesh}
      \text{\# time-steps} > \frac{T_{print}}{T_{hs}} = \mathcal{O}\left(\frac{L^3}{R^3}\right)\\
    \end{gather}
    when discretizing with \textbf{uniform time-steps}.
  \end{frame}

  \begin{frame}
    \frametitle{Part-scale simulation}
    \framesubtitle{Impossible}
    Intuitively, if we don't respect the constraint
    \begin{equation}
      \label{eq:origconstraint}
      \Delta t \leq T_{hs}
    \end{equation}
    , we won't resolve the motion of the heat source.\\
    In practice, that's indeed what happens:
    if the time-step is larger than $T_{hs}$ i.e. the heat source
    travels more than $1 R$ per time-step,
    it skips over parts of the domain,
    and generates artificial temperature spikes.
    \begin{figure}
      \begin{subfigure}[t]{0.49\textwidth}
        \includegraphics[width=\textwidth]{timestep-2ths/1R.png}
        \caption{$\Delta t = 1 T_{hs}$}
        \label{fig:spots1R}
      \end{subfigure}
      \begin{subfigure}[t]{0.49\textwidth}
        \includegraphics[width=\textwidth]{timestep-2ths/2R.png}
        \caption{$\Delta t = 2 T_{hs}$}
        \label{fig:spots2R}
      \end{subfigure}
      % \caption{2D heating track example with admissible and inadmissible
      % time-step sizes according to inequality \eqref{eq:origconstraint}.}
      % \label{fig:spots}
    \end{figure}
  \end{frame}

  \begin{frame}
    \frametitle{Part-scale simulation}
    \framesubtitle{Impossible}
    So \eqref{eq:num_timesteps_uniform_mesh} is indeed a \textbf{lower bound} on the number of time-steps
    when using uniform time-stepping.

    \underline{Bad news:}
    \begin{itemize}
      \item \textbf{Uniform time-stepping} is the \textbf{de facto standard} in LPBF simulation.
      \item Some references require time-steps much smaller than $T_{hs}$
        to ensure stability and accuracy \citep{hodge2014, hodge2021, elahi2025}.
    \end{itemize}
    There are basically 4 (!) groups in the world that can run simulations in the order of $\mathcal{O}(10 \unit{\milli\meter})$:
    Pittsburgh, Northwestern, LLNL, TUM.\\
    Decimeter-scale parts are currently unfeasible.
    \todo{Expand}
  \end{frame}

  \setbeamertemplate{frame footer}{Computational strategies for \footlineHl{time-accurate} simulation part-scale LPBF}

  \begin{frame}
    \frametitle{Part-scale simulation}
    \framesubtitle{Lumped heat source}
    What do you do if you don't have a cluster or a nice GPU ?
    You \textbf{simplify the model}.

    \begin{figure}
      \centering
      \def\svgwidth{0.5\textwidth}%
      {\small
      \import{./figures/goldakProfile}{profile.pdf_tex}
      }
      \caption{Double ellipsoidal profile \citep{goldak1984}.
        $\xi$ is the welding direction.}
      \label{fig:goldak}
    \end{figure}
  \end{frame}

  \begin{frame}
    \frametitle{Part-scale simulation}
    \framesubtitle{Lumped heat source}
    \begin{figure}
      \centering
      \begin{tikzpicture}
        % include the original figure
        \node[inner sep=0] (img) {\def\svgwidth{0.5\textwidth}%
          {\small \import{./figures/goldakProfile}{profile.pdf_tex}}};

        % red "forbidden" cross over the whole image
        \draw[line width=2pt, red] ([shift={(-2mm,-2mm)}]img.south west)
                                   -- ([shift={(+2mm,+2mm)}]img.north east);
        \draw[line width=2pt, red] ([shift={(+2mm,-2mm)}]img.south east)
                                   -- ([shift={(-2mm,+2mm)}]img.north west);
      \end{tikzpicture}
    \end{figure}
    We give up on the accurate representation of the heat source
    since it is too costly to resolve.

    We use a \textbf{lumped heat source} instead.
  \end{frame}

  \begin{frame}
    \frametitle{Part-scale simulation}
    \framesubtitle{Lumped heat source}
    A time-step is chosen regardless of the heat source path.
    Elements that are intersected by the heat source path
    during the time-step are \textbf{heated uniformly}.
    \begin{figure}
      \centering
      \begin{subfigure}[t]{0.495\textwidth}
        \centering
        \def\svgwidth{\textwidth}
        \import{./figures/lumped_hs}{heated_els.pdf_tex}
        \caption{Elements heated uniformly during $[t^{n},\,t^{n+1}]$.\\
             \begin{tikzpicture}
               \draw[red, very thick, dashed] (0,0.3) -- (0.4,0.3);
             \end{tikzpicture}
             $\; \rightarrow \;$ Heat source path
           }
        \label{fig:lhs_heated_els}
      \end{subfigure}%
      \hfill%
      \begin{subfigure}[t]{0.48508\columnwidth}
        \centering
        \includegraphics[width=\textwidth]{lumped_hs/tem.png}
        \caption{Resulting temperature field.}
        \label{fig:lhs_temp_field}
      \end{subfigure}
      \caption{Lumped heat source example.}
    \end{figure}
  \end{frame}

  \begin{frame}
    \frametitle{Part-scale simulation}
    \framesubtitle{Lumped heat source}
    \begin{figure}
      \begin{subfigure}[t]{0.32\textwidth}
        \centering
        \includegraphics[width=\textwidth]{chiumenti2017-lhs/chiumenti2017multipleHatches.png}
        \caption{Multiple hatches per time-step.}
      \end{subfigure}%
      \hfill
      \begin{subfigure}[t]{0.32\textwidth}
        \centering
        \includegraphics[width=\textwidth]{chiumenti2017-lhs/chiumenti2017singleLayer.png}
        \caption{Single layer per time-step.}
      \end{subfigure}%
      \hfill
      \begin{subfigure}[t]{0.32\textwidth}
        \centering
        \includegraphics[width=\textwidth]{chiumenti2017-lhs/chiumenti2017multipleLayers.png}
        \caption{4 layers per time-step.}
      \end{subfigure}
      \caption{Lumped heat input strategies from \cite{chiumenti2017b}.}
    \end{figure}
    \textit{Pros and cons}
    \begin{itemize}
      \item[+] Feasible simulations
      \item[-] Distributed heat input $\;\longrightarrow\;$ temperatures below melt $\;\longrightarrow\;$ Numerical calibration (flash heating)
      \item[-] Loss of history: no resolution of heat source motion
    \end{itemize}
  \end{frame}

  \setbeamertemplate{frame footer}{\footlineHl{Computational strategies for time-accurate simulation part-scale LPBF}}

  \begin{frame}
    \frametitle{Problematic}
    \begin{itemize}
      \item AMR is regarded as necessary for part-scale LPBF simulation
        to adress spatial scale disparity.
      \item The time scale disparity is \textbf{equally} severe,
        yet uniform-time stepping is the standard.
    \end{itemize}
    $\implies$ Largely overlooked challenge:
    reduce the number of global time-steps required
    in part-scale LPBF simulation
    while retaining time-accuracy.
  \end{frame}

  \begin{frame}
    \frametitle{Objectives of this work}
    \begin{itemize}
      \item
        Review existing methods for addressing time-stepping
        challenges in MAM.
      \item
        Explore novel methods for reducing the number of time-steps
        required in LPBF modeling.
      \item
        Validate the proposed methods on realistic
        test cases.
      \item
        Validate the underlying physical models
        and numerical methods against experimental data.
      \item
        Provide realistic speedup estimates and practical implementation guidelines
        for the proposed methods.
      \item Ensure open science and reproducibility.
    \end{itemize}
  \end{frame}

  \setbeamertemplate{frame footer}{}
  \begin{frame}
    \frametitle{Domain decomposition}
    \framesubtitle{Basics}
    We've established that the part-scale build is driven
    by the motion of a very localized heat source.

    What if we treat differently this region aka the Heat Affected Zone (HAZ)?

    We will do so via \textbf{domain decomposition} (DD).

    Let $\Omega_1$ and $\Omega_2$ be a non-overlapping decomposition of
    the domain $\Omega = \Omega_1 \sqcup \Omega_2$ such that
    $\Omega_1$ contains the HAZ;
    we will pose different problems in $\Omega_1$ and $\Omega_2$
    and enforce \textbf{transmission conditions} at the interface
    $\Gamma = \partial \Omega_1 \cap \partial \Omega_2$
  \end{frame}

  \begin{frame}
    \frametitle{Domain decomposition}
    \framesubtitle{Basics}
  \begin{figure}
    \centering
    \def\svgwidth{1.0\columnwidth}
    \import{figures/transmission_conds/}{drawing.pdf_tex}
    \caption{Non-overlapping DD schematic for a welding problem. $\Omega_1$ and $\Omega_2$
    cover the HAZ and underlying substrate, respectively. Transmission conditions
    must be satisfied at the interface $\Gamma = \partial \Omega_1 \cap \partial \Omega_2$.
    Admissible and inadmissible transmission conditions are shown in the miniature plots,
    which represent temperature fields across $\Gamma$.
    }
    \label{fig:transmission_conditions}
  \end{figure}
  \end{frame}

  \begin{frame}
    \frametitle{Domain decomposition}
    \framesubtitle{Basics}
    Consider the linear heat equation with Dirichlet boundary conditions.
    Solve for $T_i$ in each subdomain $\Omega_i$:
    \begin{equation}
    \label{eq:lproblem}
    \left\{
    \begin{aligned}
      \partial_t T_i - \alpha \Delta T_i &= r && \forall \mathbf{x} \in \Omega_i\\
      T_i(\mathbf{x}, t) &= T_d && \forall \mathbf{x} \in \partial \Omega_{D,i}\\
      \mathcal{I}_i \left(T_i, T_j\right) &= 0 && \forall \mathbf{x} \in \Gamma, \enskip i \neq j
    \end{aligned}
    \right.
    \end{equation}
    This is equivalent to solving the original problem in $\Omega$ given
    that the \textbf{transmission conditions}
    are satisfied
    \begin{equation}
      \label{eq:interface_eq_conditions}
      \begin{cases}
        \mathcal{I}_1(T_1, T_2) = 0\\
        \mathcal{I}_2(T_1, T_2) = 0
      \end{cases}
      \iff
      \begin{cases}
        T_1 = T_2\\
        k_1 \partial_n T_1 = k_2 \partial_n T_2
      \end{cases}
      \qquad
      \qquad
      \forall \mathbf{x} \in \Gamma
    \end{equation}
  \end{frame}

  \begin{frame}
    \frametitle{Domain decomposition}
    \framesubtitle{Basics}
    Any linearly independent combination of \eqref{eq:interface_eq_conditions}
    lays a valid pair of subproblems on $\Omega_1$ and $\Omega_2$
    {\small
    \begin{equation}
      \label{eq:lp1}
      \left\{
      \begin{aligned}
        \rho c_p \partial_t T_1 - k \Delta T_1 &= r && \text{ in } \Omega_1\\
        T_1 &= T_d && \text{ in } \partial \Omega_{D,1}\\
        \gamma_1 T_1 + \eta_1 k_1 \frac{\partial T_1}{\partial n_1} &= \gamma_1 T_2 + \eta_1 k_2 \frac{\partial T_2}{\partial n_1} && \text{ in } \Gamma
      \end{aligned}
      \right.
    \end{equation}
    \begin{equation}
      \label{eq:lp2}
      \left\{
      \begin{aligned}
        \rho c_p \partial_t T_2 - k \Delta T_2 &= r && \text{ in } \Omega_2\\
        T_2 &= T_d && \text{ in } \partial \Omega_{D,2}\\
        \gamma_2 T_2 + \eta k_2 \frac{\partial T_2}{\partial n_2} &= \gamma_2 T_2 + \eta_2 k_2 \frac{\partial T_2}{\partial n_2} && \text{ in } \Gamma
      \end{aligned}
      \right.
    \end{equation}
    }
    \begin{itemize}
      \item $\gamma_1 = \eta_2 \neq 0$ and $\gamma_2 = \eta_1 = 0 \longrightarrow\;$ Dirichlet-Neumann

      \item $\gamma_i \neq 0$ and $\eta_i \neq 0 \longrightarrow\;$ Robin-Robin

      \item $\gamma_1 \neq 0$, $\eta_1 = 0$, $\gamma_2 \neq 0$ and $\eta_2 \neq 0
        \longrightarrow\;$ Dirichlet-Robin
    \end{itemize}
  \end{frame}

  \begin{frame}
    \frametitle{Domain decomposition}
    \framesubtitle{Basics}
    Assuming steady-state and applying FEM discretization
    on both subdomains, we obtain the following
    monolithic system:
    \renewcommand{\arraystretch}{1.2} % increases spacing (default is 1.0)
    \begin{equation}
      \label{eq:monolithic}
      \begin{pmatrix}
        \mathbf{A}_{11} & \mathbf{A}_{1\Gamma_1} & \cdot & \cdot\\
        \mathbf{A}_{\Gamma_1 1} & \mathbf{A}_{\Gamma_1 \Gamma_1} + \mathbf{R}_{11} & \cdot & \mathbf{R}_{12}\\
        \cdot & \cdot & \mathbf{A}_{22} & \mathbf{A}_{2\Gamma_2}\\
        \cdot & \mathbf{R}_{21} & \mathbf{A}_{\Gamma_2 2} & \mathbf{A}_{\Gamma_2 \Gamma_2} + \mathbf{R}_{22}
      \end{pmatrix}
      \begin{pmatrix}
        \mathbf{T}_1\\
        \mathbf{T}_{\Gamma_1}\\
        \mathbf{T}_2\\
        \mathbf{T}_{\Gamma_2}
      \end{pmatrix}
      =
      \begin{pmatrix}
        \mathbf{F}_1\\
        \mathbf{F}_{\Gamma_1}\\
        \mathbf{F}_2\\
        \mathbf{F}_{\Gamma_2}
      \end{pmatrix}
    \end{equation}
    $$
        \underbrace{\int_{\Omega_i} \nabla v_i \cdot k \nabla T_i}_{\mathbf{A}}\,
      - \underbrace{\int_{\Gamma} v_i \gamma_i T_i}_{\mathbf{R}_{ii}}\,
      = \underbrace{\int_{\Omega_i} r v_i\,}_{\mathbf{F}}\,
      - \underbrace{\int_{\Gamma} v_i \left(\gamma_i T_j + \eta_i k \partial_n T_j\right)}_{\mathbf{R}_{ij}}
    $$
    Solving \eqref{eq:monolithic} directly $\; \longrightarrow\;$ \textbf{monolithic} approach
  \end{frame}

  \begin{frame}
    \frametitle{Domain decomposition}
    \framesubtitle{Basics}
    
    Solving problem on $\Omega_1$
    \renewcommand{\arraystretch}{1.2} % increases spacing (default is 1.0)
    \begin{equation}\label{eq:staggered1}
      \begin{pmatrix}
        \mathbf{A}_{11} & \mathbf{A}_{1\Gamma_1}\\
        \mathbf{A}_{\Gamma_1 1} & \mathbf{A}_{\Gamma_1 \Gamma_1} + \mathbf{R}_{11}
      \end{pmatrix}
      \begin{pmatrix}
        \mathbf{T}^{k+1}_1\\
        \mathbf{T}^{k+1}_{\Gamma_1}
      \end{pmatrix}
      =
      \begin{pmatrix}
        \mathbf{F}_1\\
        \mathbf{F}_{\Gamma_1} - \mathbf{R}_{12} \mathbf{T}^k_{\Gamma_2}
      \end{pmatrix}
    \end{equation}
    and feeding the interface values to the problem on $\Omega_2$
    \begin{equation}\label{eq:staggered2}
      \begin{pmatrix}
        \mathbf{A}_{22} & \mathbf{A}_{2\Gamma_2}\\
        \mathbf{A}_{\Gamma_2 2} & \mathbf{A}_{\Gamma_2 \Gamma_2} + \mathbf{R}_{22}
      \end{pmatrix}
      \begin{pmatrix}
        \mathbf{T}^{k+1}_2\\
        \mathbf{T}^{k+1}_{\Gamma_2}
      \end{pmatrix}
      =
      \begin{pmatrix}
        \mathbf{F}_2\\
        \mathbf{F}_{\Gamma_2} - \mathbf{R}_{21} \mathbf{T}^{k+1}_{\Gamma_1}
      \end{pmatrix}
    \end{equation}
    \renewcommand{\arraystretch}{1.0}
    and iterating until convergence $\; \longrightarrow\;$ \textbf{staggered} approach
  \end{frame}

  \begin{frame}
    \frametitle{Advected subdomain}
    Resolving the motion requires prohibitively small time-steps.

    Idea: \textbf{Get rid of the motion}!

    Define a subdomain $\Omega_m$ that moves with the heat source
    and solves the heat equation in its reference frame.

    Due to motion of reference frame, the heat equation in $\Omega_m$
    acquires an advection term $\longrightarrow$ \textbf{advected subdomain}
    (AS) method.
  \end{frame}

  \begin{frame}
    \frametitle{Advected subdomain}
    \begin{figure}
      \centering
      \def\svgwidth{1.05\columnwidth}
      \small
      \import{./figures/chimera_schematic/}{chimera_schematic.pdf_tex}
      \caption{Schematic of Robin-Robin variant of advected subdomain method.}
      \label{fig:chimera_schematic}
    \end{figure}
  \end{frame}

  \begin{frame}
    \frametitle{Advected subdomain}
    
      {\small
      \begin{gather*}
        (\mathbf{x}, t) \longrightarrow (\boldsymbol{\xi}, \eta) \thinspace , \thinspace
        \begin{cases}
          \boldsymbol{\xi} = \mathbf{x} - \int_0^t \mathbf{v}\dependson{t} dt\\
          \eta = t
        \end{cases}
      \end{gather*}
      \begin{gather*}
        \Longrightarrow
        \begin{cases}
          \partial_{x_i} &= \partial_{\xi_i} \Rightarrow \nabla_{\mathbf{x}} = \nablaxi\\
          \partial_t &= \partial_\eta - v_i \partial_{\xi_i} = \partial_\eta - \mathbf{v}\dependson{t} \cdot \nablaxi
        \end{cases}
      \end{gather*}

      Find $T_m : \Omega_m(t) \times [0, T_{\text{final}}] \to \mathbb{R}$
      and $T_f : \Omega_f(t) \times [0, T_{\text{final}}] \to \mathbb{R}$
      such that
      \begin{align*}
        \rho c_p \Big( \partial_t T^m{\scriptstyle (\xi, t)} - \mathbf{v}\dependson{t} \cdot \nabla T^m{\scriptstyle (\xi, t)} \Big) -
        \nabla \cdot ( k \nabla T^m{\scriptstyle (\xi, t)}) &= r{\scriptstyle (\xi,t)}  &\xi &\in \Omega_m(t)
      \end{align*}
      \begin{align*}
      \rho c_p \partial_t T^f{\scriptstyle (x, t)} - \nabla \cdot (k \nabla T^f{\scriptstyle (x, t)}) &= 0 \qquad &x &\in \Omega_f(t)
      \end{align*}
      }
  \end{frame}

  \begin{frame}
    \frametitle{Advected subdomain}
    \cite{slimani2024}: Dirichlet-Neumann multi-mesh variant of advected
    subdomain method for linear heat equation.
    \begin{figure}[h]
      \centering
      \includegraphics[width=0.6\textwidth]{quadTriang.png}
      \caption{A finer mesh can be attached to the heat source
      to adress spatial scale disparity: not done here.}
    \end{figure}
    \begin{itemize}
      \item VMS stabilization to handle advection in $\Omega_m$
      \item Monolithic solution of the coupled problem
      \item Serial C++ implementation available on GitHub.
    \end{itemize}
  \end{frame}

  \begin{frame}
    \frametitle{Advected subdomain}
    Goal workflow:
    \begin{gather*}
      \texttt{steadiness\_metric}(T_m) := \frac{\eunorm{T_m\big|_{t^{n+1}} - T_m\big|_{t^{n}}}}{\eunorm{T_m\big|_{t^{n+1}}}} < \epsilon\\
      \big\Downarrow\\
      \text{Increase } \Delta t \text{ for } t^{n+2} = t^{n+1} + \Delta t\\
      \big\Downarrow\\
      \textrm{Resize } \Omega_m
    \end{gather*}
  \end{frame}

  \begin{frame}
    \frametitle{Advected subdomain}
    \framesubtitle{Non-trivial workflow}
    \begin{figure}
      \begin{subfigure}[t]{0.36\textwidth}
        \includegraphics[width=\textwidth]{2d_meshing_example/afterShaping.png}
        \caption{$\mathcal{T}^m \leftarrow \texttt{shapeSubdomain}(\mathcal{T}^m_{bg})$}
      \end{subfigure}
      \hfill%
      \begin{subfigure}[t]{0.36\textwidth}
        \includegraphics[width=\textwidth]{2d_meshing_example/after_intersec1.png}
        \caption{$\mathcal{T}^m \leftarrow \texttt{intersect}\big(\mathcal{T}^m, \mathcal{T}^f \textrm{ at } t^n\big)$}
      \end{subfigure}
      \begin{subfigure}[t]{0.36\textwidth}
        \includegraphics[width=\textwidth]{2d_meshing_example/after_intersec2.png}
        \caption{$\mathcal{T}^m \leftarrow \texttt{intersect}\big(\mathcal{T}^m, \mathcal{T}^f\big) \textrm{ at } t^{n+1}$}
      \end{subfigure}
      \hfill%
      \begin{subfigure}[t]{0.36\textwidth}
        \includegraphics[width=\textwidth]{2d_meshing_example/dd.png}
        \caption{$\mathcal{T}^f \leftarrow \texttt{subtract}\big(\mathcal{T}^f, \mathcal{T}^m\big)$}
      \end{subfigure}
      \label{fig:2d_meshing_example}
    \end{figure}
  \end{frame}

  % \begin{frame}
  %   \frametitle{Advected subdomain}
  %   \begin{table}
  %     \centering
  %     \begin{tabular}{lrl}
  %       \toprule
  %       Parameter & Value & Unit \\
  %       \midrule
  %       Density  & 4300 & $kg / m^3$\\
  %       Specific heat  & 700 & $J / kg K$\\
  %       Conductivity  & 10 & $W / m K$\\
  %       Speed $V$  & 1000 & $mm / s$\\
  %       Radius $R$  & 0.1 & $mm$\\
  %       Power  & 50 & \unit{\watt}\\
  %       Environment temperature  & 25 & \unit{\celsius}\\
  %       \bottomrule
  %     \end{tabular}
  %     \label{tbl:mat}
  %   \end{table}
  % \end{frame}

  \begin{frame}
    \frametitle{Advected subdomain}
    Analytical validation with 3D welding example
    \citep{nguyen1999,vanelsen2007}
    and Ti64-like material properties.
    \vspace{-5mm}
    \begin{figure}[]
        \centering
        \begin{tabular}{cc}
    \begin{subfigure}[t]{0.49\linewidth}
      \centering
      \includegraphics[width=\textwidth]{3d_weldingLpbf/analytical.png}
      \caption{Analytical}
    \end{subfigure} &
    \addtocounter{subfigure}{2}
    \begin{subfigure}[t]{0.49\linewidth}
      \centering
      \includegraphics[width=\textwidth]{3d_weldingLpbf/allContours.png}
      \definecolor{referenceColor}{HTML}{1762FB}
      \definecolor{proposedColor}{HTML}{00AA00}
      \vspace{-0.7cm}
      \caption{
           ({\color{black} \rule[-1.5 pt]{8 pt}{8 pt}}) Analytical \;\;
           ({\color{proposedColor} \rule[-1.5 pt]{8 pt}{8 pt}}) AS \;\;
           ({\color{referenceColor} \rule[-1.5 pt]{8 pt}{8 pt}}) Reference \;\;
           \\
        Isothermals and position of maximum temperature.
      }
      \label{fig:3dweldallcontours}
    \end{subfigure} \\
    \addtocounter{subfigure}{-3}
    \begin{subfigure}[t]{0.49\linewidth}
      \centering
      \includegraphics[width=\textwidth]{3d_weldingLpbf/coupled_elsPerRad8.png}
      \caption{Advected subdomain (AS), $\Delta t = 2\,T_{hs}$}
      \label{fig:3dweldcontourprop}
    \end{subfigure} & 
    \addtocounter{subfigure}{2}
    \begin{subfigure}[t]{0.49\linewidth}
      \centering
      \includegraphics[width=\textwidth]{3d_weldingLpbf/err_coupled_elsPerRad8.png}
      \caption{Pointwise error of \ref{fig:3dweldcontourprop}. Max error is $\approx \mathbf{12}$}
    \end{subfigure} \\
    \addtocounter{subfigure}{-3}
    \begin{subfigure}[t]{0.49\linewidth}
      \centering
      \includegraphics[width=\textwidth]{3d_weldingLpbf/reference_elsPerRad8_tstepsPerRad4.png}
      \caption{Reference method, $\Delta t = 0.25\,T_{hs}$}
      \label{fig:3dweldcontourref}
    \end{subfigure} & 
    \addtocounter{subfigure}{2}
    \begin{subfigure}[t]{0.49\linewidth}
      \centering
      \includegraphics[width=\textwidth]{3d_weldingLpbf/err_reference_elsPerRad8_tstepsPerRad4.png}
      \caption{Pointwise error of \ref{fig:3dweldcontourref}. Max error is $\approx \mathbf{263}$}
      \label{fig:3dweldpointwiseref}
    \end{subfigure} \\
        \end{tabular}
      \caption{Solution (left) and error (right) contour at steady-state.}
    \end{figure}
  \end{frame}

  \begin{frame}
    \frametitle{Advected subdomain}
      \begin{figure}
        \centering
        \externalvod{width=0.8\textwidth}{videos/chimera.mp4}{2d_welding_chimera/thumbnail-chimera.png}
        \vspace{-3mm}
        \caption{2D welding example with growing time-step workflow.}
      \end{figure}
      \vspace{-3mm}
      \begin{figure}
        \centering
        \begin{subfigure}[t]{0.3\linewidth}
          \centering
          \stackinset{l}{ 0.35\textwidth}{b}{ 0.025\textwidth}{
            {\color{white}$\Omega_f$}
          }{
            \includegraphics[width=\textwidth]{2d_welding_chimera/changeOfDir_OFF.png}
          }
          \caption{No AS, $\Delta t = 0.5\,T_{hs}$.}
          \label{fig:2dWeldNoShear}
        \end{subfigure}
        \hfill
        \begin{subfigure}[t]{0.3\linewidth}
          \centering
          \stackinset{r}{ 0.32\textwidth}{b}{ 0.1\textwidth}{
            {\color{white}$\Gamma$}
          }{
          \stackinset{r}{ 0.19\textwidth}{t}{ 0.075\textwidth}{
            {\color{white}$\Omega_m$}
          }{
          \stackinset{l}{ 0.35\textwidth}{b}{ 0.025\textwidth}{
            {\color{white}$\Omega_f$}
          }{
          \includegraphics[width=\textwidth]{2d_welding_chimera/changeOfDir_ON.png}
          }}}
          \caption{AS, $\Delta t = 0.5\,T_{hs}$.}
          \label{fig:2dWeldShear_05R}
        \end{subfigure}
        \hfill
        \begin{subfigure}[t]{0.3\linewidth}
          \centering
          \stackinset{r}{ 0.32\textwidth}{b}{ 0.1\textwidth}{
            {\color{white}$\Gamma$}
          }{
          \stackinset{r}{ 0.19\textwidth}{t}{ 0.075\textwidth}{
            {\color{white}$\Omega_m$}
          }{
          \stackinset{l}{ 0.35\textwidth}{b}{ 0.025\textwidth}{
            {\color{white}$\Omega_f$}
          }{
          \includegraphics[width=\textwidth]{2d_welding_chimera/changeOfDir_ON_finer.png}
          }}}
          \caption{AS, $\Delta t = 0.25\,T_{hs}$.}
          \label{fig:2dWeldShear_025R}
        \end{subfigure}
        \vspace{-3mm}
        \caption{``Shearing'' of previous thermal tail by the AS.}
        \label{fig:2dWeldTurn}
      \end{figure}
  \end{frame}

  \begin{frame}
    \frametitle{Advected subdomain}
    \framesubtitle{Thin wall}
    \begin{figure}
      \centering
      \begin{subfigure}[t]{0.48\textwidth}
        \includegraphics[draft=\isdraft, width=\textwidth]{3d_lpbf_chimera/overview1_compressed.png}
        \caption{Mesh and bulk-powder distribution at final time-step.\\
        % \legendMats
        }
        \label{fig:3dLpbfMeshMat}
      \end{subfigure}\hfill%
      \begin{subfigure}[t]{0.52\textwidth}
        \centering
        \includegraphics[width=\textwidth]{3d_lpbf_chimera/overview2.png}
        \caption{Temperature contour at intermediate time-step, metal only.
        last $\Delta t = 8 T_{hs}$, \enskip $\Gamma$ extends $50 R$ behind
        heat source.}
        \label{fig:3dLpbfPerspective}
      \end{subfigure}%
      \vspace{-3mm}
      \caption{Mesh and $\Omega_m$.}
    \end{figure}
    One track of $100R$ per layer, 10 layers, Ti64-like material properties.
  \end{frame}

  \begin{frame}
    \frametitle{Advected subdomain}
    \framesubtitle{Thin wall}
    \begin{minipage}[t]{0.58\textwidth}
    \pgfplotsset{compat=newest}
    \begin{figure}
      \centering
      \begin{tikzpicture}[
          trim axis left,
          trim axis right,
          baseline,
      ]

        \begin{axis}[
              clip=true,
              enlarge x limits=false,
              enlarge y limits=false,
              width=1.21\textwidth,
              grid=major,
              xlabel={x},
              x unit=\unit{\milli\meter},
              ylabel={T},
              y unit=\unit{\celsius},
              xmin=-6.0,
              xmax=+6.0,
              ymin=20.0,
              ymax=2575.0,
              ytick={25,500,1000,2000,2500},
              extra x ticks={-5, 5},
              every extra y tick/.append style={overlay,},
              legend cell align=left,
              legend pos=north west,
              legend style={font=\footnotesize},
              tick label style={font=\footnotesize},
              ylabel style={yshift=-2mm, font=\footnotesize, at={(axis description cs:-0.05,0.6)}},
              xlabel style={font=\footnotesize, at={(axis description cs:0.5,-0.1)}},
          ]
        \addlegendentry{AS, last $\Delta t = 8\,T_{hs}$}
        \addplot
        [mark=none, color=red, opacity=0.6, thin]
        table [col sep=comma, x="Points:0", y="T"] {figures/3d_lpbf_chimera/plots/endFifthLayer_coupled_rerun.csv};

        \addlegendentry{Reference, $\Delta t = 0.5\,T_{hs}$}
        \addplot
        [mark=none, color=blue, opacity=0.6, thin]
        table [col sep=comma, x="Points:0", y="T"] {figures/3d_lpbf_chimera/plots/endFifthLayer_ref_normal.csv};

        \addlegendentry{Reference, $\Delta t = 0.125\,T_{hs}$}
        \addplot
        [mark=none, color=black, opacity=0.6, thin]
        table [col sep=comma, x="Points:0", y="T"] {figures/3d_lpbf_chimera/plots/endFifthLayer_ref_fine.csv};

        \addplot
        [forget plot, color=red, thin, dashed, opacity=0.6]
        coordinates
        {(0.93, \pgfkeysvalueof{/pgfplots/ymin})
         (0.93, \pgfkeysvalueof{/pgfplots/ymax})};

        \addplot
        [forget plot, color=red, thin, dashed, opacity=0.6]
        coordinates
        {(5.0, \pgfkeysvalueof{/pgfplots/ymin})
         (5.0, \pgfkeysvalueof{/pgfplots/ymax})};
      \end{axis}

      \end{tikzpicture}
      \vspace{-3mm}
      \caption{Plot along top centerline of build towards end of fifth layer.}
      \label{fig:3dLpbf_plot5}
    \end{figure}
    \end{minipage}%
    \hspace{-0.01\textwidth}%
    \begin{minipage}[t]{0.43\textwidth}
      \small
      \begin{itemize}
        \item About 6 times fewer time-steps with AS for the same build.
        \item Each AS time-step is roughly 2 times more expensive to compute.
        \item[$\rightarrow$] Overall net speedup of about 3 times in wall-clock time.
        \item Cost of boolean operations is negligible; overhead mainly comes
              from the increased linear solve time.
      \end{itemize}
    \end{minipage}
  \end{frame}

  \begin{frame}
    \frametitle{Advected subdomain}
    \framesubtitle{Intermediate conclusions}
    Conclusions of \cite{slimani2024}:
    \begin{itemize}
       \item AS recovers accurate thermal profiles in the HAZ and enables
         larger time-steps.
       \item Trade-off: increased algebraic complexity and potential
         oscillations at subdomain discontinuities.
       \item Performance degrades for short scanning tracks; minimum track
         length $\approx 5R$ recommended.
       \item Future work: combine with substepping algorithms and explore
         alternative domain decomposition strategies.
    \end{itemize}
  \end{frame}

  \begin{frame}
    \frametitle{Substepping}
    \begin{figure}
      \centering
      \def\svgwidth{1.0\columnwidth} {\small \import{./figures/substepping/}{schematic.pdf_tex}}
      \caption{Non-overlapping substepping decomposition. The HAZ is included in the \textbf{fast partition} $\Omega_f$.
      The remainder of the domain belongs to the \textbf{slow partition} $\Omega_s$.}
      \label{fig:substepping}
    \end{figure}
  \end{frame}

  \begin{frame}
    \frametitle{Substepping}
      \begin{figure}
        \centering
        \externalvod{width=1.0\textwidth}{videos/basic-demo-ss.mp4}{thumbnail-basic-ss.png}
        \caption{Basic demo of substepping method.}
      \end{figure}
  \end{frame}

  \begin{frame}
    \frametitle{Substepping}
    \cite{slimani2025}:
    \begin{itemize}
      \item Examines SOA of substepping methods for LPBF;
        identifies common structures and shortcomings.
      \item New Robin-Robin substepper for LPBF is compared against
        Dirichlet substepper of \cite{hodge2021}.
      \item Realistic speedup estimates provided.
      \item Alternate predictor schemes are proposed and tested.
      \item Robin-Robin version of AS method is proposed.
      \item AS is nested within the fast partition of said substeppers.
    \end{itemize}
  \end{frame}

  \begin{frame}
    \frametitle{Substepping + advected subdomain}
    \begin{figure}
      \centering
      \externalvod{width=1.0\textwidth}{videos/basic-demo-css.mp4}{thumbnail-basic-css.png}
      \caption{Basic demo of combined substepping + advected subdomain method.}
    \end{figure}
  \end{frame}

  \begin{frame}
    \frametitle{Substepping}
    \framesubtitle{Example time-step}
    \begin{figure}
      \begin{subfigure}[t]{0.46\textwidth}
        \centering
        \includegraphics[width=\textwidth]{{ss_robin/prev_sol.png}}
        \caption{$T^{n}$}
        \label{fig:robin_prev}
      \end{subfigure}%
      \hfill%
      \begin{subfigure}[t]{0.46\textwidth}
        \centering
        \includegraphics[width=\textwidth]{{ss_robin/predictor.png}}
        \caption{$\tilde{T}^{n+1}$ after predictor step}
        \label{fig:robin_predictor}
      \end{subfigure}
      \caption{Previous solution and predictor step.}
    \end{figure}
  \end{frame}

  \begin{frame}
    \frametitle{Substepping}
    \framesubtitle{Example time-step}
    \begin{figure}
    \begin{subfigure}[t]{0.46\textwidth}
      \centering
      \includegraphics[width=\textwidth]{{ss_robin/micro_step.png}}
      \caption{Intermediate \textbf{Robin} substep.}
      \label{fig:robin_substep}
    \end{subfigure}%
    \hfill%
    \begin{subfigure}[t]{0.46\textwidth}
      \centering
      \begin{tikzpicture}
        \node[anchor=south west, inner sep=0] (A) {\includegraphics[width=\textwidth]{{ss_robin/macro_step.png}}};
      \begin{scope}[x={(A.south east)},y={(A.north west)}]
        % Rectangle highlight (coordinates between 0 and 1)
        %\draw[teal!20, step=0.05] (0,0) grid (1,1);%grid, comment out for draft
        \draw[red, thick, dash pattern=on 1pt off 1pt] (0.286, 0.5475) circle (0.02);
        \draw[red, thick, dash pattern=on 1pt off 1pt] (0.286, 0.6385) circle (0.02);
      \end{scope}
      \end{tikzpicture}
      \caption{\textbf{Robin} corrector step.}
      \label{fig:robin_corrector}
    \end{subfigure}
    \caption{Robin-Robin substepper.}
    \end{figure}
  \end{frame}

  \begin{frame}
    \frametitle{Substepping}
    \framesubtitle{Example time-step}
    \begin{figure}
      \begin{subfigure}[t]{0.46\textwidth}
        \centering
        \begin{tikzpicture}
          \node[anchor=south west, inner sep=0] (A) {\includegraphics[width=\textwidth]{{hodge/micro_step.png}}};
        \begin{scope}[x={(A.south east)},y={(A.north west)}]
          % Rectangle highlight (coordinates between 0 and 1)
          %\draw[teal!20, step=0.01] (0,0) grid (1,1);%grid, comment out for draft
          \draw[red, thick, dash pattern=on 1pt off 1pt, rounded corners] (0.0625, 0.450) rectangle (0.2, 0.5);
          \draw[red, thick, dash pattern=on 1pt off 1pt, rounded corners] (0.2500, 0.5505) rectangle (0.3100, 0.7100);
        \end{scope}
        \end{tikzpicture}
        \caption{Intermediate \textbf{Dirichlet} substep. The Dirichlet
        condition does not let the solution deviate from the predictor.}
        \label{fig:hodge_substep}
      \end{subfigure}%
      \hfill%
      \begin{subfigure}[t]{0.46\textwidth}
        \centering
        \begin{tikzpicture}
          \node[anchor=south west, inner sep=0] (A) {\includegraphics[width=\textwidth]{{hodge/macro_step.png}}};
        \begin{scope}[x={(A.south east)},y={(A.north west)}]
          % Rectangle highlight (coordinates between 0 and 1)
          %\draw[teal!20, step=0.01] (0,0) grid (1,1);%grid, comment out for draft
          \draw[red, thick, dash pattern=on 1pt off 1pt, rounded corners] (0.065, 0.4550) rectangle (0.1905, 0.4905);
          \draw[red, thick, dash pattern=on 1pt off 1pt, rounded corners] (0.2600, 0.5605) rectangle (0.3000, 0.7000);
        \end{scope}
        \end{tikzpicture}
    \caption{\textbf{Dirichlet} corrector step.}
    \label{fig:hodge_corrector}
  \end{subfigure}
      \caption{Dirichlet substepper of \cite{hodge2021}.}
    \end{figure}
  \end{frame}

  \begin{frame}
    \frametitle{Substepping}
    \framesubtitle{No heat source predictor}

    \begin{figure}
      \begin{subfigure}[t]{0.33\textwidth}
        \centering
        \includegraphics[width=\textwidth]{hodge/to_predictor.png}
        \caption{Predictor step without heat input}
        \label{fig:to_hodge_predictor}
      \end{subfigure}%
      \hfill
      \begin{subfigure}[t]{0.33\textwidth}
        \centering
        \includegraphics[width=\textwidth]{hodge/to_micro_step.png}
        \caption{Micro-step.}
        \label{fig:to_hodge_micro}
      \end{subfigure}%
      \hfill
      \begin{subfigure}[t]{0.33\textwidth}
        \centering
        \includegraphics[width=\textwidth]{hodge/to_macro_step.png}
        \caption{Corrector step.}
        \label{fig:to_hodge_macro}
      \end{subfigure}
      \caption{Example of Dirichlet substepper step using a predictor step without including the source term.}
      \label{fig:to_hodge}
    \end{figure}
  \end{frame}

  \begin{frame}
    \frametitle{Substepping}
    \framesubtitle{2D square track}

      \begin{adjustwidth}{-10mm}{-10mm}
      \begin{figure}
      \centering
      \begin{tabular}{cccc}
        \begin{subfigure}[t]{0.28\textwidth}
          \centering
          \begin{tikzpicture}
            \node[anchor=south west, inner sep=0] (image) at (0, 0) {
              \includegraphics[width=\textwidth, clip=true, trim={0cm 0cm 12cm 0cm}]{{2d_square_track/sols/linear_ref_pred0_16els_tf0_5_ts8_0_time0_0032.png}}
            };
          \end{tikzpicture}
          \caption{Solution -- No substepping.}
          \label{fig:sol_ref_t4}
        \end{subfigure}
        &
        \begin{subfigure}[t]{0.28\textwidth}
          \centering
          \begin{tikzpicture}
            \node[anchor=south west, inner sep=0] (image) at (0, 0) {
              \includegraphics[width=\textwidth, clip=true, trim={0cm 0cm 12cm 0cm}]{{2d_square_track/sols/robin_linear_ss_rr_pred0_16els_tf0_5_ts8_0_time0_0032.png}}
            };
          \node[text opacity=1] at ($(image.west)!0.35!(image.east) + (0.0, 0.0)$) {\labelbox{$\Omega_f$}};
          \end{tikzpicture}
          \caption{Solution -- Robin.}
          \label{fig:sol_robin_t4}
        \end{subfigure}
        &
        \begin{subfigure}[t]{0.28\textwidth}
          \centering
          \begin{tikzpicture}
            \node[anchor=south west, inner sep=0] (image) at (0, 0) {
              \includegraphics[width=\textwidth, clip=true, trim={0cm 0cm 12cm 0cm}]{{2d_square_track/sols/chimera_ss_RR_linear_css_pred0_16els_tf0_5_ts8_0_time0_0032.png}}
            };
            \node[text opacity=1] at ($(image.west)!0.35!(image.east) + (0.0, 0.0)$) {\labelbox{$\Omega_f$}};
            \node[text opacity=1, anchor=south, inner sep=1pt] at ($(image.south west)!0.36!(image.south east) + (0.0, 0.0)$) {\labelbox{$\Omega_m$}};
          \end{tikzpicture}
          \caption{Solution -- Robin with AS.}
          \label{fig:sol_chimera_robin_t4}
        \end{subfigure}
        &
        \begin{subfigure}[t]{0.1\textwidth}
          \raisebox{0.1\height}{
            \vcolorbar{rainbow_blended_white}{0.2cm}{25}{2000}{$u_h$}{}
          }
        \end{subfigure}
        \\[5mm]
        \begin{subfigure}[t]{0.28\textwidth}
          \centering
          \includegraphics[width=\textwidth, clip=true, trim={0cm 0cm 12cm 0cm}]{{2d_square_track/errs/linear_ref_pred0_16els_tf0_5_ts8_0_time0_0032.png}}
          \caption{Squared error -- No substepping.}
          \label{fig:err_ref_t4}
        \end{subfigure}
        &
        \begin{subfigure}[t]{0.28\textwidth}
          \centering
          \includegraphics[width=\textwidth, clip=true, trim={0cm 0cm 12cm 0cm}]{{2d_square_track/errs/robin_linear_ss_rr_pred0_16els_tf0_5_ts8_0_time0_0032.png}}
          \caption{Squared error -- Robin.}
          \label{fig:err_robin_t4}
        \end{subfigure}
        &
        \begin{subfigure}[t]{0.28\textwidth}
          \centering
          \includegraphics[width=\textwidth, clip=true, trim={0cm 0cm 12cm 0cm}]{{2d_square_track/errs/chimera_ss_RR_linear_css_pred0_16els_tf0_5_ts8_0_time0_0032.png}}
          \caption{Squared error -- Robin with AS.}
          \label{fig:err_chimera_robin_t4}
        \end{subfigure}
        &
        \begin{subfigure}[t]{0.1\textwidth}
          \raisebox{0.1\height}{
          \vcolorbar{fast}{0.2cm}{0}{10000}{$(u_h - u_{ex})^2$}{}
          }
        \end{subfigure}
      \end{tabular}
      % \caption{Solutions (top row) and pointwise squared errors (bottom row) in the linear case for the Robin substepper
      %   with and without an AS, with $\Delta t_s = 8 T_{hs}$ and $\Delta t_f = 0.5 T_{hs}$.
      %   The unsubstepped solution with $\Delta t = 0.5 T_{hs}$ is shown for reference.
      %   The substeppers introduce some error throughout the domain.
      %   The advected subdomain reduces the error in the HAZ but increases the error near the track turns.
      % }
      \label{fig:err_linear_vsadvected}
    \end{figure}
    \end{adjustwidth}
  \end{frame}
  
  \begin{frame}
    \frametitle{Substepping}
    \framesubtitle{2D square track}
    \begin{figure}[ht]
  \centering
  \pgfplotslegendfromname{legendmidline}\\[1em]
  \begin{subfigure}[t]{0.5\linewidth}
    \centering
    \begin{tikzpicture}[
        trim axis left,
        trim axis right,
        baseline,
    ]
    \begin{axis}[
      width=1.0\linewidth,
      oscillationstyle,
      legend to name=legendmidline,
      legend cell align=left,
      legend style={font=\footnotesize,
        /tikz/every even column/.append style={column sep=0.5cm},
        legend columns=3,
        transpose legend,
      },
      legend image post style={xscale=0.8, yscale=0.8},
    ]
      \addlegendentry{Dirichlet, $\Delta t_s = 4 T_{hs}$}
      \addplot+
      [mark=none, opacity=0.8]
      table [col sep=comma, x=x, y=uh] {./plots/2dsquare_midline/midline_CSVs/smslinear-Ths=4-midline.csv};

      \addlegendentry{Dirichlet, $\Delta t_s = 8 T_{hs}$}
      \addplot+
      [mark=none, opacity=0.8]
      table [col sep=comma, x=x, y=uh] {./plots/2dsquare_midline/midline_CSVs/smslinear-Ths=8-midline.csv};

      \addlegendentry{Dirichlet, $\Delta t_s = 16 T_{hs}$}
      \addplot+
      [mark=none, opacity=0.8]
      table [col sep=comma, x=x, y=uh] {./plots/2dsquare_midline/midline_CSVs/smslinear-Ths=16-midline.csv};

      \addlegendentry{Robin, $\Delta t_s = 4 T_{hs}$}
      \addplot+
      [mark=none, opacity=0.8, unbounded coords=jump]
      table [col sep=comma, x=x, y=uh] {{plots/2dsquare_midline/midline_CSVs/ssrobinlinear-Ths=4-midline.csv}};

      \addlegendentry{Robin, $\Delta t_s = 8 T_{hs}$}
      \addplot+
      [mark=none, opacity=0.8, unbounded coords=jump]
      table [col sep=comma, x=x, y=uh] {{plots/2dsquare_midline/midline_CSVs/ssrobinlinear-Ths=8-midline.csv}};

      \addlegendentry{Robin, $\Delta t_s = 16 T_{hs}$}
      \addplot+
      [mark=none, opacity=0.8, unbounded coords=jump]
      table [col sep=comma, x=x, y=uh] {{plots/2dsquare_midline/midline_CSVs/ssrobinlinear-Ths=16-midline.csv}};
    \end{axis}
    \end{tikzpicture}
    \caption{Linear case.}
    \label{fig:midline_linear}
  \end{subfigure}%
  \begin{subfigure}[t]{0.5\linewidth}
    \centering
    \begin{tikzpicture}[
        trim axis left,
        trim axis right,
        baseline,
    ]
    \begin{axis}[
      width=1.0\linewidth,
      oscillationstyle,
      yticklabel={$$},
      ylabel={},
      y unit={},
    ]
      \addplot+
      [mark=none, opacity=0.8]
      table [col sep=comma, x=x, y=uh] {./plots/2dsquare_midline/midline_CSVs/smsnnlinear-Ths=4-midline.csv};
      \addplot+
      [mark=none, opacity=0.8]
      table [col sep=comma, x=x, y=uh] {./plots/2dsquare_midline/midline_CSVs/smsnnlinear-Ths=8-midline.csv};
      \addplot+
      [mark=none, opacity=0.8]
      table [col sep=comma, x=x, y=uh] {./plots/2dsquare_midline/midline_CSVs/smsnnlinear-Ths=16-midline.csv};

      \addplot+
      [mark=none, opacity=0.8, unbounded coords=jump]
      table [col sep=comma, x=x, y=uh] {{plots/2dsquare_midline/midline_CSVs/ssrobinnnlinear-Ths=4-midline.csv}};

      \addplot+
      [mark=none, opacity=0.8, unbounded coords=jump]
      table [col sep=comma, x=x, y=uh] {{plots/2dsquare_midline/midline_CSVs/ssrobinnnlinear-Ths=8-midline.csv}};

      \addplot+
      [mark=none, opacity=0.8, unbounded coords=jump]
      table [col sep=comma, x=x, y=uh] {{plots/2dsquare_midline/midline_CSVs/ssrobinnnlinear-Ths=16-midline.csv}};
    \end{axis}
    \end{tikzpicture}
    \caption{Non-linear case.}
    \label{fig:midline_nnlinear}
  \end{subfigure}

  \caption{Solution profiles along first heating track
  for increasing $\Delta t_s$ and constant $\Delta t_f = \tfrac{1}{8} T_{hs}$,
  for both substeppers without AS.}
  \label{fig:1sttrackmidline}
\end{figure}

  \end{frame}

  \begin{frame}
    \frametitle{Substepping}
    \framesubtitle{2D square track}
    \begin{adjustwidth}{-7mm}{-5mm}
      \begin{figure}
  \centering
  \begin{tikzpicture}[
      baseline,
      spy using outlines= {circle, connect spies},
    ]
    \begin{groupplot}[
      group style={
        group size=3 by 1,
        horizontal sep=0.4cm,
      },
    ]
    \nextgroupplot[
      truncatedddstyle,
      legend style={legend columns=2, at={(0.5,0.02)}, anchor=south},
      width=0.31\textwidth,
      legend cell align=left,
      title={Dirichlet-Neumann},
      ]

      \gammafs{}

      \addlegendentry{Iter. 1}
      \addplot+[mark=none, thin, unbounded coords=jump, opacity=0.8]
      table [col sep=comma, x=x, y=uh] {./plots/truncated_iters/CSVs/ssdnaitken_relax-midline-t1.csv};
      \addlegendentry{Iter. 2}
      \addplot+[mark=none, thin, unbounded coords=jump, opacity=0.8]
      table [col sep=comma, x=x, y=uh] {./plots/truncated_iters/CSVs/ssdnaitken_relax-midline-t2.csv};
      \addlegendentry{Iter. 4}
      \addplot+[mark=none, thin, unbounded coords=jump, opacity=0.8]
      table [col sep=comma, x=x, y=uh] {./plots/truncated_iters/CSVs/ssdnaitken_relax-midline-t4.csv};
      \addlegendentry{Iter. 8}
      \addplot+[mark=none, thin, unbounded coords=jump, opacity=0.8]
      table [col sep=comma, x=x, y=uh] {./plots/truncated_iters/CSVs/ssdnaitken_relax-midline-t8.csv};
    \nextgroupplot[
      truncatedddstyle,
      ylabel={},
      y unit={},
      yticklabel={$$},
      width=0.31\textwidth,
      legend pos=south east,
      legend cell align=left,
      title={Robin},
      ]

      \gammafs{}

      \addlegendentry{Iter. 1}
      \addplot+[mark=none, thin, unbounded coords=jump, opacity=0.8]
      table [col sep=comma, x=x, y=uh] {./plots/truncated_iters/CSVs/ssrobin-midline-t1.csv};
      \addlegendentry{Iter. 2}
      \addplot+[mark=none, thin, unbounded coords=jump, opacity=0.8]
      table [col sep=comma, x=x, y=uh] {./plots/truncated_iters/CSVs/ssrobin-midline-t2.csv};

      \coordinate (spypointrobin) at (axis cs:-0.15,  1418.0);
      \coordinate (magnifyglassrobin) at (axis cs:-0.5,  1900.0);
    \nextgroupplot[
      truncatedddstyle,
      legend style={legend columns=2},
      ylabel={},
      y unit={},
      yticklabel={$$},
      width=0.31\textwidth,
      legend pos=south east,
      legend cell align=left,
      title={Dirichlet},
      ]

      \gammafs{}

      \addlegendentry{Iter. 1}
      \addplot+[mark=none, thin, unbounded coords=jump, opacity=0.8]
      table [col sep=comma, x=x, y=uh] {./plots/truncated_iters/CSVs/sms-midline-t1.csv};
      \addlegendentry{Iter. 2}
      \addplot+[mark=none, thin, unbounded coords=jump, opacity=0.8]
      table [col sep=comma, x=x, y=uh] {./plots/truncated_iters/CSVs/sms-midline-t2.csv};
      \addlegendentry{Iter. 4}
      \addplot+[mark=none, thin, unbounded coords=jump, opacity=0.8]
      table [col sep=comma, x=x, y=uh] {./plots/truncated_iters/CSVs/sms-midline-t4.csv};

      \coordinate (spypointsms) at (axis cs:-0.15,  1430.0);
      \coordinate (magnifyglasssms) at (axis cs:-0.5,  1900.0);
  \end{groupplot}

  \spy [black, width=1.5cm, height=1.5cm, magnification=10] on (spypointrobin) in node[fill=white] at (magnifyglassrobin);
  \spy [black, width=1.5cm, height=1.5cm, magnification=10] on (spypointsms) in node[fill=white] at (magnifyglasssms);


\end{tikzpicture}
\caption{Solution profiles along the heating track after multiple staggered
iterations for a linear problem with $\Delta t_s = 8 T_{hs}$ and $\Delta t_f =
0.125 T_{hs}$. Substeppers from left to right: Dirichlet–Neumann with Aitken
relaxation, Robin, and Dirichlet.}
\label{fig:truncated_iters}
\end{figure}

    \end{adjustwidth}
  \end{frame}

  \begin{frame}
    \frametitle{Substepping}
    \framesubtitle{Experimental validation}
    Validation against experimental melt pool shape data of \cite{lane2020} (AMB2018-02),
    as done in \cite{kopp2022}.\\
    \begin{adjustwidth}{-5mm}{-5mm}
      \begin{figure}
  \begin{subfigure}[b]{0.44\linewidth}
  \centering
  \hcolorbar{rainbow}{0.4cm}{25}{2700}{$T$}{0.485985}\\
  \begin{tikzpicture}
    \node[anchor=south west, inner sep=0] (image) at (0, 0) {
      \includegraphics[width=\linewidth]{ambench_perspective_chimera_stagg.png}
    };
    \node[fill=white, fill opacity=0.7, text opacity=1, inner sep=1pt, rounded corners=2pt] at ($(image.west)!0.5!(image.east) + (2.4, -0.5)$) {$\Omega_m$};
    \node[fill=white, fill opacity=0.7, text opacity=1, inner sep=1pt, rounded corners=2pt] at ($(image.west)!0.9!(image.east) + (0.0, -1.05)$) {$\Omega_f$};
    \node[fill=white, fill opacity=0.7, text opacity=1, inner sep=1pt, rounded corners=2pt] at ($(image.south)!0.9!(image.north) + (2.0, 0.0)$) {$\Omega_s$};
  \end{tikzpicture}
  \caption{Intermediate substep of Robin substepper with an AS.}
  \label{fig:perspective_chimera}
\end{subfigure}%
\quad
\begin{subfigure}[b]{0.52\linewidth}
  \centering
\begin{tikzpicture}[
    spy using outlines= {rectangle, connect spies}
  ]
  \begin{groupplot}[
    group style={
      group size=1 by 2,
      vertical sep=0.6cm, % adjust as needed
    },
    trim axis left,
    trim axis right,
    width=8cm, % set width for both plots
    height=5cm,
    meltpoolstyle,
  ]
    \nextgroupplot[
        xlabel style={at={(axis description cs:1.0, -0.1)},anchor=west},
        x unit=\si{\micro\meter},
        ymin=0.0,
        ymax=+100.0,
        ylabel style={at={(axis description cs:1.05,0.3)},anchor=west},
        legend style={
          at={(0.5, +1.2)},
          anchor=south,
          legend columns=-1,
          cells={align=left},
          font=\footnotesize,
        },
        legend image post style={xscale=0.8},
        title={Top view},
        title style={at={(0.5,0.5)}, anchor=south},
    ]
    \addlegendentry{\citeauthor{kopp2022}}
    \addplot+[mark=none, opacity=0.6, thick]
      table [col sep=comma, x expr=\thisrow{x}*1e6, y expr=\thisrow{y}*1e6] {./plots/meltpool_contours/CSVs/kopp_ulp200_bird_view_sorted.csv} -- cycle;
    \addlegendentry{w/o AS}
    \addplot+[mark=none, opacity=0.6, thick]
      table [col sep=comma, x expr=\thisrow{x}*1e6, y expr=\thisrow{y}*1e6] {./plots/meltpool_contours/CSVs/hodge_8els_s025_bird_view_sorted.csv} -- cycle;
    \addlegendentry{w/ AS}
    \addplot+[mark=none, opacity=0.6, thick]
      table [col sep=comma, x expr=\thisrow{x}*1e3 - 1e3, y expr=\thisrow{y}*1e3] {./plots/meltpool_contours/CSVs/chimera_robin_8els_s025_bird_view_sorted.csv};
    \coordinate (spypoint_bird) at (axis cs:27.5, 0);
    \coordinate (magnifyglass_bird) at (axis cs:30,50);

    \nextgroupplot[
      xticklabels={},
      xlabel style={at={(axis description cs:0.5,1.05)},anchor=south},
      ymin=-50.0,
      ymax=+0.0,
      title={Side view},
      title style={at={(0.5,-0.2)}, anchor=north},
    ]
    \addplot+[mark=none, opacity=0.6, thick]
      table [col sep=comma, x expr=\thisrow{x}*1e4, y expr=\thisrow{y}*1e4] {./plots/meltpool_contours/CSVs/kopp_ulp200_side_view_sorted.csv};
    \addplot+[mark=none, opacity=0.6, thick]
      table [col sep=comma, x expr=\thisrow{x}*1e3 - 1e3, y expr=\thisrow{y}*1e3] {./plots/meltpool_contours/CSVs/robin_8els_s025_side_view_sorted.csv};
    \addplot+[mark=none, opacity=0.6, thick]
      table [col sep=comma, x expr=\thisrow{x}*1e3 - 1e3, y expr=\thisrow{y}*1e3] {./plots/meltpool_contours/CSVs/chimera_robin_8els_s025_side_view_sorted.csv};
    \coordinate (spypoint_side) at (axis cs:-310.0, 0);
    \coordinate (magnifyglass_side) at (axis cs:-304,-25);
  \end{groupplot}
  \spy [black, width=0.7cm, height=0.8cm, anchor=south west, magnification=3]
    on (spypoint_bird) in node[fill=white] at (magnifyglass_bird);
  \spy [black, width=0.9cm, height=0.6cm, anchor=north east, magnification=2.5]
    on (spypoint_side) in node[fill=white] at (magnifyglass_side);
\end{tikzpicture}
  \caption{Top and side views of simulated melt pool contours, with and without AS, compared against results from \cite{kopp2022}.}
  \label{fig:side_kopp_contours}
\end{subfigure}
\end{figure}


    \end{adjustwidth}
  \end{frame}

  \begin{frame}
    \frametitle{Substepping}
    \framesubtitle{3D LPBF cube}
    \begin{figure}
      \centering
      \begin{subfigure}[b]{0.33\textwidth}
        \centering
        \includegraphics[width=\textwidth]{{stacked_cubes/mesh.png}}
        \caption{Mesh and material composition at start of print.}
        \label{fig:3d_lpbf_mesh}
      \end{subfigure}%
      \hfill
      \begin{subfigure}[b]{0.33\textwidth}
        \centering
        \includegraphics[width=\textwidth]{{stacked_cubes/scan_path.png}}
        \caption{Printing path.}
        \label{fig:3d_lpbf_path}
      \end{subfigure}%
      \hfill
      \begin{subfigure}[b]{0.33\textwidth}
        \centering
        \includegraphics[width=\textwidth]{{stacked_cubes/bulk_mesh.png}}
        \caption{Bulk material at final time.}
        \label{fig:3d_lpbf_bulk_mesh}
      \end{subfigure}
      \caption{\centering \SI{1}{\milli\meter} cube. Powder elements
        corresponding to each layer are activated during recoating time-steps.
      \qquad \legendpowderbulk{}}
    \end{figure}
  \end{frame}

  \begin{frame}
    \frametitle{Substepping}
    \framesubtitle{3D LPBF cube}

    \vspace{-7mm}
    \definecolor{OIblue}{HTML}{0072B2}
\definecolor{OIverm}{HTML}{E69F00}
\definecolor{OIred}{HTML}{D55E00}
\definecolor{OIgreen}{HTML}{009E73}


\pgfplotsset{
  refstyle/.style={
    very thick, color=OIblue, solid, mark=none,
  },
  staggeredstyle/.style={
    thick, color=OIverm, densely dashed, mark=none,
  },
  dirichletstyle/.style={
    thick, color=OIred, dash pattern=on 7pt off 3pt,
    mark=none,
  },
  chimerastyle/.style={
    thick, color=OIgreen!85!black, densely dotted,
    mark=none,
  },
  layerzerostyle/.style={
    enlarge x limits=false,
    axis x discontinuity=crunch,
    axis x line*=bottom,
    axis y line*=left,
    xtick=\empty,
    clip=false,
    yticklabel style={font=\tiny, /pgf/number format/fixed},
    xticklabel style={font=\tiny},
    grid=major,
  },
  layerstyle/.style={
    layerzerostyle,
    axis y line=none,
    yticklabels=\empty,
    xticklabel style={font=\tiny},
  },
}

\newcommand{\addlayer}[6]{
  \ifnum#1=0
    \nextgroupplot[
      layerzerostyle,
      ylabel={\tiny #3},
      y unit={\tiny #4},
      ymin=#5, ymax=#6,
    ]
  \else
    \nextgroupplot[
      layerstyle,
      ymin=#5, ymax=#6,
    ]
  \fi
  \addplot[refstyle] table [x=time, y=#2, col sep=comma] {{plots/stacked_cubes_probe/ref_layers/layer_#1.csv}};
  \addplot[staggeredstyle] table [x=time, y=#2, col sep=comma] {{plots/stacked_cubes_probe/robin_layers/layer_#1.csv}};
  \addplot[dirichletstyle] table [x=time, y=#2, col sep=comma] {{plots/stacked_cubes_probe/hodge_layers/layer_#1.csv}};
  \addplot[chimerastyle] table [x=time, y=#2, col sep=comma] {{plots/stacked_cubes_probe/chimera_layers/layer_#1.csv}};
  \node at (rel axis cs:0.5,-0.05) {\tiny {\the\numexpr#1+1\relax}};
  \draw[thin, gray!50] (rel axis cs:0.0,0.0) -- (rel axis cs:0.0,0.05);

  \if#1=0
    \coordinate (spypoint) at (rel axis cs:4.5, 0.12);
    \coordinate (magnifyglass) at (rel axis cs:4.0, 0.8);
  \fi

}

\begin{figure}
  \centering
  \begin{tikzpicture}[
    baseline,
    ]
  \begin{axis}[
    hide axis,
    xmin=0, xmax=1, ymin=0, ymax=1, % keeps the box tiny
    height=2cm,
    legend columns=-1,
    legend cell align=left,
    legend style={
      draw=none,
      /tikz/every even column/.append style={column sep=8pt},
    },
  ]

  % One dummy point per entry:
  \addplot[refstyle]      coordinates {(0,0)};
  \addlegendentry{Unsubstepped}

  \addplot[staggeredstyle] coordinates {(0,0)};
  \addlegendentry{Robin}

  \addplot[dirichletstyle] coordinates {(0,0)};
  \addlegendentry{Dirichlet}

  \addplot[chimerastyle]  coordinates {(0,0)};
  \addlegendentry{Robin w/ AS}

  \end{axis}
  \end{tikzpicture}\\
  \begin{subfigure}[t]{0.45\textwidth}
  \centering
  \begin{tikzpicture}[
      trim left={0mm},
      baseline,
      spy using outlines= {circle, connect spies}
      ]
    \begin{groupplot}[
      group style={
        group size=20 by 1,
        horizontal sep=0pt, % panels touch
        ylabels at=edge left,
        xlabels at=edge bottom
      },
      width=1.85cm,
      height=7cm,
    ]

    \addlayer{0}{pt1}{T}{\unit{\celsius}}{25}{1825}
    \addlayer{1}{pt1}{T}{\unit{\celsius}}{25}{1825}
    \addlayer{2}{pt1}{T}{\unit{\celsius}}{25}{1825}
    \addlayer{3}{pt1}{T}{\unit{\celsius}}{25}{1825}
    \addlayer{4}{pt1}{T}{\unit{\celsius}}{25}{1825}
    \addlayer{5}{pt1}{T}{\unit{\celsius}}{25}{1825}
    \addlayer{6}{pt1}{T}{\unit{\celsius}}{25}{1825}
    \addlayer{7}{pt1}{T}{\unit{\celsius}}{25}{1825}
    \addlayer{8}{pt1}{T}{\unit{\celsius}}{25}{1825}
    \addlayer{9}{pt1}{T}{\unit{\celsius}}{25}{1825}
    \addlayer{10}{pt1}{T}{\unit{\celsius}}{25}{1825}
    \addlayer{11}{pt1}{T}{\unit{\celsius}}{25}{1825}
    \addlayer{12}{pt1}{T}{\unit{\celsius}}{25}{1825}
    \addlayer{13}{pt1}{T}{\unit{\celsius}}{25}{1825}
    \addlayer{14}{pt1}{T}{\unit{\celsius}}{25}{1825}
    \addlayer{15}{pt1}{T}{\unit{\celsius}}{25}{1825}
    \addlayer{16}{pt1}{T}{\unit{\celsius}}{25}{1825}
    \addlayer{17}{pt1}{T}{\unit{\celsius}}{25}{1825}
    \addlayer{18}{pt1}{T}{\unit{\celsius}}{25}{1825}
    \addlayer{19}{pt1}{T}{\unit{\celsius}}{25}{1825}

    \end{groupplot}
    \node at ($(group c1r1.south)!0.5!(group c20r1.south)-(0,5mm)$) {\tiny Layer};
    \spy [black, width=2cm, height=2cm, magnification=3] on (spypoint) in node[fill=white] at (magnifyglass);

  \end{tikzpicture}
  \end{subfigure}%
  \hfill
  \begin{subfigure}[t]{0.45\textwidth}
  \centering
  \begin{tikzpicture}[
      trim left={0mm},
      baseline
      ]
    \begin{groupplot}[
      group style={
        group size=20 by 1,
        horizontal sep=0pt, % panels touch
        ylabels at=edge left,
        xlabels at=edge bottom
      },
      width=1.85cm,
      height=7cm,
    ]

    \addlayer{0}{W}{Melt pool width}{\unit{\milli\meter}}{0.0}{0.35}
    \addlayer{1}{W}{Melt pool width}{\unit{\milli\meter}}{0.0}{0.35}
    \addlayer{2}{W}{Melt pool width}{\unit{\milli\meter}}{0.0}{0.35}
    \addlayer{3}{W}{Melt pool width}{\unit{\milli\meter}}{0.0}{0.35}
    \addlayer{4}{W}{Melt pool width}{\unit{\milli\meter}}{0.0}{0.35}
    \addlayer{5}{W}{Melt pool width}{\unit{\milli\meter}}{0.0}{0.35}
    \addlayer{6}{W}{Melt pool width}{\unit{\milli\meter}}{0.0}{0.35}
    \addlayer{7}{W}{Melt pool width}{\unit{\milli\meter}}{0.0}{0.35}
    \addlayer{8}{W}{Melt pool width}{\unit{\milli\meter}}{0.0}{0.35}
    \addlayer{9}{W}{Melt pool width}{\unit{\milli\meter}}{0.0}{0.35}
    \addlayer{10}{W}{Melt pool width}{\unit{\milli\meter}}{0.0}{0.35}
    \addlayer{11}{W}{Melt pool width}{\unit{\milli\meter}}{0.0}{0.35}
    \addlayer{12}{W}{Melt pool width}{\unit{\milli\meter}}{0.0}{0.35}
    \addlayer{13}{W}{Melt pool width}{\unit{\milli\meter}}{0.0}{0.35}
    \addlayer{14}{W}{Melt pool width}{\unit{\milli\meter}}{0.0}{0.35}
    \addlayer{15}{W}{Melt pool width}{\unit{\milli\meter}}{0.0}{0.35}
    \addlayer{16}{W}{Melt pool width}{\unit{\milli\meter}}{0.0}{0.35}
    \addlayer{17}{W}{Melt pool width}{\unit{\milli\meter}}{0.0}{0.35}
    \addlayer{18}{W}{Melt pool width}{\unit{\milli\meter}}{0.0}{0.35}
    \addlayer{19}{W}{Melt pool width}{\unit{\milli\meter}}{0.0}{0.35}

    \end{groupplot}
    \node at ($(group c1r1.south)!0.5!(group c20r1.south)-(0,5mm)$) {\tiny Layer};

  \end{tikzpicture}
  \end{subfigure}
  \caption{Temperature history at a point (left)
    and melt pool width (right)
    throughout the print of the 1.0 \unit{\milli\meter}
    cube using $\Delta t_s = 10 T_{hs}$.
    The probe is located at the center of the top surface of the substrate,
    that is, right below the first layer.
    Cooling intervals are excluded for clarity.
  }
  \label{fig:3d_lpbf_probe}
\end{figure}

  \end{frame}

  \begin{frame}
    \frametitle{Substepping}
    \framesubtitle{3D LPBF cube}
    \begin{itemize}
      \item Speed-up factors range from about $1.3$--$2.0$ for fast DOF ratios around
      $30\%$, up to roughly $4$--$6.5$ when the fast DOF ratio drops below
      $10\%$, depending on substepper and use of AS.
      \item Robin substepping slightly outperforms Dirichlet.
      \item AS approach yields increasing speed-ups with domain size; substantial gains for cubes $\geq 2\,\si{\milli\meter}$.
      \item For short tracks and large macro-steps, fast-domain DOF fraction can exceed profitability threshold and reduce speed-up.
      \item AS distorts thermal tail when advected subdomain overlaps previous tracks.
    \end{itemize}
  \end{frame}

  \begin{frame}
    \frametitle{Substepping + advected subdomain}
    \framesubtitle{Final conclusions}
    \begin{itemize}
      \item The AS method improved HAZ accuracy and allowed much larger
        time-steps (up to several \(T_{hs}\)), at the cost of losing SPD
        structure and increasing per-step cost by roughly a factor of 2–3.
      \item AS works best on long, simple tracks; direction
        changes and overlapping tracks shear thermal tails and can severely
        degrade accuracy, limiting applicability as a general speed-up
        technique.
      \item Substepping, via fast/slow domain partitioning with
        \(\Delta t_f\) and \(\Delta t_s \gg \Delta t_f\), provided a more
        robust and broadly applicable way to exploit time-scale disparity in
        LPBF.
      \end{itemize}
    \end{frame}
    \begin{frame}
      \frametitle{Substepping + advected subdomain}
      \framesubtitle{Final conclusions}
      \begin{itemize}
      \item Fast/slow DOF ratio was identified as the primary driver of
        speed-up, with \(\Delta t_s / \Delta t_f\) playing a secondary role;
        overly large macro-steps can enlarge the fast domain and erode gains.
      \item The proposed Robin–Robin scheme
        achieved accuracy comparable to
        \cite{hodge2021}. Both clearly outperformed Dirichlet–Neumann coupling.
      \item Extra staggered iterations in the substepping loop brought negligible accuracy benefits.
      \item The main accuracy limiter for large macro-steps was the slow-region
        time integrator (Backward Euler)
      % \item Diffusion-focused predictors for the substepper improved stability and
      %   accuracy relative to source-driven predictors, especially at large
      %   macro-steps.
      % \item
      %   An enthalpy-based latent-heat treatment proved substantially more
      %   robust than the apparent heat capacity method for powder-to-solid
      %   transitions.
      % \item Calibration against melt pool experiments required
      %   an unphysical absorptivity dip, indicating that important heat-loss
      %   physics—likely tied to melt pool flow and evaporation—is missing from
      %   the thermal model.
      % \item Overall, advected subdomains and substepping
      %   together form a flexible toolbox that makes high-fidelity, part-scale
      %   LPBF simulations more computationally feasible while retaining local
      %   accuracy near the heat source.
    \end{itemize}
  \end{frame}

  \begin{frame}
    \frametitle{Substepping + advected subdomain}
    \framesubtitle{Future works}
    \begin{itemize}
      \item Improve the advected subdomain by using spatially and temporally varying advection fields (e.g.\ ALE-based fading and ramped velocities) and adaptive deactivation when thermal gradients misalign with scan direction.
      \item Enhance substepping via better characterization of LPBF time-scales, higher-order time integration for the slow partition, improved predictors/interpolation, adaptive fast-domain definitions, and dynamic load balancing.
    \end{itemize}

  \end{frame}

  \begin{frame}
    \frametitle{Thank you!}
    \begin{figure}
      \externalvod{width=0.8\textwidth}{videos/thx.mp4}{thumbnail-thx.png}
      \caption{Robin substepper thanks you!}
    \end{figure}
  \end{frame}


\bibliographystyle{plainnat}
\bibliography{refs}

\end{document}
