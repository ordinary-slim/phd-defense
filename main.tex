%! TeX program = lualatex
\documentclass{beamer}
\usetheme{metropolis}

\setbeamerfont{page number in head/foot}{size=\tiny}
\setbeamercolor{footline}{fg=gray!70}

% Footline highlight
\colorlet{footlineHlColor}{black!60}
\newcommand{\footlineHl}[1]{
  \textbf{\textcolor{footlineHlColor}{#1}}
}

\newif\ifisdraft
\isdrafttrue
\newcommand{\todo}[1]{
  \ifisdraft {
    \Large
    $\color{black}\star$\color{red}~\textit{#1}\color{black}~$\star$
  } \fi
}

\definecolor{powderColor}{rgb}{0.878,0.859,0.811}
\definecolor{metalColor}{rgb}{0.420,0.408,0.384}
\newcommand{\legendpowderbulk}{%
  \begin{tabular}{rlrl}
     ({\color{powderColor} \rule[-1.5 pt]{8 pt}{8 pt}}) & Powder & ({\color{metalColor} \rule[-1.5 pt]{8 pt}{8 pt}})  & Bulk
  \end{tabular}
}

\defbeamertemplate*{title page}{customized}[1][]
{
  \usebeamerfont{title}\inserttitle\par
  \usebeamerfont{subtitle}\usebeamercolor[fg]{subtitle}\insertsubtitle\par
  \bigskip
  \usebeamerfont{author}\insertauthor\par
  \usebeamerfont{institute}\insertinstitute\par
  \usebeamerfont{date}\insertdate\par
  \usebeamercolor[fg]{titlegraphic}\inserttitlegraphic
}
\makeatletter
  \setbeamertemplate{frametitle}{%
    \begin{beamercolorbox}[%
      wd=\paperwidth,%
      sep=0pt,%
      leftskip=\metropolis@frametitle@padding,%
      rightskip=\metropolis@frametitle@padding,%
      ]{frametitle}%
      \metropolis@frametitlestrut@start%
      \insertframetitle%
      \ifx\insertframesubtitle\@empty%
      \else%
        \hfill%
        {\usebeamerfont{framesubtitle}\usebeamercolor[fg]{framesubtitle}\insertframesubtitle}%
      \fi%
      \nolinebreak%
      \metropolis@frametitlestrut@end%
    \end{beamercolorbox}%
  }
\makeatother

\usepackage{caption}
\captionsetup{font=scriptsize,labelfont=scriptsize}
\usepackage{subcaption}
\usepackage{siunitx}
\sisetup{
  range-phrase = --,
  range-units = single,
}
\usepackage{multirow}
\usepackage{booktabs}

\usepackage{cleveref}

\usepackage{natbib}

\usepackage{multimedia}


% CASTEL PACKAGES
\usepackage{import}
\usepackage{xifthen}
\usepackage{transparent}
% END CASTEL PACKAGES
\newcommand{\castelincfig}[2][1.0]{%
  \def\svgwidth{#1\columnwidth}
  \import{./figures/}{#2.pdf_tex}
}

% Math commands
\newcommand{\dependson}[1]{
  {\scriptstyle(#1)}
}

\usepackage{calc}
\usepackage{tikz}
\usetikzlibrary{calc,positioning}
\tikzset{
  annotate box/.style={
    fill=white,
    fill opacity=.4,
    text opacity=1,
    rounded corners=2pt,
    inner xsep=4pt,
    inner ysep=3pt,
    draw=black,
    line width=.4pt,
  },
}
\DeclareRobustCommand{\labelbox}[2][]{%
  % Inline TikZ node; safe in text, math, tabular, makebox, etc.
  \tikz[baseline=(X.base)]\node[annotate box,#1] (X) {#2};%
}
\newcommand{\playthumb}[2][]{%
  \begin{tikzpicture}
    % Thumbnail image
    \node[inner sep=0] (img) {\includegraphics[#1]{#2}};
    % Dark circle
    \draw[fill=black!60, draw=white, line width=0.8pt]
      (img.center) circle[radius=0.6cm];
    % Triangle
    \draw[fill=white, draw=none, rounded corners=1.5pt]
      ([xshift=0.34cm]img.center) --
      ([xshift=-0.19cm,yshift=0.3cm]img.center) --
      ([xshift=-0.19cm,yshift=-0.3cm]img.center) -- cycle;
  \end{tikzpicture}%
}

\title{Computational strategies for time-accurate simulation of part-scale LPBF}
\subtitle{Time scale disparity in moving heat source problems}
\author{Mehdi Slimani}
\date{January 23, 2026}

\graphicspath{{figures/}}


\begin{document}

  \maketitle

  \setbeamertemplate{frame footer}{Computational strategies for time-accurate simulation of part-scale \footlineHl{LPBF}}

  \begin{frame}
    \frametitle{{LPBF}}
    \framesubtitle{MAM technology}
    Also known as PBF-LB/M (ISO nomenclature), one of the
    main Metal Additive Manufacturing (MAM) technologies

    \begin{figure}
      \begin{subfigure}[t]{0.32\textwidth}
        \centering
        \includegraphics[width=\linewidth]{waam.jpg}
        \textbf{Wire Arc Additive Manufacturing
        \raisebox{5pt}{(}\includegraphics[height=16pt]{waam-comic-effect.png}\raisebox{5pt}{)}}
      \end{subfigure}%
      \hfill
      \begin{subfigure}[t]{0.32\textwidth}
        \centering
        \includegraphics[width=\linewidth]{ded.jpg}
        \textbf{Directed Energy Deposition (DED)}
      \end{subfigure}%
      \hfill
      \begin{subfigure}[t]{0.32\textwidth}
        \centering
        \includegraphics[width=\linewidth]{lpbf.png}
        \textbf{Laser Powder Bed Fusion (LPBF)}
      \end{subfigure}
    \end{figure}
  \end{frame}

  \begin{frame}
    \frametitle{LPBF}
    \framesubtitle{How does it work?}
    \todo{Add schematic of LPBF process here.}
  \end{frame}

  \begin{frame}
    \frametitle{{LPBF}}
    \framesubtitle{Fast, small, precise}

    \begin{table}
      \centering
      \begin{tabular}{l
        S
        S
        S }
        \toprule
        & {Radius $R$} 
        & {Speed $V$} 
        & {Power $P$} \\
        \midrule
        WAAM 
          & \qtyrange[]{2}{4}{\milli\meter}
          & \qtyrange[]{3}{10}{\milli\meter\per\second}
          & \qtyrange[]{5}{15}{\kilo\watt} \\
        DED 
          & \qtyrange[]{0.5}{1.5}{\milli\meter}
          & \qtyrange[]{5}{20}{\milli\meter\per\second}
          & \qtyrange[]{1}{4}{\kilo\watt} \\
        LPBF 
          & \qtyrange[]{25}{100}{\micro\meter}
          & \qtyrange[]{400}{1400}{\milli\meter\per\second}
          & \qtyrange[]{0.2}{1}{\kilo\watt} \\
        \bottomrule
      \end{tabular}
      \caption{Characteristic heat source parameters for the main MAM technologies.}
    \end{table}

    \vspace{-3mm}
    % LPBF offers a finer resolution and smoother surface finish,
    % but it is limited to smaller parts.
    \begin{figure}
      \centering
      \begin{subfigure}[t]{0.48\textwidth}
        \centering
        \includegraphics[width=0.9\linewidth]{waam-example.jpg}\\
        {\scriptsize WAAM: large parts (> \qty{1}{\meter}), coarse features.}
      \end{subfigure}%
      \hfill
      \begin{subfigure}[t]{0.48\textwidth}
        \centering
        \includegraphics[width=0.6\linewidth]{lpbf-fine.png}\\
        {\scriptsize LPBF: small part (< \qty{1}{\meter}), fine features.}
      \end{subfigure}
    \end{figure}

  \end{frame}

  \setbeamertemplate{frame footer}{Computational strategies for time-accurate \footlineHl{simulation} of \footlineHl{part-scale} LPBF}

  \begin{frame}
    \frametitle{LPBF}
    \framesubtitle{Extremely multiscale}
    LPBF is an \textit{extremely multiscale} application \citep{hodge2021}.\\
    Let's quantify this statement:
    \begin{itemize}
      \item The \textbf{smallest} spatial and temporal \textbf{scales} are governed by the
        \textbf{heat source}, characterized by its radius $\mathbf{R}$ and by
        the time it takes to travel one radius,
        \[
          \mathbf{T_{hs}} := \frac{R}{V}.
        \]
      \item The \textbf{largest} spatial and temporal \textbf{scales} are set
        by the \textbf{part} and the printer.
        We choose here the characteristic part length $\mathbf{L_{part}}$ and
        the net printing time $\mathbf{T_{print}}$, i.e. the cumulative
        laser-on time.
    \end{itemize}
  \end{frame}

  \begin{frame}
    \frametitle{LPBF}
    \framesubtitle{Extremely multiscale}
    \centering
    \begin{minipage}{0.59\textwidth}
      \begin{figure}[ht]
        \def\svgwidth{\columnwidth}
        \import{./figures/cube-example}{cube-scan-schematic.pdf_tex}
        \caption{Cube scan path of \textbf{side length} $L$, \textbf{layer thickness} $t$
        and \textbf{hatch spacing} $h$.}
      \end{figure}
    \end{minipage}%
    \hfill%
    \begin{minipage}{0.38\textwidth}
      Consider printing a cube of side length $L$.
      The ratio of volume scales is
      $$
      \frac{L^3}{R^3}
      $$
      Let's compute the time scale ratio.
      Assume $t~=~h~=~R$ for simplicity, so that
      $$N_{layers} = N_{hatches\\/layer} = \frac{L}{R}$$
    \end{minipage}
  \end{frame}

  \begin{frame}
    \frametitle{LPBF}
    \framesubtitle{Extremely multiscale}
    For this simple geometry, the net print time is
    \begin{gather*}
      T_{print} = N_{hatches} \cdot T_{hatch} = N_{layers} \cdot N_{hatches/layer} \cdot T_{hatch}\\
      T_{hatch} = \frac{L}{V}\\
      \implies
      T_{print} = \frac{L}{R} \cdot \frac{L}{R} \cdot \frac{L}{V} = \frac{L^2}{R^2} \cdot \frac{L}{V} = \frac{L^3}{R^2 V}
    \end{gather*}
    Therefore, the time scale disparity is
    $$
    \frac{T_{print}}{T_{hs}} = \frac{L^3}{R^2 V} \cdot \frac{V}{R} = \left(\frac{L}{R}\right)^3
    $$
  \end{frame}

  \begin{frame}
    \frametitle{Part-scale simulation}
    \framesubtitle{Why?}

    We've established that LPBF is extremely multiscale.
    In a few slides, we'll see why this is a problem for
    part-scale simulation.

    Why part-scale simulation?
    \begin{itemize}
      \item Avoid costly experimental trial-and-error
      \item Predict residual stresses and distortions
      \item Predict microstructure features
      \item Optimize process parameters
      \item Save time and costs in the design cycle
    \end{itemize}
  \end{frame}

  \begin{frame}
    \frametitle{Multiphysics \todo{Better title}}
    \begin{figure}[ht]
      \centering
      \includegraphics[height=0.8\textheight]{bayat2021.png}
      \caption{Relevant physics at melt-pool and part scales \citep{bayat2021}.}
      \label{fig:bayat2021}
    \end{figure}
  \end{frame}

  \begin{frame}
    \frametitle{Problem statement}
    \framesubtitle{Domain}
    \begin{figure}[ht]
      \def\svgwidth{\columnwidth}
      \hspace{-6mm}
      \import{./figures/lpbf_schematic}{schematic.pdf_tex}
      \caption{Schematic of the LPBF computational domain 
      $\Omega(t)$, encompassing the bulk (part and substrate) and powder bed regions,
      together with the applied heat source and convective/radiative heat losses.
    }
    \end{figure}
  \end{frame}

  \begin{frame}
    \frametitle{Problem statement}
    \framesubtitle{PDE system}
    {\small
    Define the extended temperature and liquid fraction fields
    \begin{equation*}
      T_e\dependson{\mathbf{x}, t} =
      \begin{cases}
        T\dependson{\mathbf{x}, t} & \mathbf{x} \in \overline{\Omega}(t)\\
        T_{dep} & \mathbf{x} \in \overline{\Omega}(t_{final}) \setminus \overline{\Omega}(t)
      \end{cases}
      \qquad
      f_{l,\; e}\dependson{\mathbf{x}, t} = f_l(T_e\dependson{\mathbf{x}, t})
    \end{equation*}
    where $T_{dep}$ is the deposition temperature.\\
    Find $T : \Omega(t) \times [0, T_{\text{final}}] \to \mathbb{R}$ such that
    \begin{align*}
      \rho c_p \partial_t T_e + \rho L \partial_t f_{l,\; e} - k \Delta T
      &= r\dependson{\mathbf{x}, t} &&\forall \mathbf{x} \in \Omega(t)\\
      - k \partial_n T &= h_{conv} (T - T_{\text{env}}) + \varepsilon \sigma (T^4 - T_{\text{env}}^4) &&\forall \mathbf{x} \in \partial \Omega(t)\\
      T\dependson{\mathbf{x}, 0} &= T_0 \qquad &&\forall \mathbf{x} \in \Omega(0)
    \end{align*}
    }
      \todo{Comment on phase change treatment here}
  \end{frame}

  \begin{frame}
    \frametitle{Discretization}
      \todo{Add discretization details here (FEM, BDF1)}
  \end{frame}

  \begin{frame}
    \frametitle{Discretization}
    \framesubtitle{Element activation}
    \begin{figure}
      \begin{subfigure}[t]{0.30\textwidth}
        \includegraphics[width=\textwidth]{schematic_melting/0.png}
        \caption{Bare substrate below an inactive powder layer.}
        \label{fig:refModelBareSubstrate}
      \end{subfigure}\hfill%
      \begin{subfigure}[t]{0.30\textwidth}
        \includegraphics[width=\textwidth]{schematic_melting/1.png}
        \caption{A powder layer is activated during a recoating step.}
        \label{fig:powderLayer}
      \end{subfigure}\hfill%
      \begin{subfigure}[t]{0.30\textwidth}
        \includegraphics[width=\textwidth]{schematic_melting/2.png}
        \caption{After a heating step, elements whose average temperature
        surpasses $T_m$ are set to bulk.}
        \label{fig:activPhaseChange3}
      \end{subfigure}\hfill%
      \caption{Illustration of deposition and melting processes.\qquad
        \legendpowderbulk{}
      }
      \label{fig:activPhaseChange}.
    \end{figure}
  \end{frame}

  \begin{frame}
    \frametitle{Discretization}
    \begin{figure}
      \movie[externalviewer]{\playthumb[width=0.8\textwidth]{thumbnail-2dlpbf_2d_lpbf_ref.png}}{../videos/2dlpbf_2d_lpbf_ref.mp4}
      \caption{Demo simulation of 2D LPBF with element activation. Wireframe elements correspond to powder region.}
    \end{figure}
    \todo{Merge with previous slide?}
  \end{frame}

  \begin{frame}
    \frametitle{Part-scale simulation}
    \framesubtitle{Impossible}
    Previous slide: ``uniform'' mesh in part and powder region.
    Recall the volume scale ratio;
    In 3D, we would need
    \begin{equation}
      \label{eq:num_elements_uniform_mesh}
      \text{\# elements} = \mathcal{O}\left(\frac{L^3}{R^3}\right)
    \end{equation}
    to resolve the heat source throughout the print.

    In practice, no one uses uniform meshes for LPBF simulation;
    \textbf{AMR} is regarded as the \textbf{de facto standard}.
  \end{frame}

  \begin{frame}
    \frametitle{Part-scale simulation}
    \framesubtitle{Impossible}
    But what about time-steps?
    The interval of interest is $]0, T_{final}[$ with
    $$
    T_{final} = T_{print} + T_{cool}
    $$
    i.e. the net printing time plus cooling.
    The time scale disparity requires again \cref{eq:num_elements_uniform_mesh} time-steps
    \begin{gather*}
      \text{\# time-steps} > \frac{T_{print}}{T_{hs}} = \mathcal{O}\left(\frac{L^3}{R^3}\right)\\
    \end{gather*}
    when discretizing with \textbf{uniform time-steps}.

    Surely people aren't employing uniform time-steps for LPBF simulation?
  \end{frame}

  \begin{frame}
    \frametitle{Part-scale simulation}
    \framesubtitle{Impossible}
    \textbf{Uniform time-stepping} is the \textbf{de facto standard} in LPBF simulation.
    \todo{Actual examples}
    \todo{Comment on ressources needed and examples from literature}
    Decimeter-scale parts are currently unfeasible.
  \end{frame}

  \setbeamertemplate{frame footer}{Computational strategies for \footlineHl{time-accurate} simulation part-scale LPBF}

  \begin{frame}
    \frametitle{Part-scale simulation}
    \framesubtitle{Lumped heat source}
    What do you do if you don't have a cluster or a nice GPU ?
    You simplify the model.
    \todo{Explain it as thermal history simplification}
  \end{frame}

  \begin{frame}
    \frametitle{Part-scale simulation}
    \framesubtitle{Lumped heat source}
    \todo{Comment on Van Elsen}
  \end{frame}

  \setbeamertemplate{frame footer}{\footlineHl{Computational strategies for time-accurate simulation part-scale LPBF}}

  \begin{frame}
    \frametitle{Objectives of this work}
    \begin{itemize}
      % \item Investigate practical MAM cases, assess the
      %   feasibility of high-fidelity simulations and quantify
      %   time-stepping bottlenecks in simulations of DED and LPBF at the part scale.
      \item
        Review existing methods for addressing time-stepping
        challenges in MAM.
      \item
        Explore novel methods for reducing the number of time-steps
        required in LPBF modeling.
      \item
        Validate the proposed methods on realistic
        test cases.
      \item
        Validate the underlying physical models
        and numerical methods against experimental data.
      \item
        Provide realistic speedup estimates and practical implementation guidelines
        for the proposed methods.
      \item Ensure open science and reproducibility.
    \end{itemize}
  \end{frame}

  \begin{frame}
    \frametitle{Advected subdomain}
    Key idea:
     We need small time-steps to resolve the motion of the heat source ->
     get rid of it!
  \end{frame}

  \begin{frame}
    \frametitle{Substepping}
    Key idea:
    We don't need small time-steps everywhere
  \end{frame}

  \begin{frame}
    \frametitle{Parallel scalings}
    (Support slide)
  \end{frame}

  \begin{frame}
    \frametitle{Is Celentano's method source-based or enthalpy-based}
    (Support slide)
      \todo{Add discussion on Celentano's method here}
  \end{frame}

\bibliographystyle{plainnat}
\bibliography{refs}

\end{document}
