\begin{frame}
  \frametitle{Substepping}
  \begin{figure}
    \centering
    \def\svgwidth{1.0\columnwidth} {\small \import{./figures/substepping/}{schematic.pdf_tex}}
    \caption{Non-overlapping substepping decomposition. The HAZ is included in the \textbf{fast partition} $\Omega_f$.
    The remainder of the domain belongs to the \textbf{slow partition} $\Omega_s$.}
    \label{fig:substepping}
  \end{figure}
\end{frame}

\begin{frame}
  \frametitle{Substepping}
    \begin{figure}
      \centering
      \video<1>[above,height=0.7,aspectratio=62/27,fit=fill,controls]
    at (0.5,0.18) {videos/basic-demo-ss.mp4}
      \vspace{0.7\pageheight+3pt}
      \videoCaption{
        Basic demo of substepping method.
      }
    \end{figure}
\end{frame}

\begin{frame}
  \frametitle{Substepping}
  \cite{slimani2025}:
  \begin{itemize}
    \item Examines SOA of substepping methods for LPBF;
      identifies common structures and shortcomings.
    \item New Robin-Robin substepper for LPBF is compared against
      Dirichlet substepper of \cite{hodge2021}.
    \item Realistic speedup estimates provided.
    \item Alternate predictor schemes are proposed and tested.
    \item Robin-Robin version of AS method is proposed.
    \item AS is nested within the fast partition of said substeppers.
  \end{itemize}
\end{frame}

\begin{frame}
  \frametitle{Substepping + advected subdomain}
    \begin{figure}
      \centering
      \video<1>[above,height=0.7,aspectratio=62/27,fit=fill,controls]
    at (0.5,0.18) {videos/basic-demo-css.mp4}
      \vspace{0.7\pageheight+3pt}
      \videoCaption{
        Basic demo substepping + advected subdomain
      }
    \end{figure}
\end{frame}
