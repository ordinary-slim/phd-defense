\begin{frame}
  \frametitle{Advected subdomain}
  \begin{figure}
    \centering
    \def\svgwidth{1.05\columnwidth}
    \small
    \import{./figures/chimera_schematic/}{chimera_schematic.pdf_tex}
    \caption{Schematic of Robin-Robin variant of advected subdomain method.}
    \label{fig:chimera_schematic}
  \end{figure}
\end{frame}

\begin{frame}
  \frametitle{Advected subdomain}

    {\small
    \begin{gather*}
      (\mathbf{x}, t) \longrightarrow (\boldsymbol{\xi}, \eta) \thinspace , \thinspace
      \begin{cases}
        \boldsymbol{\xi} = \mathbf{x} - \int_0^t \mathbf{v}\dependson{t} dt\\
        \eta = t
      \end{cases}
    \end{gather*}
    \begin{gather*}
      \Longrightarrow
      \begin{cases}
        \partial_{x_i} &= \partial_{\xi_i} \Rightarrow \nabla_{\mathbf{x}} = \nablaxi\\
        \partial_t &= \partial_\eta - v_i \partial_{\xi_i} = \partial_\eta - \mathbf{v}\dependson{t} \cdot \nablaxi
      \end{cases}
    \end{gather*}

    Find $T_m : \Omega_m(t) \times [0, T_{\text{final}}] \to \mathbb{R}$
    and $T_f : \Omega_f(t) \times [0, T_{\text{final}}] \to \mathbb{R}$
    such that
    \begin{align*}
      \rho c_p \Big( \partial_t T^m{\scriptstyle (\xi, t)} - \mathbf{v}\dependson{t} \cdot \nabla T^m{\scriptstyle (\xi, t)} \Big) -
      \nabla \cdot ( k \nabla T^m{\scriptstyle (\xi, t)}) &= r{\scriptstyle (\xi)}  &\xi &\in \Omega_m(t)
    \end{align*}
    \begin{align*}
    \rho c_p \partial_t T^f{\scriptstyle (x, t)} - \nabla \cdot (k \nabla T^f{\scriptstyle (x, t)}) &= 0 \qquad &x &\in \Omega_f(t)
    \end{align*}
    }
\end{frame}

\begin{frame}
  \frametitle{Advected subdomain}
  \cite{slimani2024}: Dirichlet-Neumann multi-mesh variant of advected
  subdomain method for linear heat equation.
  \begin{figure}[h]
    \centering
    \includegraphics[width=0.6\textwidth]{quadTriang.png}
    \caption{A finer mesh can be attached to the heat source
    to adress spatial scale disparity: not done here.}
  \end{figure}
  \begin{itemize}
    \item VMS stabilization to handle advection in $\Omega_m$
    \item Monolithic solution of the coupled problem
    \item Serial C++ implementation available on GitHub.
  \end{itemize}
\end{frame}

\begin{frame}
  \frametitle{Advected subdomain}
  \framesubtitle{Growing $\Delta t$ workflow}
  We shifted the motion of the heat source into $\Omega_m$;
  the time-scale of the motion is no longer $T_{hs}$,
  it is $T_{\Omega_m}$:
  $$
    T_{\Omega_m} := \frac{L_{\Omega_m}}{V} > T_{hs} \quad \text{ given } \quad L_{\Omega_m} > R
  $$
  \begin{itemize}
    \item $L_{\Omega_m}$ can be chosen and changed at each time-step.
    \item The PDE in $\Omega_m$ can admit a steady solution,
          provided that the geometry and material composition
          remain constant over a time interval.
  \end{itemize}
  $\longrightarrow$ \textbf{Growing} $\Delta t$ \textbf{workflow}\;:
  at the end of a time-step:

  \begin{center}
  \begin{tikzpicture}[node distance=1em, every node/.style={align=center}]
    \tikzset{
      blockdiagr/.style={
        draw,
        rounded corners,
        align=center,
        inner sep=4pt
      }
    }
    \node[blockdiagr] (metric) {$\texttt{steadiness\_metric}(T_m) < \epsilon$};
    \node[blockdiagr, below=of metric] (dt) {Increase $\Delta t$ for $t^{n+2} = t^{n+1} + \Delta t$};
    \node[blockdiagr, below=of dt] (omega) {Resize $\Omega_m$};
    \draw[->] (metric) -- (dt);
    \draw[->] (dt) -- (omega);
  \end{tikzpicture}
  \end{center}
  % where \texttt{steadiness\_metric} can be chosen as e.g.
  % $$
  % \max_{\Omega_m} \left| \frac{T_m^{n+1} - T_m^{n}}{\Delta t} \right|
  % \quad
  % \text{or}
  % \quad
  % \frac{\|T_m\big|_{t^{n+1}} - T_m\big|_{t^{n}}\|}{\|T_m\big|_{t^{n+1}}\|} 
  % $$
\end{frame}

\begin{frame}
  \frametitle{Advected subdomain}
  \framesubtitle{Workflow}
  \begin{columns}
    \begin{column}{0.4\textwidth}
      \begin{itemize}
      \item Dynamic resizing of $\Omega_m$\\
        $\longrightarrow$ background mesh
      \item $\Omega_m(t^n),\Omega_m(t^{n+1})\subset\Omega$\\
        $\longrightarrow$ \texttt{intersect} at $t^n$ and $t^{n+1}$
      \item Sizing of $\Omega_m$:
        \begin{itemize}
          \item Front and side lengths larger than $2 R$
            to contain support of heat source.
          \item Back length larger than $V \cdot \Delta t$
            to not skip regions of domain.
        \end{itemize}
      \end{itemize}
    \end{column}%
    \begin{column}{0.6\textwidth}
      \begin{figure}
        \begin{subfigure}[t]{0.49\columnwidth}
          \includegraphics[height=0.34\pageheight]{2d_meshing_example/afterShaping.png}
          \caption{$\mathcal{T}^m \leftarrow \texttt{shapeSubdomain}(\mathcal{T}^m_{bg})$}
        \end{subfigure}%
        \hfill%
        \begin{subfigure}[t]{0.49\columnwidth}
          \includegraphics[height=0.34\pageheight]{2d_meshing_example/after_intersec1.png}
          \caption{$\mathcal{T}^m \leftarrow \texttt{intersect}\big(\mathcal{T}^m, \mathcal{T}^f \textrm{ at } t^n\big)$}
        \end{subfigure}\\
        \begin{subfigure}[t]{0.49\columnwidth}
          \includegraphics[height=0.34\pageheight]{2d_meshing_example/after_intersec2.png}
          \caption{$\mathcal{T}^m \leftarrow \texttt{intersect}\big(\mathcal{T}^m, \mathcal{T}^f\big) \textrm{ at } t^{n+1}$}
        \end{subfigure}%
        \hfill%
        \begin{subfigure}[t]{0.49\columnwidth}
          \includegraphics[height=0.34\pageheight]{2d_meshing_example/dd.png}
          \caption{$\mathcal{T}^f \leftarrow \texttt{subtract}\big(\mathcal{T}^f, \mathcal{T}^m\big)$}
        \end{subfigure}
        \label{fig:2d_meshing_example}
      \end{figure}
    \end{column}%
  \end{columns}
\end{frame}

