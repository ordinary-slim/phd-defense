\begin{frame}
  \frametitle{LPBF}
  \framesubtitle{Extremely multiscale}
  LPBF is an \textit{extremely multiscale} application \citep{hodge2021}.\\
  Let's quantify this statement:
  \begin{itemize}
    \item The \textbf{smallest} spatial and temporal \textbf{scales} are governed by the
      \textbf{heat source}, characterized by its radius $\mathbf{R}$ and by
      the time it takes to travel one radius,
      \[
        \mathbf{T_{hs}} := \frac{R}{V}.
      \]
    \item The \textbf{largest} spatial and temporal \textbf{scales} are set
      by the \textbf{part} and the printer.
      We choose here the characteristic part length $\mathbf{L_{part}}$ and
      the net printing time $\mathbf{T_{print}}$, i.e. the cumulative
      laser-on time.
  \end{itemize}
\end{frame}

\begin{frame}
  \frametitle{LPBF}
  \framesubtitle{Extremely multiscale}
  \centering
  \begin{minipage}{0.59\textwidth}
    \begin{figure}[ht]
      \def\svgwidth{0.84\columnwidth}
      \import{./figures/cube-example}{cube-scan-schematic.pdf_tex}
      \caption{Cube scan path of \textbf{side length} $L$, \textbf{layer thickness} $t$
      and \textbf{hatch spacing} $h$.}
    \end{figure}
  \end{minipage}%
  \hfill%
  \begin{minipage}{0.38\textwidth}
    \textbf{Space}:
    \begin{gather*}
      \text{Length scale ratio} = \frac{L}{R}\\
      \Large\Downarrow\\
      \alert<2>{\text{Volume scale ratio} = \frac{L^3}{R^3}}
    \end{gather*}
    \textbf{Time}:
    Assuming $h$ and $t$ are equal to $R$ for simplicity,
    \begin{gather*}
      T_{print} = \frac{L^3}{R^2 V}, \quad T_{hs} = \frac{R}{V}\\
      \Large\Downarrow\\
      \alert<2>{\text{Time scale ratio} = \frac{T_{print}}{T_{hs}} = \frac{L^3}{R^3}}
    \end{gather*}
  \end{minipage}
\end{frame}

% \begin{frame}
%   \frametitle{LPBF}
%   \framesubtitle{Extremely multiscale}
%   For this simple geometry, the net print time is
%   \begin{gather*}
%     T_{print} = N_{hatches} \cdot T_{hatch} = N_{layers} \cdot N_{hatches/layer} \cdot T_{hatch}\\
%     T_{hatch} = \frac{L}{V}\\
%     \implies
%     T_{print} = \frac{L}{R} \cdot \frac{L}{R} \cdot \frac{L}{V} = \frac{L^2}{R^2} \cdot \frac{L}{V} = \frac{L^3}{R^2 V}
%   \end{gather*}
%   Therefore, the time scale disparity is
%   $$
%   \frac{T_{print}}{T_{hs}} = \frac{L^3}{R^2 V} \cdot \frac{V}{R} = \left(\frac{L}{R}\right)^3
%   $$
% \end{frame}

\begin{frame}
  \frametitle{LPBF}
  \framesubtitle{Multiphysics}
  \begin{figure}
    \centering
    \includegraphics[height=0.8\textheight]{bayat2021.png}
    \caption{Relevant physics at melt-pool and part scales \citep{bayat2021}.}
    \label{fig:bayat2021}
  \end{figure}
\end{frame}
