\begin{frame}
  \frametitle{Part-scale simulation}
  \framesubtitle{Why?}

  LPBF is both \textbf{extremely multiscale} and a \textbf{multiphysics} application.
  Despite this complexity, predictions are required at the \textbf{part scale}.

  \vspace{0.5em}
  Part-scale simulation aims to:
  \begin{itemize}
    \item Reduce costly experimental trial-and-error
    \item Predict residual stresses and part distortions
    \item Link thermal history to microstructure features
    \item Enable process-parameter optimization
    \item Shorten the design-to-manufacturing cycle
  \end{itemize}
\end{frame}

\begin{frame}
  \frametitle{Part-scale simulation}
  \framesubtitle{Problem statement}
  \begin{figure}[ht]
    \def\svgwidth{0.9\columnwidth}
    \hspace{-6mm}
    \import{./figures/lpbf_schematic}{schematic.pdf_tex}
    \caption{Schematic of the LPBF computational domain 
    $\Omega(t)$, encompassing the bulk (part and substrate) and powder bed regions,
    together with the applied heat source and convective/radiative heat losses.
  }
  \end{figure}
\end{frame}

\begin{frame}
  \frametitle{Part-scale simulation}
  \framesubtitle{PDE}
  {\small
  Define the extended temperature and liquid fraction fields
  \begin{equation*}
    T_e\dependson{\mathbf{x}, t} =
    \begin{cases}
      T\dependson{\mathbf{x}, t} & \mathbf{x} \in \overline{\Omega}(t)\\
      T_{dep} & \mathbf{x} \in \overline{\Omega}(t_{final}) \setminus \overline{\Omega}(t)
    \end{cases}
    \qquad
    f_{l,\; e}\dependson{\mathbf{x}, t} = f_l(T_e\dependson{\mathbf{x}, t})
  \end{equation*}
  where $T_{dep}$ is the deposition temperature.\\
  Find $T : \Omega(t) \times [0, T_{\text{final}}] \to \mathbb{R}$ such that
  \begin{align}
    \label{eq:original_pde}
    \rho c_p \partial_t T_e + \rho L \partial_t f_{l,\; e} - k \Delta T
    &= r\dependson{\mathbf{x}, t} &&\forall \mathbf{x} \in \Omega(t)\\
    \notag
    - k \partial_n T &= h_{conv} (T - T_{\text{env}}) + \varepsilon \sigma (T^4 - T_{\text{env}}^4) &&\forall \mathbf{x} \in \partial \Omega(t)\\
    \notag
    T\dependson{\mathbf{x}, 0} &= T_0 \qquad &&\forall \mathbf{x} \in \Omega(0)
  \end{align}
  }
    \todo{Comment on phase change treatment here}
\end{frame}

\begin{frame}
  \frametitle{Part-scale simulation}
  \framesubtitle{Discretization}
    Source-based latent heat treatment of \cite{celentano1994}:
    $f_l(T)$ is directly discretized in time.
    \begin{figure}
      \centering
      \begin{subfigure}[t]{0.33\textwidth}
        \begin{tikzpicture}
        \begin{axis}[
            xmin=-0.1, xmax=1.1,
            ymin=+0, ymax=1.1,
            axis lines=center,
            axis on top=true,
            ytick={0, 1},
            xtick={0.05, 0.95},
            xticklabels={$T_s$, $T_l$},
            domain=-0.1:1.1,
            ylabel=$f_l$,
            xlabel=$T$,
            %legend style={at={(0.03,0.5)},anchor=west},
            height=0.4\pageheight
            ]

            \def\Ts{0.05}
            \def\Tl{0.95}
            \newcommand{\flalpha}[1]{2*#1/(\Tl - \Ts)}
            \newcommand{\fl}[1]{
              0.5*(tanh(\flalpha{#1}*(\x - 0.5))+1)
            }
            \addplot+ [mark=none, ultra thick, smooth] {\fl{1}};
            \addplot+ [mark=none, ultra thick, smooth] {\fl{2}};
            \addplot+ [mark=none, ultra thick, smooth] {\fl{4}};
            
            %% Add the asymptotes
            \draw [blue, dotted, thick] (axis cs:+1.1,+1)-- (axis cs:0,+1);
        \end{axis}
        \end{tikzpicture}
      \end{subfigure}%
      \hfill
      \begin{subfigure}[t]{0.33\textwidth}
        \begin{tikzpicture}
        \begin{axis}[
            xmin=-0.1, xmax=1.1,
            ymin=+0, ymax=4.5,
            axis lines=center,
            axis on top=true,
            ytick={0},
            xtick={0.05, 0.95},
            xticklabels={$T_s$, $T_l$},
            domain=-0.1:1.1,
            ylabel=$f_l'$,
            legend style={at={(0.7,0.8)},anchor=west},
            height=0.4\pageheight
            ]

            \def\Ts{0.05}
            \def\Tl{0.95}
            \newcommand{\flalpha}[1]{2*#1/(\Tl - \Ts)}
            \newcommand{\flp}[1]{
              \flalpha{#1}/2*(1 - (tanh(\flalpha{#1}*(\x - 0.5)))^2)
            }
            \addplot+ [mark=none, ultra thick, smooth] {\flp{1}};
            \addlegendentry{$S = 1$}                
            \addplot+ [mark=none, ultra thick, smooth] {\flp{2}};
            \addlegendentry{$S = 2$}                
            \addplot+ [mark=none, ultra thick, smooth] {\flp{4}};
            \addlegendentry{$S = 4$}
            
        \end{axis}
        \end{tikzpicture}
      \end{subfigure}%
      \hfill
      \begin{subfigure}[t]{0.33\textwidth}
        \begin{tikzpicture}
        \begin{axis}[
            xmin=-0.1, xmax=1.1,
            ymin=-4, ymax=4,
            axis lines=center,
            axis on top=true,
            ytick={0},
            xtick={0.05, 0.95},
            xticklabels={$T_s$, $T_l$},
            domain=-0.1:1.1,
            ylabel=$f_l''$,
            height=0.4\pageheight
            ]

            \def\Ts{0.05}
            \def\Tl{0.95}
            \newcommand{\flalpha}[1]{2*#1/(\Tl - \Ts)}
            \newcommand{\flpp}[1]{
              \flalpha{#1}^2*tanh(\flalpha{#1}*(\x - 0.5))*((tanh(\flalpha{#1}*(\x - 0.5)))^2 - 1)
            }
            \addplot+ [mark=none, ultra thick, smooth] {\flpp{1}};
            \addplot+ [mark=none, ultra thick, smooth] {\flpp{2}};
            \addplot+ [mark=none, ultra thick, smooth] {\flpp{4}};
            
        \end{axis}
        \end{tikzpicture}
      \end{subfigure}
      \caption{Liquid fraction $f_l$ and derivatives.}
      \label{fig:fl_fld_fldd}
    \end{figure}
    Multiply \cref{eq:original_pde}
    by $\phi \in V_T(t) = H^{1}\left(\Omega(t)\right)$;
    integrate over $\Omega\dependson{t}$;
    apply integration by parts on the diffusion term; insert BCs;
    apply BDF1
    {\small
    \begin{gather*}
      \label{eq:weak_heat}
      \int_{\Omega} \phi \rho \left({c_p \frac{T^{n+1} - T^n}{\Delta t} + L \frac{f_l(T^{n+1}) - f_l(T^n)}{\Delta t}}\right)
      + \int_{\Omega} \nabla \phi \cdot \left(k \nabla T\right)\\
      \notag
      \forall \phi \in V_T(t) \hspace{1cm}
      \;=\; \int_{\Omega} \phi r
      + \int_{\partial \Omega} \phi \left({h_{\text{conv}} \left( T - T_{\text{env}} \right)
      + \varepsilon \sigma \left( {T}^4 - T_{\text{env}}^4 \right)}\right)
    \end{gather*}
    }

\end{frame}

\begin{frame}
  \frametitle{Part-scale simulation}
  \framesubtitle{Discretization}
  \begin{figure}
    \begin{subfigure}[t]{0.33\textwidth}
      \includegraphics[height=0.35\pageheight]{schematic_melting/0.png}
      \caption{Bare substrate below an inactive powder layer.}
      \label{fig:refModelBareSubstrate}
    \end{subfigure}\hfill%
    \begin{subfigure}[t]{0.33\textwidth}
      \includegraphics[height=0.35\pageheight]{schematic_melting/1.png}
      \caption{A powder layer is activated during a recoating step.}
      \label{fig:powderLayer}
    \end{subfigure}\hfill%
    \begin{subfigure}[t]{0.33\textwidth}
      \includegraphics[height=0.35\pageheight]{schematic_melting/2.png}
      \caption{After a heating step, elements whose average temperature
      surpasses $T_m$ are set to bulk.}
      \label{fig:activPhaseChange3}
    \end{subfigure}\hfill%
    \caption{Illustration of deposition and melting processes.\qquad
      \legendpowderbulk{}
    }
    \label{fig:activPhaseChange}.
  \end{figure}
  Same treatment of phase change as in \citep{kollmannsberger2018}
\end{frame}

\begin{frame}
  \frametitle{Part-scale simulation}
  \framesubtitle{Demo video}

  \begin{figure}
    \video<1>[above,height=0.7,aspectratio=62/27,fit=fill,controls]
  at (0.5,0.18) {videos/2dlpbf_2d_lpbf_ref.mp4}
    \vspace{0.7\pageheight+3pt}
    \videoCaption{
      Demo simulation of 2D LPBF with element activation.\\
      Wireframe elements (\;\wireframeTriangle{}) correspond to powder region.
    }
  \end{figure}

\end{frame}
