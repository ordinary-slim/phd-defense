\begin{frame}
  \frametitle{Part-scale simulation}
  \framesubtitle{Impossible}
  Previous slide: ``uniform'' mesh in part and powder region.
  Recall the volume scale ratio;
  In 3D, we would need
  \begin{equation}
    \label{eq:num_elements_uniform_mesh}
    \text{\# elements} = \mathcal{O}\left(\frac{L^3}{R^3}\right)
  \end{equation}
  to resolve the heat source throughout the print.

  In practice, no one uses uniform meshes for LPBF simulation;
  \textbf{AMR} is regarded as the \textbf{de facto standard}.
\end{frame}

\begin{frame}
  \frametitle{Part-scale simulation}
  \framesubtitle{Impossible}
  But what about time-steps?
  The interval of interest is $]0, T_{final}[$ with
  $$
  T_{final} = T_{print} + T_{cool}
  $$
  i.e. the net printing time plus cooling.
  The time scale disparity requires again \cref{eq:num_elements_uniform_mesh} time-steps
  \begin{gather*}
    \label{eq:num_timesteps_uniform_mesh}
    \text{\# time-steps} > \frac{T_{print}}{T_{hs}} = \mathcal{O}\left(\frac{L^3}{R^3}\right)
  \end{gather*}
  when discretizing with \textbf{uniform time-steps}.
\end{frame}

\begin{frame}
  \frametitle{Part-scale simulation}
  \framesubtitle{Impossible}
  Intuitively, if we don't respect the constraint
  \begin{equation}
    \label{eq:origconstraint}
    \Delta t \leq T_{hs},
  \end{equation}
  we won't resolve the motion of the heat source.\\
  In practice, that's indeed what happens:
  if the time-step is larger than $T_{hs}$ i.e. the heat source
  travels more than $1 R$ per time-step,
  it skips over parts of the domain,
  and generates artificial temperature spikes.
  \begin{figure}
    \begin{subfigure}[t]{0.49\textwidth}
      \includegraphics[width=\textwidth]{timestep-2ths/1R.png}
      \caption{$\Delta t = 1 T_{hs}$}
      \label{fig:spots1R}
    \end{subfigure}
    \begin{subfigure}[t]{0.49\textwidth}
      \includegraphics[width=\textwidth]{timestep-2ths/2R.png}
      \caption{$\Delta t = 2 T_{hs}$}
      \label{fig:spots2R}
    \end{subfigure}
    % \caption{2D heating track example with admissible and inadmissible
    % time-step sizes according to inequality \eqref{eq:origconstraint}.}
    % \label{fig:spots}
  \end{figure}
\end{frame}

\begin{frame}
  \frametitle{Part-scale simulation}
  \framesubtitle{Impossible}
  So \eqref{eq:num_timesteps_uniform_mesh} is indeed a \textbf{lower bound} on the number of time-steps
  when using uniform time-stepping.

  \underline{Bad news:}
  \begin{itemize}
    \item \textbf{Uniform time-stepping} is the \textbf{de facto standard} in LPBF simulation.
    \item Some references require time-steps much smaller than $T_{hs}$
      to ensure stability and accuracy \citep{hodge2014, hodge2021, elahi2025}.
  \end{itemize}
  There are basically 4 (!) groups in the world that can run simulations in the order of \textbf{centimeters}:
  Pittsburgh, Northwestern, LLNL, TUM.\\
  \textbf{Decimeter}-scale parts are currently unfeasible.
  \todo{Expand}
\end{frame}

\begin{frame}
  \frametitle{Part-scale simulation}
  \framesubtitle{Lumped heat source}
  
  What do you do if you don't have a cluster or a nice GPU ?
  You \textbf{simplify the model}.

  \begin{figure}
    \centering
    \begin{tikzpicture}
      % include the original figure
      \node[inner sep=0] (img) {\def\svgwidth{0.3\textwidth}%
        {\small \import{./figures/goldakProfile}{profile.pdf_tex}}};

      % red "forbidden" cross over the whole image (hidden on overlay 1)
      \uncover<2>{
        \draw[line width=2pt, red] ([shift={(-2mm,-2mm)}]img.south west)
                                   -- ([shift={(+2mm,+2mm)}]img.north east);
        \draw[line width=2pt, red] ([shift={(+2mm,-2mm)}]img.south east)
                                   -- ([shift={(-2mm,+2mm)}]img.north west);
      }
    \end{tikzpicture}
    \caption{Double ellipsoidal profile \citep{goldak1984}.
      $\xi$ is the welding direction.}
    \label{fig:goldak}
  \end{figure}

  \uncover<2>{
    We give up on the accurate representation of the heat source
    since it is too costly to resolve.

    We use a \textbf{lumped heat source} instead.
  }
\end{frame}

\begin{frame}
  \frametitle{Part-scale simulation}
  \framesubtitle{Lumped heat source}
  A time-step is chosen regardless of the heat source path.
  Elements that are intersected by the heat source path
  during the time-step are \textbf{heated uniformly}.
  \begin{figure}
    \centering
    \begin{subfigure}[t]{0.495\textwidth}
      \centering
      \def\svgwidth{\textwidth}
      \import{./figures/lumped_hs}{heated_els.pdf_tex}
      \caption{Elements heated uniformly during $[t^{n},\,t^{n+1}]$.\\
           \begin{tikzpicture}
             \draw[red, very thick, dashed] (0,0.3) -- (0.4,0.3);
           \end{tikzpicture}
           $\; \rightarrow \;$ Heat source path
         }
      \label{fig:lhs_heated_els}
    \end{subfigure}%
    \hfill%
    \begin{subfigure}[t]{0.48508\columnwidth}
      \centering
      \includegraphics[width=\textwidth]{lumped_hs/tem.png}
      \caption{Resulting temperature field.}
      \label{fig:lhs_temp_field}
    \end{subfigure}
    \caption{Lumped heat source example.}
  \end{figure}
\end{frame}

\begin{frame}
  \frametitle{Part-scale simulation}
  \framesubtitle{Lumped heat source}
  \begin{figure}
    \begin{subfigure}[t]{0.32\textwidth}
      \centering
      \includegraphics[width=\textwidth]{chiumenti2017-lhs/chiumenti2017multipleHatches.png}
      \caption{Multiple hatches per time-step.}
    \end{subfigure}%
    \hfill
    \begin{subfigure}[t]{0.32\textwidth}
      \centering
      \begin{tikzpicture}[
        spy using outlines= {connect spies}
      ]
        \node[inner sep=0] (img) {\includegraphics[width=\textwidth]{chiumenti2017-lhs/chiumenti2017singleLayer.png}};
        \coordinate (spypoint) at ($(img.south west)!.88!(img.south east)!0.325!(img.north east)$);
        \coordinate (magnifypoint) at ($($(img.south west)!.5!(img.south east)$)!.2!($(img.north west)!.5!(img.north east)$)$);

        \only<2->{
          \spy [black, width=2cm, height=1cm, magnification=2.5] on (spypoint) in node[fill=white] at (magnifypoint);
        }
      \end{tikzpicture}
      \caption{Single layer per time-step.}
    \end{subfigure}%
    \hfill
    \begin{subfigure}[t]{0.32\textwidth}
      \centering
      \includegraphics[width=\textwidth]{chiumenti2017-lhs/chiumenti2017multipleLayers.png}
      \caption{4 layers per time-step.}
    \end{subfigure}
    \caption{Lumped heat input strategies from \cite{chiumenti2017b}.}
  \end{figure}
  \vspace{-5mm}
  \textit{Pros and cons}
  \begin{itemize}
    \item[+] Feasible simulations
    \uncover<2->{\item[-] Distributed heat input $\;\longrightarrow\;$ temperatures below melt $\;\longrightarrow\;$ Numerical calibration (flash heating)}
    \uncover<2->{\item[-] Loss of history: no resolution of heat source motion}
  \end{itemize}
\end{frame}
