\begin{figure}[H]
  \vspace{-1.0cm}
  \centering
  \begin{subfigure}[t]{0.425\textwidth}
    \centering
    \includegraphics[width=\textwidth]{{ss_robin/prev_sol.png}}
    \caption{Previous solution $T^{n}$}
    \label{fig:robin_prev}
  \end{subfigure}%
  \hspace{1.5cm}
  \begin{subfigure}[t]{0.425\textwidth}
    \centering
    \includegraphics[width=\textwidth]{{ss_robin/predictor.png}}
    \caption{$\tilde{T}^{n+1}$ after predictor step}
    \label{fig:robin_predictor}
  \end{subfigure}\\
  \begin{subfigure}[t]{0.425\textwidth}
    \centering
    \begin{tikzpicture}
      \node[anchor=south west, inner sep=0] (A) {\includegraphics[width=\textwidth]{{hodge/micro_step.png}}};
    \begin{scope}[x={(A.south east)},y={(A.north west)}]
      % Rectangle highlight (coordinates between 0 and 1)
      %\draw[teal!20, step=0.01] (0,0) grid (1,1);%grid, comment out for draft
      \draw[red, thick, dash pattern=on 1pt off 1pt, rounded corners] (0.0625, 0.450) rectangle (0.2, 0.5);
      \draw[red, thick, dash pattern=on 1pt off 1pt, rounded corners] (0.2500, 0.5505) rectangle (0.3100, 0.7100);
    \end{scope}
    \end{tikzpicture}
    \caption{Intermediate \textbf{Dirichlet} substep. The Dirichlet
    condition does not let the solution deviate from the predictor.}
    \label{fig:hodge_substep}
  \end{subfigure}%
  \hspace{1.5cm}
  \begin{subfigure}[t]{0.425\textwidth}
    \centering
    \begin{tikzpicture}
      \node[anchor=south west, inner sep=0] (A) {\includegraphics[width=\textwidth]{{hodge/macro_step.png}}};
    \begin{scope}[x={(A.south east)},y={(A.north west)}]
      % Rectangle highlight (coordinates between 0 and 1)
      %\draw[teal!20, step=0.01] (0,0) grid (1,1);%grid, comment out for draft
      \draw[red, thick, dash pattern=on 1pt off 1pt, rounded corners] (0.065, 0.4550) rectangle (0.1905, 0.4905);
      \draw[red, thick, dash pattern=on 1pt off 1pt, rounded corners] (0.2600, 0.5605) rectangle (0.3000, 0.7000);
    \end{scope}
    \end{tikzpicture}
    \caption{\textbf{Dirichlet} corrector step. Note the leftover temperature peaks from the predictor.}
    \label{fig:hodge_corrector}
  \end{subfigure}\\
  \begin{subfigure}[t]{0.425\textwidth}
    \centering
    \includegraphics[width=\textwidth]{{ss_robin/micro_step.png}}
    \caption{Intermediate \textbf{Robin} substep. Compared to the Dirichlet substep, the solution is free to diffuse away from the predictor.}
    \label{fig:robin_substep}
  \end{subfigure}%
  \hspace{1.5cm}
  \begin{subfigure}[t]{0.425\textwidth}
    \centering
    \begin{tikzpicture}
      \node[anchor=south west, inner sep=0] (A) {\includegraphics[width=\textwidth]{{ss_robin/macro_step.png}}};
    \begin{scope}[x={(A.south east)},y={(A.north west)}]
      % Rectangle highlight (coordinates between 0 and 1)
      %\draw[teal!20, step=0.05] (0,0) grid (1,1);%grid, comment out for draft
      \draw[red, thick, dash pattern=on 1pt off 1pt] (0.286, 0.5475) circle (0.02);
      \draw[red, thick, dash pattern=on 1pt off 1pt] (0.286, 0.6385) circle (0.02);
    \end{scope}
    \end{tikzpicture}
    \caption{\textbf{Robin} corrector step. Note the discontinuities in the solution.}
    \label{fig:robin_corrector}
  \end{subfigure}
  \caption{Examples of Dirichlet and Robin substepping cycles.}
  \label{fig:substepin_examples}
\end{figure}
