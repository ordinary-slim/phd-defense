\begin{frame}
  \frametitle{Substepping}
  \framesubtitle{Example time-step}
  \begin{figure}
    \begin{subfigure}[t]{0.48\textwidth}
      \centering
      \includegraphics[height=0.6\pageheight]{{ss_robin/prev_sol.png}}
      \caption{$T^{n}$}
      \label{fig:robin_prev}
    \end{subfigure}%
    \hfill%
    \begin{subfigure}[t]{0.48\textwidth}
      \centering
      \includegraphics[height=0.6\pageheight]{{ss_robin/predictor.png}}
      \caption{$\tilde{T}^{n+1}$ after predictor step}
      \label{fig:robin_predictor}
    \end{subfigure}
    \caption{Previous solution and predictor step.}
  \end{figure}
\end{frame}

\begin{frame}
  \frametitle{Substepping}
  \framesubtitle{Example time-step}
  \begin{figure}
  \begin{subfigure}[t]{0.48\textwidth}
    \centering
    \includegraphics[height=0.6\pageheight]{{ss_robin/micro_step.png}}
    \caption{Intermediate \textbf{Robin} substep.}
    \label{fig:robin_substep}
  \end{subfigure}%
  \hfill%
  \begin{subfigure}[t]{0.48\textwidth}
    \centering
    \begin{tikzpicture}[
        spy using outlines= {connect spies}
      ]
      \node[anchor=south west, inner sep=0] (A) {\includegraphics[height=0.6\pageheight]{{ss_robin/macro_step.png}}};
    \begin{scope}[x={(A.south east)},y={(A.north west)}]
      % Rectangle highlight (coordinates between 0 and 1)
      %\draw[teal!20, step=0.05] (0,0) grid (1,1);%grid, comment out for draft
      \draw[red, thick, dash pattern=on 1pt off 1pt] (0.286, 0.5475) circle (0.02);
      \draw[red, thick, dash pattern=on 1pt off 1pt] (0.286, 0.6385) circle (0.02);

      \coordinate (spypointrobinex) at (0.286, 0.62);
      \coordinate (magnifyglassrobinex) at (0.9, 0.5);
      \spy [black, width=1cm, height=2cm, magnification=2] on (spypointrobinex) in node[fill=white] at (magnifyglassrobinex);
    \end{scope}
    \end{tikzpicture}
    \caption{\textbf{Robin} corrector step.}
    \label{fig:robin_corrector}
  \end{subfigure}
  \caption{Robin-Robin substepper. Note the discontinuity in the solution
    at the interface $\Gamma$.}
  \end{figure}
\end{frame}

\begin{frame}
  \frametitle{Substepping}
  \framesubtitle{Example time-step}
  \begin{figure}
    \begin{subfigure}[t]{0.48\textwidth}
      \centering
      \begin{tikzpicture}
        \node[anchor=south west, inner sep=0] (A) {\includegraphics[height=0.6\pageheight]{{hodge/micro_step.png}}};
      \begin{scope}[x={(A.south east)},y={(A.north west)}]
        % Rectangle highlight (coordinates between 0 and 1)
        %\draw[teal!20, step=0.01] (0,0) grid (1,1);%grid, comment out for draft
        \draw[red, thick, dash pattern=on 1pt off 1pt, rounded corners] (0.0625, 0.450) rectangle (0.2, 0.5);
        \draw[red, thick, dash pattern=on 1pt off 1pt, rounded corners] (0.2500, 0.5505) rectangle (0.3100, 0.7100);
      \end{scope}
      \end{tikzpicture}
      \caption{Intermediate \textbf{Dirichlet} substep. The Dirichlet
      condition does not let the solution deviate from the predictor.}
      \label{fig:hodge_substep}
    \end{subfigure}%
    \hfill%
    \begin{subfigure}[t]{0.48\textwidth}
      \centering
      \begin{tikzpicture}
        \node[anchor=south west, inner sep=0] (A) {\includegraphics[height=0.6\pageheight]{{hodge/macro_step.png}}};
      \begin{scope}[x={(A.south east)},y={(A.north west)}]
        % Rectangle highlight (coordinates between 0 and 1)
        %\draw[teal!20, step=0.01] (0,0) grid (1,1);%grid, comment out for draft
        \draw[red, thick, dash pattern=on 1pt off 1pt, rounded corners] (0.065, 0.4550) rectangle (0.1905, 0.4905);
        \draw[red, thick, dash pattern=on 1pt off 1pt, rounded corners] (0.2600, 0.5605) rectangle (0.3000, 0.7000);
      \end{scope}
      \end{tikzpicture}
      \caption{\textbf{Dirichlet} corrector step.}
      \label{fig:hodge_corrector}
    \end{subfigure}
    \caption{Dirichlet substepper of \cite{hodge2021}.}
  \end{figure}
\end{frame}

\begin{frame}
  \frametitle{Substepping}
  \framesubtitle{No heat source predictor}

  \begin{figure}
    \begin{subfigure}[t]{0.33\textwidth}
      \centering
      \includegraphics[width=\textwidth]{hodge/to_predictor.png}
      \caption{Predictor step without heat input}
      \label{fig:to_hodge_predictor}
    \end{subfigure}%
    \hfill
    \begin{subfigure}[t]{0.33\textwidth}
      \centering
      \includegraphics[width=\textwidth]{hodge/to_micro_step.png}
      \caption{Micro-step.}
      \label{fig:to_hodge_micro}
    \end{subfigure}%
    \hfill
    \begin{subfigure}[t]{0.33\textwidth}
      \centering
      \includegraphics[width=\textwidth]{hodge/to_macro_step.png}
      \caption{Corrector step.}
      \label{fig:to_hodge_macro}
    \end{subfigure}
    \caption{Example of Dirichlet substepper step using a predictor step without including the source term.}
    \label{fig:to_hodge}
  \end{figure}
\end{frame}
