% “This slide summarizes my main technical contributions; I won’t go through
% them one by one, but they fall into three themes: moving-domain techniques,
% substepping algorithms, and their integration for part-scale LPBF.”
\subsection{Contributions}
\begin{frame}
  \frametitle{Contributions}

  \begin{itemize}
    \item The \textbf{advected subdomain} method, introducing a moving
    subdomain attached to the heat source in its reference frame, with
    support for MPI distributed-memory architectures.

    \item A complementary \textit{steadiness workflow} that adaptively
    increases the time-step size based on monitored solution steadiness.

    \item A synthesis of existing substepping approaches for LPBF,
    identifying common patterns, practical guidelines, expected speed-ups,
    accuracy trade-offs, and open challenges.

    \item A novel \textbf{Robin--Robin substepping} scheme using custom
    Robin coefficients inspired by \cite{chen2014robin, canuto2019}, achieving
    mesh-independent convergence behavior.

    \item New predictor-step strategies for substepping, motivated by
    time-scale separation, improving accuracy for the
    \cite{hodge2021} substepper.

    \item Integration of the advected subdomain method within the
    substepping framework, demonstrating compounded speed-ups in
    favorable scenarios.
  \end{itemize}
\end{frame}

\subsection{Conclusions}
\begin{frame}
  \frametitle{Thesis conclusions}

  \begin{itemize}
    \item Part-scale LPBF thermal simulation is fundamentally limited by
    extreme local time-scale disparities; exploiting time-scale separation
    is essential to achieve computational tractability without sacrificing
    local accuracy.

    \item The advected subdomain method bypasses time-step
      restrictions and improves accuracy close to the heat source,
      but it is not robust to complex scan paths
      involving direction changes and track overlaps.

    \item Substepping provides a robust and flexible framework to localize
      temporal resolution; we believe it will become a standard tool in
      FEM-based LPBF simulation.

    \item Substepping is a staggered Domain Decomposition method
      truncated after the first iteration;
      as long as the underlying DD scheme is well chosen,
      accuracy is preserved.

    \item First order time integration in the slow partition
      limits the maximum macro-step size.

    \item Ballpark figures for achievable speed-ups with substepping
      are in the range of \(2\times\) to \(5\times\)
      for fast/slow DOF ratios between \(20\%\) and \(5\%\).
  \end{itemize}
\end{frame}

\subsection{Future works}
\begin{frame}
\frametitle{Future works}
\begin{itemize}
  \item Improve the robustness of the advected subdomain:
    \begin{itemize}
      \item Smooth the transition between non-advected and advected
      regimes via ALE.
      \item Systematically assess when the thermal field is adapted
      to an advected subdomain time-step.
    \end{itemize}
  \item Extend substepping methodologies:
    \begin{itemize}
      \item Improve understanding and characterization of the time-scales
        at which the different physical processes operate.
      \item Higher order time-integration in the slow partition.
      \item Improve fast/slow partition to avoid heuristics (current).
      \item Explore alternative coupling structures and time-interpolation
        schemes within the substepping framework.
      \item Improve parallel scalability.
    \end{itemize}
\end{itemize}

\end{frame}

\begin{frame}
\frametitle{Thank you!}
  \begin{figure}
    \video<1>[above,height=0.7,aspectratio=734/422,fit=fill,autoplay]
  at (0.5,0.18) {videos/thx.mp4}
    \vspace{0.7\pageheight+3pt}
    \videoCaption{
      Robin substepper thanks you!
    }
  \end{figure}
\end{frame}
