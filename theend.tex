\begin{frame}
  \frametitle{Substepping + advected subdomain}
  \framesubtitle{Final conclusions}
  \begin{itemize}
  \item Fast/slow DOF ratio was identified as the primary driver of
    speed-up, with \(\Delta t_s / \Delta t_f\) playing a secondary role;
    overly large macro-steps can enlarge the fast domain and erode gains.
  \item The proposed Robin–Robin scheme
    achieved accuracy comparable to
    \cite{hodge2021}. Both clearly outperformed Dirichlet–Neumann coupling.
  \item Extra staggered iterations in the substepping loop brought negligible accuracy benefits.
  \item The main accuracy limiter for large macro-steps was the slow-region
    time integrator (Backward Euler)
  % \item Diffusion-focused predictors for the substepper improved stability and
  %   accuracy relative to source-driven predictors, especially at large
  %   macro-steps.
  % \item
  %   An enthalpy-based latent-heat treatment proved substantially more
  %   robust than the apparent heat capacity method for powder-to-solid
  %   transitions.
  % \item Calibration against melt pool experiments required
  %   an unphysical absorptivity dip, indicating that important heat-loss
  %   physics—likely tied to melt pool flow and evaporation—is missing from
  %   the thermal model.
  % \item Overall, advected subdomains and substepping
  %   together form a flexible toolbox that makes high-fidelity, part-scale
  %   LPBF simulations more computationally feasible while retaining local
  %   accuracy near the heat source.
\end{itemize}
\end{frame}

\begin{frame}
\frametitle{Substepping + advected subdomain}
\framesubtitle{Future works}
\begin{itemize}
  \item Improve the advected subdomain by using spatially and temporally varying advection fields (e.g.\ ALE-based fading and ramped velocities) and adaptive deactivation when thermal gradients misalign with scan direction.
  \item Enhance substepping via better characterization of LPBF time-scales, higher-order time integration for the slow partition, improved predictors/interpolation, adaptive fast-domain definitions, and dynamic load balancing.
\end{itemize}

\end{frame}

\begin{frame}
\frametitle{Thank you!}
  \begin{figure}
    \video<1>[above,height=0.7,aspectratio=734/422,fit=fill,controls]
  at (0.5,0.18) {videos/thx.mp4}
    \vspace{0.7\pageheight+3pt}
    \videoCaption{
      Robin substepper thanks you!
    }
  \end{figure}
\end{frame}


